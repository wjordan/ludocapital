This dissertation is a contribution to a theory of ludocapitalism, understood as the incorporation of the concept of game-playing into capitalism in contemporary technoculture. The term is derived from the Latin word \emph{ludus}, meaning \emph{game} or \emph{play}. By this compound term, I mean to describe a hybrid or transitional moment of capitalism that describes its processes of commodity production and capital accumulation through reference to play as a central concept of human activity and social organization, superseding the concept of work as the locus of rationality in traditional capitalist labor formations. Through a (post)humanistic study of the discourses and practices of software and game development viewed as paradigmatic instantiations of conditions of ludocapitalism, I develop an approach to engage these practices in ethical-political discourse derived from the critical theory tradition.

In this dissertation, I relate this dialectic of play to classical discourses of modernity in order to draw comparisons, contrasts and historical transitions from the liberal human subject of the Enlightenment to a technoliberal, posthuman subject prompted by the computerization of society and the expansion of digital play. A central question of this dissertation is whether and in what modified form traditional discourses of critical theory can maintain relevance in this techno-ludic context. Through an analysis of the ambiguity of play and creativity in discourses of game design and software development, and as a complement to textual and rhetorical theories of digital media, I advance a critical approach to digital game and software studies, attentive to the ambivalent potential of specific technical-social platforms upon which these new forms of media are constructed, with an aim to advance and reconfigure the conditions of ludocapitalist society toward better sustaining our collective forms of life.
