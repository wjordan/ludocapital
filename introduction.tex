\chapter*{Introduction}
\addcontentsline{toc}{chapter}{Introduction} \markboth{INTRODUCTION}{}

This dissertation is a contribution to a theory of \emph{ludocapitalism}, understood as the incorporation of the concept of game-playing into capitalism in contemporary technoculture. The term is derived from the Latin word \emph{ludus}, meaning \emph{game} or \emph{play}. By this compound term, I mean to describe a hybrid or transitional moment of capitalism that describes its processes of commodity production and capital accumulation through reference to play as a central concept of human activity and social organization, superseding the concept of work as the locus of rationality in traditional capitalist labor formations. Through a (post)humanistic study of the discourses and practices of software and game development viewed as paradigmatic instantiations of conditions of ludocapitalism, I develop an approach to engage these practices in ethical-political discourse derived from the critical theory tradition.

In contemporary information society, play is displacing work as the primary mode
of capitalist organization. Creativity, construed as a valuable commodity mined
from the intangible cultural resources of everyday life, displaces the
protestant, modernist work ethic with a playful hacker ethos of the information
age. As \citeauthor{Wark2007-ya} illustrates in \citetitle{Wark2007-ya}, the celebrated player of digital gaming
enthusiastically expands into the contemporary subject of the universal
gamification of everyday life, marking a transformation from the labor-leisure
distinction within industrial capitalism to the hybrid institutions of
ludocapitalism: work-as-play environments structured to accumulate creative
capital produced by new classes of professional knowledge-workers, and
play-as-work commodity forms designed to extract surplus social and cultural
value from mass entertainment player-consumers. However,
the same concept of play that provides contemporary capital its progressive
ludic dynamism also preserves a critical potential for disruptive play that can
escape and dismantle such fixed constructions. Ludocapitalism deploys a
technical infrastructure and social ideology that contains the seeds of its own
undoing within the same rhetoric of play that it embraces and depends on for its
own progressive survival.

In this dissertation, I relate this dialectic of play to classical discourses of modernity in order to draw comparisons, contrasts and historical transitions from the liberal human subject of the Enlightenment to a technoliberal, posthuman subject prompted by the computerization of society and the expansion of digital play. A central question of this dissertation is whether and in what modified form traditional discourses of critical theory can maintain relevance in this techno-ludic context. Through an analysis of the ambiguity of play and creativity in discourses of game design and software development, and as a complement to textual and rhetorical theories of digital media, I advance a critical approach to digital game and software studies, attentive to the ambivalent potential of specific technical-social platforms upon which these new forms of media are constructed, with an aim to advance and reconfigure the conditions of ludocapitalist society toward better sustaining our collective forms of life.

\section*{Subjects and Theories}
Before I proceed to outline the specific historical trajectories of this
analysis, I will first further define the key theoretical terms and
methodological assumptions that comprise my approach. By the \emph{liberal human
subject}, I have in mind a broad, familiar constellation of political-economic
and social-theoretical positions emerging in the Enlightenment period that link
together liberal concepts of freedom, autonomy and humanity to secular ideals of
rational thought, public reason, private ownership, and democratic
self-governance. This subject assumed its modern philosophical form starting
with Descartes and developed in the German idealism of Kant, Fichte and Hegel,
was notably challenged and problematized within Marx's critique of liberal
political economy and Adorno and Horkheimer's dialectic of myth and
enlightenment, and was decentered within Foucault's poststructuralist dispersion
into historically-specific fields of discursive formation. Alongside this
familiar sequence of movements in the discourse of subject-formation, I also
include Schiller's post-Kantian ideals of game-playing as particularly central
to the formation of the liberal human subject in relation to contemporary
ludocapitalism, a position I will expand upon in \cref{play}.

For the \emph{posthuman} subject, I look both to Mark Poster's thematic of the mode of
information in the context of electronically mediated communication,\footnote{
	\citeauthor{Poster1984} developed his thematic of the mode of information from the mid 1980s as
	an extension of Foucault's concept of discourse, within the contemporary context
	of \blockcquote[168]{Poster1984}{new forms of social interaction based on
		electronic communications devices}.
} and to Katherine Hayles's studies of the transformation of the liberal human subject of
possessive individualism within the field of cybernetics\footnote{
	Hayles's analysis of the posthuman in Weiner's cybernetic theory is measured
	against a liberal humanist subject marked by possessive individualism:\blockquote{To elucidate the significant shift in underlying
		assumptions about subjectivity signaled by the posthuman, we can recall one of
		the definitive texts characterizing the liberal humanist subject: C. B.
		Macpherson's analysis of possessive individualism. >Its possessive quality is
		found in its conception of the individual as essentially the proprietor of his
		own person or capacities, owing nothing to society for them.…The
		human essence is freedom from the wills of others, and freedom is a function of
		possession.< The italicized phrases mark convenient points of departure for
		measuring the distance between the human and the posthuman}\autocite*[3]{Hayles1999-de}.}
as my initial two points of departure. However, as neither Hayles nor Poster
devote much attention to either digital games or the broader topic of
game-playing within their theoretical apparatus, my aim is to extend their
theories of digital subjectivity in a way that places them in closer dialog
with the field of game studies. For this reason, I prepend the term
\emph{technoliberal} to my account of the posthuman under conditions of
ludocapitalism. This is a term introduced by Thomas \citeauthor{Malaby2011-my} in \citetitle{Malaby2011-my}, used to indicate emerging
cultures of game design that expand the naturalization of market forces in
liberal economic theory to a heterogeneous embrace of contrived complex systems
throughout all aspects of social life:
\blockcquote[133]{Malaby2011-my}{While Adam Smith conceived of a market that was
	in a way a natural and ineradicable part of the landscape,…and
	neoliberal thought continues to see the market in this way, technoliberalism
	holds up the idea that such complex systems can be contrived, in their entirety.
	The liberal component is the imagined freedom of individuals to perform as such
	within designed systems, generating collective effects that are thereby
	legitimate.}
Finally, my references to >critical theory< specifically invoke the
ethical-political\footnote{
  While I use the term \emph{ethical-political} in a general sense to refer to the humanistic orientation of critical theory, I also refer in a specific sense to \citeauthor{Habermas1987-yd}'s concept of a concrete, historically-situated discourse concerning "who we are and who we seriously \emph{want to be}" \autocite*[180]{Habermas-bfn}, as an "affirmation of a form of life in light of critically appropriated traditions" \autocite[163]{Habermas-bfn}. Derived from Hegel's \emph{sittlichkeit} (ethical life), Habermas offers a fragmented, less nationalistic concept suitable for partial, overlapping collective identities: "A pluralism in the ways of reading fundamentally ambivalent traditions has sparked a growing number of debates over the collective identities of nations, states, cultures, and other groups. Such discussions make it clear that the disputing parties are expected to consciously choose the continuities they want to live out of, which traditions they want to break off or continue. To the extent that collective identities can develop only in the fragile, dynamic, and fuzzy shape of a decentered, even fragmented public consciousness, \emph{ethical-political discourses} that reach into the depths have become both possible and unavoidable" \autocite[97]{Habermas-bfn}.
  } orientation of the Frankfurt School tradition, in order to
provide a coherent, emancipatory focus to my project that is largely absent
within the fields of software and game studies. I understand the term in the
restricted sense assigned to it by Horkheimer, summarized by \citeauthor{Poster1984} as a
discourse that >attempts to promote the project of emancipation by furthering what it understands as the theoretical effort of the critique of domination begun by the Enlightenment and continued by Karl Marx< \autocite*[1]{Poster1989-pt}. In his 1937 essay \citetitle{Horkheimer1972-nu}, \citeauthor{Horkheimer1972-nu} develops the concept of critical theory by distinguishing
>two ways of knowing: one is based on |Descartes's| Discourse on Method, the other on Marx's critique of political economy< \autocite[244]{Horkheimer1972-nu}. In contrast to the \textcquote[197]{Horkheimer1972-nu}{traditional idea of theory |which| is based on scientific activity as carried on within the division of labor at a particular stage in the latter's development}, the idea of theory Horkheimer identifies as "critical"\footnote{
  \textcquote[206n14]{Horkheimer1972-nu}{The term is used here less in the sense it has in the idealist critique of pure reason than in the sense it has in the dialectical critique of political economy}.
} is "a human activity which has society itself for its object" \autocite[206]{Horkheimer1972-nu}:
\blockcquote[210--1]{Horkheimer1972-nu}{
	Critical thinking is the function neither of the isolated individual nor of a
	sum-total of individuals. Its subject is rather a definite individual in his
	real relation to other individuals and groups, in his conflict with a particular
	class, and, finally, in the resultant web of relationships with the social
	totality and with nature. The subject is no mathematical point like the ego of
	bourgeois philosophy; his activity is the construction of the social present.
}
This distinction is central to my own theoretical approach in relation to
contemporary fields: throughout this dissertation I firmly position my
methodological assumptions, grounded in this form of critical theory, against
both those of a social science which would maintain at all costs the rhetorical
perspective of an objectively-neutral observer, as well as those of a literary
criticism which would dismiss any emancipatory project in favor of techniques of
rhetorical analysis or production of meaning within the bounds of established
order. The difficult task of preserving this humanist orientation of critical
theory and its "concern for reasonable conditions of life" \autocite[199]{Horkheimer1972-nu} within a positive vision of the posthuman that advances beyond the transcendental
assumptions of the modern human subject articulates a concept of
"post-humanities" as a field within which I situate my project.\footnote{
  \citeauthor{Goldberg2014-tb} outlines such an affirmative concept of the post-humanities in \citetitle{Goldberg2014-tb}: "By post-humanities I mean then not the end of the humanities, their death or demise, and so their post-mortem. Rather, I intend by this the posing of alternative modalities for taking up, for doing, for engaging (and for an engaging) humanities.…I am urging that this conception of a post-humanities encourage and embrace a reformulating of public reason, of what it amounts to and how vigorously to promote it."}

Let me further clarify this distinction with reference to two authors whose works have had a substantial influence upon my own. First, Luhmann's sociological systems theory is a monumental effort to reconstruct sociological theory in the general tradition of Weber and Parsons, and I largely understand both the unique contributions and the theoretical limits of my own project in relation to his work. In an intellectual progression running largely parallel to post-structuralist trends in contemporary critical theory, Luhmann abandons Parsons's adherence to action theory based on classical cybernetic systems theory, incorporating concepts of functional differentiation, self-reference and paradox, and eventually adopting the technical jargon of second-order cybernetics, to construct a grand theory of society based on communication as its primordial operation. Like the Frankfurt School, \citeauthor{Luhmann1994-qz} also rejects the classical, Cartesian "transcendental concept of the subject" as a "dead end way of thinking" \autocite*[xli]{Luhmann1995-et}, "deconstructs the very distinction between Subject and Object" \autocite*[135]{Luhmann1994-qz} that grounded traditional theory and, like Horkheimer, refuses to "give up the search for describing the unity of society" \autocite[127]{Luhmann1994-qz}, preserving society as his self-reflexive object by way of second-order observation.

However, despite its substantial and creative synthesis of a wide range of theoretical tools into a unified theoretical framework, I find that \citeauthor{Habermas-bfn}'s critique offers a crucial distinction between the premises of Luhmann's project and those of critical theory: to the extent that "subject-centered reason is replaced by systems rationality," systems theory "replaces metaphysical background convictions with metabiological ones" \autocite*[385]{Habermas1987-yd}. In other words, systems theory replaces the ethical-political engagements emerging from concrete social groups with a unified, rationalized world of systems emerging out of a scientific-technical consensus. Though sociocybernetics does, in theory, provide the tools to effect social change---in Luhmann's words, it "could create a surplus of structural variations that could induce the observed function systems to consider alternatives to their own modes of operation" \autocite*[136]{Luhmann1994-qz}, and "offer novel possibilities for observation" \autocite[138]{Luhmann1994-qz}---and it incorporates literary techniques such as parody and irony into its dynamic theory-construction, in practice, its scientific rigor (and impenetrably abstract prose) actively discourages the development of sustained critical projects within its purview, leaving it to preside over an elaborate systematic description of society as it is observed by the conservative organs of scientific method. In contrast to critical theory's central concern with constructing new techniques of social emancipation, resistance against injustice and sustaining concrete forms of public reason within an increasingly complex modern society, Luhmann's "methodological antihumanism" \autocite[378]{Habermas1987-yd} coldly maintains that society is nothing other than what is observed through sociology's neutral scientific apparatus: "Evolution is sufficient for survival" \autocite[Luhmann, qtd. in][377]{Habermas1987-yd}.

Second, I refer to Bogost's pioneering work of videogame criticism as a central point of departure throughout my project. I cite his work often not to challenge his claims so much as to signpost my own work, to note the various subtle but fundamental distinctions between my critical-theoretical position and his own mode of comparative criticism. Bogost's work has exerted a great deal of influence within software and game studies as well as the commercial videogame industry over the last decade. I will leave my concrete analysis of his work in relation to the specific topics of game-playing, intellectual property, and procedural literacy to the relevant moments in each chapter; for now, I will comment on his method of comparative criticism and his associated theory of unit operations.

The approach to criticism that \citeauthor{Bogost2006-ec} develops in \citetitle{Bogost2006-ec} is "fundamentally a comparative one" \autocite[ix]{Bogost2006-ec}, and is derived from his academic training within the field of comparative literature. Citing the American Comparative Literature Association's mission to study "the interactions between literature and other forms of human activity" \autocite[50]{Bogost2006-ec}, Bogost initially understands the role of criticism within his study of videogames as a primarily hermeneutical activity:
\blockcquote[53]{Bogost2006-ec}{
  Instead of focusing on how games work, I suggest that we turn to what games do---how they inform, change, or otherwise participate in human activity, to borrow the ACLA's words. Such a comparative videogame criticism would focus principally on the expressive capacity of games and, true to its grounding in the humanities, would seek to understand how videogames reveal what it means to be human.
}
Considering my similar training in comparative literature, I also find value in such a comparative approach. However, I find that in Bogost's work, this critical activity of revealing "what it means to be human" is largely elided in favor of developing the technique of comparison itself as a "general practice of criticism" \autocite[15]{Bogost2006-ec}, with intrinsic value. Comparative criticism is construed by Bogost as an autonomous, generic procedure of comparative analysis that offers to "uncover the discrete meaning-making in texts of all kinds" \autocite[15]{Bogost2006-ec}, a "useful lever of explication" \autocite[54]{Bogost2006-ec} with little regard for how any such activity relates to a humanist concern for reasonable conditions of life. In fact, the latent traces of humanism that marked Bogost's early work would later be extinguished entirely, by a defiant methodological antihumanism confirming his complete separation from humanistic concerns: "The philosophical subject must cease to be limited to humans and things that influence humans. Instead it must become \emph{everything}, full stop" \autocite[10]{Bogost2012-wr}.

Accordingly, I understand Bogost's theory of unit operations as a proto-sociocybernetic concept, akin to Luhmann's systems theory.\footnote{
  Note that Luhmann's concept of \emph{system} does not match Bogost's own use of the same term. Although Bogost focuses on a distinction between "unit operations" as "characteristically succinct, discrete, referential, and dynamic" and "system operations" as "characteristically protracted, dependent, sequential, and static" \autocite[4]{Bogost2006-ec}, Luhmann's concept of system in fact has much more in common with Bogost's concept of \emph{unit} than \emph{system}.}
Like the autopoietic systems of Luhmann's theory, Bogost's unit-operational systems construct meaning from self-referential relations among parts: "The difference between systems of units and systems as such is that the former derive meaning from the interrelations of their components, whereas the latter regulate meaning for their constituents" \autocite[4]{Bogost2006-ec}.\footnote{
  Compare to Luhmann's summary of autopoietic systems: "|E|verything which is used as a unit by the system is produced as a unit by the system itself. This applies to elements, processes, boundaries and other structures, and last but not least to the unity of the system itself" \autocite*[174]{Luhmann1986-kl}.
  }
Starting from the premise that "any medium,…can be read as a configurative system" \autocite[ix]{Bogost2006-ec} "may be observed in any artifact, or any portion of any artifact, rather arbitrarily" \autocite[14]{Bogost2006-ec}, providing the critic-observer with "a lever for understanding any form of human production as potentially procedural" \autocite[15]{Bogost2006-ec}. All forms of human expression are thus subsumed under the transcendental unit-operation concept: "stories are instances of unit-operational expression" \autocite[69]{Bogost2006-ec}; Cultural artifacts "implement their expression through unit operations" \autocite[73]{Bogost2006-ec}; Unit operations "function at a higher level than linguistic signs" \autocite[105]{Bogost2006-ec}; even the forms of culture and consciousness themselves are packaged into "cultural unit operations" \autocite[45]{Bogost2006-ec} and "psychic unit operations" \autocite[150]{Bogost2006-ec}.

I therefore distinguish my approach in relation to Bogost's unit operational comparative criticism along similar lines as Luhmann's systems theory, on account of their mutual refusal of the (post)human condition. In comparison to Habermas's guiding image of Luhmann's social systems as "the flow of official documents among administrative authorities" \autocite[378]{Habermas1987-yd}, or the image of Turing's ideal life as "the discrete state machine, communicating by teleprinter alone" \autocite[Hodges, qtd. in][xii]{Hayles1999-de}, I take the guiding image for \citeauthor{Bogost2006-ec}'s unit operations \parentext{>not only applicable to software in general and videogames in particular, but also<---as an afterthought--->poetry, literature, cinema, and art< \autocite[ix]{Bogost2006-ec}} to be a software engineer-cum-hacker, disassembling, reverse engineering, or just messing around with a console videogame system, such as the Atari 2600.\footnote{
  See \citeauthor{Montfort-rtb}'s \citetitle{Montfort-rtb}, a fascinating interdisciplinary study of the Atari 2600 platform across a variety of technical, cultural and artistic registers, for this particular image.
} For this reason, my relation to Bogost's comparative videogame criticism has additional significance to my project, reminiscent of Marx's generative relation to Smith, Ricardo and Mill: it condenses into an abstract, coherent theoretical form several universal principles and assumptions of contemporary ludocapitalism that comprise the object of my critical (post)humanist response---not as a simple rejection of the theoretical foundation of the software and game studies fields, but as a dialectical elaboration of their fundamental principles, theoretical concepts and objects of engagement into the domain of ethical-political human activity central to critical theory.

\section*{Method}
I break down the transitions characterizing contemporary technoculture into three analytical aspects: a management style, an economic paradigm, and a techno-linguistic infrastructure. The management style is \emph{game design}, an indirect technique of behavioral control through the procedural regulation of a social group or activity. The economic paradigm is \emph{ludocapitalism}---while related to such popular concepts as the New Economy \autocite{Kelly1999-zr}, friction-free capitalism \autocite{Gates1995-lq}, or post-industrial society \autocite{Bell1973-pis}, I wish to place particular emphasis on its contemporary affinities with digital games, virtual worlds and other ludic modes of contrived competition.\footnote{
  The term \emph{ludocapitalism} was coined by Julian \citeauthor{Dibbell2007-dd}, who speculates that "the economic system we inhabit must,…contrive new meanings for our daily grind" \autocite[298]{Dibbell2007-dd}. I will discuss Dibbell's work at length in \cref{bitcoin}. Malaby has also developed this argument in relation to his concept of technoliberalism mentioned earlier.
} Finally, the techno-linguistic infrastructure is \emph{software}, simultaneously comprised of an idealized sphere of perfectly-rational communication, and a material landscape of a complex, continuously-evolving sociotechnical literacy of codes, standards, languages and protocols.

Each of these aspects has been championed as an inversion of the historically prior paradigm it purports to replace---game design inverts hierarchical management of work; the friction-free capitalism of extra-sovereign, multinational corporations inverts that of industrial capitalism tied to state power and fixed, localized flows of capital; and software development and the digital communication of protected, controlled ideas, brands and virtual goods inverts hardware manufacturing and the physical delivery of material goods. However, I characterize these transitions not in terms of pure negation or replacement, but rather through the Hegelian concept of \emph{aufhebung} (sublation) or determinate negation: the prior elements are preserved and upheld in a synthesis that signifies more than merely the absence or opposite of the original elements. Sublation accounts for the novelty or newness of new media while preserving the history of the social forms they transcend. Beyond Hegel's historical term, this general notion of sublation can be found in Schumpeter's concept of "creative destruction" as well as McLuhan's pair of popular slogans, "the medium is the message" and "the content of a medium is another medium."\footnote{
  See Grosswiler's \citetitle{Grosswiler1996-ip} for a survey of the implicit affinities between McLuhan's method and dialectical theory.
} The old, historical paradigms of work, industrial capitalism, and hardware production are still essential components of the new forms of play, ludocapitalism, and software, all of which are only comprehensible in relation to their sublated elements. In this model, critical thought involves reaffirming the presence of the negated elements within the synthetic concept, so that its critique does not imply a conservative return to the original but a more nuanced consideration of alternative syntheses.

To take one example, the concept of work as productive, alienated industrial labor is often contrasted against the notion of a free-spirited, humanist culture of game-playing, an expression that originates with Fourier, and extends through Marcuse to contemporary theorists of digital-ludic utopia such as McGonigal and Castronova. Discourses of game-playing reinscribe alternative claims for rational and productive distributions of activity, such as education, practice or training for future vocation, under the guise of intrinsic motivation (fun, happiness and/or pleasure). \citeauthor{Marcuse1974-oy} writes: "The transformation of labor into pleasure is the central idea in Fourier's giant socialist utopia" \autocite[207]{Marcuse1974-oy}. The realization of such a harmonious ideal, however, is difficult to reconcile with necessity: "Jobs which Fourier describes as isolated and lacking attraction, such as staffing the watchtower or coach driving, >will be assigned to a few individuals whose temperaments are appropriate to such tasks, which they will transform into games<" \autocite[37]{Granter2012-lo}. As shown here, when the productive capacities of such so-called "games" become aligned with the objectives of pragmatic institutions (such as government administration, school systems, or business organizations), they may retain little in common with the liberating ideals of play that they depend on for their progressive legitimacy. \citeauthor{Marcuse1974-oy} negates this negation of work transformed into play, commenting: "Work as free play cannot be subject to administration; only alienated labor can be organized and administered by rational routine" \autocite[208]{Marcuse1974-oy}, a sentiment Marx also held: "Work cannot become a game, as Fourier would like it to be" \autocite[qtd. in][66]{Granter2012-lo}. We can then draw a distinction between (serious) games designed to direct the intrinsic motivation of games towards fixed objectives such as, and (free) play irreducible to rational social administration.

Further, it is not sufficient to examine each of these transitional aspects of ludocapitalism in isolation, but they must be understood in relation to one another in their contemporaneity. Each of these three aspects can also be said to follow what Edwards calls a \emph{mutual orientation}:\footnote{
  The term \emph{mutual orientation} is adapted from \citeauthor{Edwards1997-df}'s \citetitle{Edwards1997-df}, where he describes the mutual orientation between MIT's Whirlwind project and the Air Force around 1950: "The source of funding, the political climate, and their personal experiences oriented Forrester's [Whirlwind] group toward military applications, while the group's research eventually oriented the military toward new concepts of command and control" \autocite[82]{Edwards1997-df}. He summarizes the concept as "each guiding the other's conception of research problems and potential solutions" \autocite[222]{Edwards1997-df}.
} each presupposes and takes for granted the natural reality of each other, to the point that cause and effect among them is difficult to determine in isolation. The aspects mutually \emph{coevolve}, in the sense that Hayles describes the co-evolution of technical artifacts and humans,\footnote{
  See \citeauthor{Hayles2005-ss}'s \citetitle{Hayles2005-ss}: "These two dynamics---the continuing development of intelligent machines and the shifting meanings of key terms---work together to create a complex field of interactions in which humans and intelligent machines mutually constitute each other. Neither kind of entity is static or fixed; both change through time, evolution, technology, and culture. In other words, to use an aphorism that cultural materialists have long realized as a truth of human culture: what we make and what (we think) we are coevolve" \autocite[216]{Hayles2005-ss}.
} or in the sense that Luhmann describes, following Parsons, as \emph{interpenetration} of various functionally differentiated subsystems of society.\footnote{
  See \citeauthor{Luhmann1995-et}'s \citetitle{Luhmann1995-et}: "We use the concept of >interpenetration< to indicate a specific way systems with a system's environment contribute to system formation.…Interpenetration is not a general relation between system and environment but an intersystem relation between systems that are environments for each other" \autocite[213]{Luhmann1995-et}; "The concept of interpenetration does not indicate merely an intersection of elements, but a reciprocal contribution to the selective constitution of elements that leads to such an intersection" \autocite[215]{Luhmann1995-et}.
}

Game design, as a paradigmatic management style, presupposes the deployment of complex software systems capable of processing the near-instantaneous circuits of individualized feedback necessary to produce valuable results. Software development, in turn, presupposes the New Economy ideology of a natural, global, frictionless "marketplace of ideas" to provide legitimacy for the expansive, global intellectual property protections that make the most profitable returns on large-scale software production possible. Ludocapitalism presupposes game design as a decentralized system of control that produces its own extra-sovereign legitimation, where multinational power can be safely deposited and invested beyond the political influence and boundaries of the nation-state.

Finally, these mutually-oriented transitions converge in a pervasive culture of \emph{computationalism}\footnote{
  David \citeauthor{Golumbia2009}'s \citetitle{Golumbia2009} defines the term \emph{computationalism} as "a commitment to the view that, a great deal, perhaps all, of human and social experience can be explained via computational processes" \autocite[8]{Golumbia2009}. I read this as an extension of Hayles's concept of the "Regime of Computation" \autocite*[ch.~1]{Hayles2005-ss}, expressed from a more critical-leftist ideological perspective.
} as a twentieth-century form of rationalism and the autonomous embodiment of knowledge, against which I argue in support of open discourses that recognizably relate to public interests. The progression of this culture can be summarized by comparing the following quotes from two famous figures in computing history: the first is from Ada Lovelace, who collaborated with Charles Babbage in 1843 to publish a series of notes on the Analytical Engine: "The Analytical Engine has no pretensions whatever to \emph{originate} any thing. It can do whatever we \emph{know how to order it} to perform" \autocite*[722]{Lovelace1843-yi}.\footnote{
  In Turing's seminal article \citetitle{Turing1950-qu} that introduced the "imitation game" as an approximation to the question: "Can machines think?" \autocite[442]{Turing1950-qu}, he popularized (and readily dismissed) Lovelace's comments as "Lady Lovelace's Objection" \autocite[450]{Turing1950-qu}. In light of Turing's assumption that "|t|here is no theoretical difficulty in the idea of a computer with an unlimited store" \autocite[438]{Turing1950-qu}, it is worth considering Lovelace's pragmatic sensibility immediately preceding her oft-quoted lines: "It is desirable to guard against the possibility of exaggerated ideas that might arise as to the powers of the Analytical Engine. In considering any new subject, there is frequently a tendency, first, to overrate what we find to be already interesting or remarkable; and, secondly, by a sort of natural reaction, to undervalue the true state of the case, when we do discover that our notions have surpassed those that were really tenable" \autocite*[722]{Lovelace1843-yi}.
} The next is from John von Neumann, reportedly given at a 1948 talk in response to an audience member questioning the ability of a machine to think: "You insist that there is something a machine cannot do. If you will tell me precisely what it is that a machine cannot do, then I can always make a machine which will do just that!" \autocite[qtd. in][7]{Jaynes2003-wa}. Although both make a similar point about the theoretically unlimited performative abstraction of computing machines, the shift from Lovelace's negative qualification to von Neumann's positive hubris reflects a broad transition in the hegemonic ideology of computing machines that was applied to universal descriptions of reality in the twentieth century. While Lovelace emphasized the machine's practical inability to transcend its constrained, order-following subservience, von Neumann emphasized the machine's theoretical superiority as always a mere matter of programming away from the execution of pure thought.

This rationalist position thrived in early 20th century positivism, surviving in some mainstream fields of the modern empirical social sciences, including cognitive science and behavioral psychology. Such disciplines attempt to equate human thought with scientific logic, presenting formalized, computational models of human behavior that can be implemented on computing machines. Any task a machine can't already perform is presented as a theoretical possibility, a "not yet"---a solution not yet described in a sufficiently formalized language; not yet executed on a computer with sufficient memory or processing resources, not yet presented with sufficient environmental inputs.

The primary problem with such arguments in favor of whether computers can >think< is not only that it elides the philosophical impossibility of perfectly articulating the entirety of conscious thought into formal procedure (one can never describe exactly what one is thinking), or that it ignores the material time and space costs of pragmatically executing any such procedure on a concrete machine (an algorithm might require a computer with as many bits of memory as there exist atoms in the universe running for a million trillion years to complete its computation), but that such theoretical zeal for artificial intelligence conceals the political stakes of control over the manufacture of such intelligence. If (or when) aspects of human thought are reproduced by a machine, control of the machine will amount to control over those aspects of humanity.

I frame my critique of the computationalist narrative that runs through this project as a twentieth-century refraction of the classical Marxist argument against the myopic deskilling of physical labor in service of monolithic technological progress, now taking the form of the deskilling of intellectual labor, or knowledge work. Following critical theory and Marx's attitude toward praxis (succinctly stated in \citetitle{Marx2002-md}: >Philosophers have hitherto only \emph{interpreted} the world in various ways; the point is to \emph{change} it< [11]\footnote{
  Compare to Luhmann's emphatic comment on systems theory that "There is no Eleventh Thesis in Parsons!" \autocite*[129]{Luhmann1994-qz}.
}), the final methodological element of this project explicitly moves beyond descriptive systems analysis to modes of ethical and political action guided by the critical axes outlined above. I frame such action through a mix of locally-oriented languages of resistance, ambivalence and contingency, derived from, among others, Foucault's analysis of power relations through forms of resistance, Feenberg's Marxist-inspired concept of the ambivalence of technology, and Coombe's adoption within critical legal studies of a postmodern "ethics of contingency" respectively.\footnote{
  On Foucault's analysis of power relations and forms of resistance, see \autocite*[780]{Foucault1982-bg}. On Feenberg's concept of ambivalence, see \citetitle{Feenberg1990-tw}. On Coombe's ethics of contingency, see \autocite[297--9]{Coombe1998-yv}.
}

As a meta-narrative of transition, ludocapitalism is reminiscent of earlier periodizations of capitalist development such as post-modern \autocite{Lyotard1984-kb}, late \autocite{Jameson1990-zi}, post-Fordist \autocites{Amin1994-wf}{Hall1988-ob}, and post-industrial \autocite{Bell1973-pis}, all of which have been productively applied to critical theories of digital media. However, I believe that the connotations of game-playing within this key term can promote a ludic turn in critical theory through a more direct dialog with the nascent field of digital game studies, which in turn can lead to better accounts of contemporary modes and struggles of cultural creation and social organization, both within software and game studies in particular and throughout digital media discourse in general. My dissertation will develop a range of such accounts within the conceptual framework of ludocapitalism I have set out above. Next, I will present a cross-section of categories under which I have organized this body of research, and through which I will relate these concrete studies back to my overall critical project.

\section*{Project outline}
The following four categories (game-playing, property, literacy, and money) present a vocabulary of critical terms and a geography of confrontations for the public significance of a range of ludocapitalist artifacts in contemporary media environments. Each of these key terms resonates with the liberal humanist subject of modernity, but is transformed through conditions of ludocapitalism into altogether different forms of technical and institutional struggle:

\begin{enumerate}
  \item{
    \emph{Game-playing}: the gamification of business and culture threatens to reinscribe traditional hierarchies of power through the institutions regulating behavior through techniques of procedural persuasion, pitting a ludic Utopia of well-designed games against a resistant, radical free play escaping such game objects through aesthetic affirmations of an inexhaustible (post)human subject.
    }
  \item{
    \emph{Property}: the commodification of creativity confronts the radical openness of freely shared culture, and the privatization of knowledge impinges upon cultures of orality leading to the enclosure of social commons, reduced freedoms of public speech, and increasingly transactional social and political relations biased toward oligopolistic market structures.
    }
  \item{
    \emph{Literacy}: languages accessible and accountable to civil society confront proprietary and expert-oriented technical codes constraining everyday communication and public expression.
    }
  \item{
    \emph{Money}: decentralized computational networks indirectly governed through technological protocol confront regulated political spaces of collective self-governance.
    }
\end{enumerate}

This cross-section of categories of the liberal humanist subject is undergoing social-historical transformations into the technoliberal posthuman subject of contemporary technoculture, ushering in the game-like political-economic system I have characterized as ludocapitalism. Each category corresponds to a chapter in my dissertation, where I develop the abstract transitions outlined here in relation to concrete case studies of particular game/software projects and social movements within the contemporary technocultural landscape.

\section*{Chapter outlines}
\emph{\cref{play}} develops a critical concept of game-playing, taking a critical perspective on modern applications of the "gameful" theories of game studies to such disparate fields as education, corporate training, advertising, politics, and other institutions of modern everyday life. I analyze the rhetoric of gamification and serious games, which range from viewing everyday social institutions from a ludic perspective to the total view of life itself as a game or a game-playing world as a form of Utopia, exemplified in the work of McGonigal and Suits. As such gameful theory commands increasing popular attention, it is also beginning to amass significant cultural capital within academic thought and fast-growing game design research centers fueled by game-industry collaborations, such as MIT's game lab collaboration with Singapore's media industry.

I place gameful theory in a historical context, comparing its Utopian ideals and universalization of the concept of game-playing to a strikingly similar rhetoric of play that motivated the German idealist Schiller's liberal aesthetic ideology. Schiller's philosophy is founded upon a concept of the "play-drive" [\emph{spieltrieb}] that elevates play to the transcendental constitution of humanity. Following de Man and Warminski's critical commentary, I read Schiller's philosophy as a humanization of Kant's earlier concept of "freeplay" [\emph{freies Spiel}], one that closes off the paradoxical unboundedness of the concept of play in favor of a particular aesthetic worldview conceived as a totality. Reading Schiller's play-drive as a ludic model of classical liberal humanism with Kant's freeplay maintaining critical resistance to its totalization, and drawing parallels between Schiller's play-drive and McGonigal's gamefulness, we can begin to imagine a resistance to the aesthetic totalization of digital game-playing by distinguishing such a concept from the contemporary equivalent of Kant's freeplay.

I develop what such Kantian resistance looks like in contemporary game-playing discourse by identifying the dominant ideology underlying modern game studies, which I find in Malaby's concept of technoliberalism as faith in the free-market manipulation of technology such as software tool creation or digital game design to solve social problems. Through an account of the entrepreneurial virtual world of Second Life and the parallel organization of its Silicon Valley startup company Linden Lab, I analyze how the rhetoric of universal creative freedom in Second Life's digital frontier is in fact contrived through particular technological and social developments that mark the boundaries of play possible in a particular entrepreneurial model of the digital subject envisioned by the company. I contrast this entrepreneurial vision against the creative anarchy taking place around the Minecraft computer game and its proliferation of unofficial mods and add-ons. Minecraft has its own limitations and unanswered questions regarding the uncertainty of its future leadership and its contrived freeplay; however, it demonstrates an alternate and unprecedented vision of a creative subject produced by digital game. The question of digital freeplay, as the creative human potential that resists being captured in our concept of well-designed digital games, is not one that submits to a definitive answer, but such questioning is produced through comparative studies of communities of open-ended creative play.

\emph{\cref{tetris}} traces the ironic, spectral history of the iconic game brand Tetris. Through this history, I explore how creative play under capitalism became commodified as objects of intellectual property, and how game design developed from a communal activity of social becoming to an authored process, a singular manifestation of universally protected ideas of individual genius.

The puzzle videogame Tetris, from its origins in cold war USSR where private intellectual property rights were virtually nonexistent and where circulation was as vital to cultural value as conception, to its global status as one of the most celebrated and litigated video game properties of all time. In the game's simple concept of falling tetromino-shaped blocks collecting in a rectangular glass, the protected design selectively abstracted to absurdity is separated from its material history of creation and distribution, leaving behind a silenced cultural history of file-sharing and platform adaptations, design and technical variations, and communities of passionate players whose creative efforts are unceremoniously absorbed into the singular corporate mass of protected intellectual property.

A close reading of the arguments presented in a recent copyright and trade dress infringement case won by The Tetris Company against a small iPhone game startup reveal the fundamental ambiguities and contradictions in the history of modern intellectual property law that has served to commodify a concept of singular, creative genius at the expense of the cultural and social environment that participates in the production of such creativity. Adopting a discourse of critical legal studies advocating shifts in the interpretation of intellectual property laws in favor of a more balanced consideration of the various social and cultural factors at work in creative expression, I argue that works of mourning that preserve the ephemeral memories of the silenced specters haunting intellectual property's objects can contribute to a discussion of how best to preserve and promote the cultural commons increasingly enclosed by the corporate commodification of creativity in a digitally designed world.

\emph{\cref{literacy}} examines the expansion of computational infrastructure from its military-industrial origins to its present-day democratic tool of open exchange. In particular, I interrogate the issue of "procedural literacy" as a form of new media pedagogy, and the politics of expertise that guides control of the communication infrastructures comprising digital civil society, as a topic of increasingly public policy and debate. In my analysis, I focus on the relations between the discourse of digital languages and code literacy, the increasing oligarchical enclosure and ownership of the technical infrastructure of mass communications media by information technology corporations, and the administration and self-conception of the liberal public sphere in civil society.

The theoretical work in this chapter attempts to distinguish valid calls for new mechanisms of participation in a reconstituted and revitalized public sphere from more restricted calls for technical mastery in service of controlling technocratic interests. I argue that a critical concept of literacy can be distinguished from a more vocational notion of mastery through the emancipatory political potential found in the liberal concept of the public sphere, which in order to be legitimate should be held accountable to an open, inclusive, democratic discursive process that validates the knowledge and experience of its various communities. In terms of procedural literacy, I argue that this process requires an appreciation of several points commonly ignored in rhetorics of mastery: an ethics of complexity in relation to a code's subjects, a morality of code that recognizes accounts of injustice and inequality, and a critique of operational efficiency that considers additional values such as readability and flexibility in addition or in place of technical execution.

Taking these points into account in relation to the code literacy of today, I argue that a critical code literacy should not be presented as a set of training techniques to deliver mastery of today's technical languages and vocabularies to an educational mass-market, but must be conceived as an ongoing project of self-fashioning digital environments that can accommodate the varied discourses of creative subjects, through structures of inclusive, collaborative design more democratic than the autocratic or oligarchic control underlying most open-source software projects or programming language specifications. Such a literacy project favors the creation of new technical vocabularies and formations of new structures of knowledge before and above the education of digital neophytes in the use of existing professional technical tools.

Taking the JavaScript programming language as an object of critique, I explore the institutional interests underlying its design evolution, and use a model of design viscosity to examine the barriers in place inhibiting local community adaptations that favor a univocal representation of procedural knowledge. I present two programming language projects, DrScheme and Processing, as exhibiting alternative structures of procedural literacy. Through a comparative analysis and appreciation of the contributions these language designs offer, I conclude this chapter by offering the practice of end-user programming language design itself as a form of critical code literacy that demands greater attention by those calling for a more inclusive and democratic digital public sphere.

\emph{\cref{bitcoin}} studies the decentralized virtual currency project, Bitcoin, as an archetypal, ideal type of money in ludocapitalist economy. Following a comparative analysis of metallism and cartalism in classical modern theories of money, I proceed to read Bitcoin as a form of "play money," placing it in the context of game currencies within virtual world economies first studied by Castronova and expanded by Dibbell into a general theory of ludocapitalism. Despite the similarities to game economies, however, the differences between traditional game-economy designs and Bitcoin makes for a very different type of money-play incomprehensible as game-playing according to either economic rational-actor game theory or theories of virtual world game design, and it is precisely its incomprehensibility that makes it a generative object of study from a more open-ended ludic perspective.

Next, I consider the multivalent identity of Bitcoin along various hermeneutic perspectives: as a technical software project, as a political ideological statement, as a speculative fiction, as authored text, as performance, as money, and as a financial asset or investment vehicle. The discourse of Bitcoin cuts across all of these disciplinary boundaries while questioning and transforming classical understandings of each, leading me to consider it in terms of a Foucauldian initiation of discursive practice.

Rather than enthusiastically accept or emphatically reject Bitcoin's premises and promises of the future of money, I take a balanced appreciation of its role as a sustained experimental intervention in economic discourse, albeit one that harbors a strong techno-libertarian ideology embedded in the project's software-centric social organization. Amid the varied interpretations of Bitcoin and speculations about its future, I consider how to sustain critical discourse through such a ludicrous technological project by tracking how communities are thinking through Bitcoin's popularity to generate alternative forks, adaptations, and clones of the distributed currency system as an emerging mode of technological commentary and discursive struggle over future meanings of money. Beyond Bitcoin itself, I read the broader field of crypto-currencies both competing against and derived from Bitcoin as a diaspora of competing protocological language-games, each enacted through the self-governing consensus of the machinic voices of its committed participants. Guided by this analysis, I offer an interpretation of protocol as play that challenges Galloway's reduction of protocol to a "physical logic" without alternative that can only be countered through hypertrophy-inducing exploits within the protocol itself. Rather, the field of play money as produced by the Bitcoin diaspora encourages us to read political and social implications within protocol design itself, recognizing such debates as an emerging, experimental form of public policy operating in an extra-governmental realm of networked imaginary.

\section*{Summary}
This study contributes to the fields of software and game studies a critique of the forms of cultural hegemony the conditions of ludocapitalism impose on software and game production, and that is reinscribed within politically-neutral social-scientific and systems-theoretic studies. Rejecting the unified rationality of game rules, intellectual property, technical expertise and fungible money pointing to computation as the ground of being and circumscribing the modern subject of contemporary ludocapitalism, this study offers alternative dialectical categories of game-playing, ownership, literacy, and wealth that are instead produced through social and historical tensions and struggle, tethered neither to a fixed human essence nor to a naturalized technological inheritance. What is at stake in such a reconstructed (post)human subject is the crucial link from software and game criticism to political and ethical deliberations on the impact of the conditions of ludocapitalism on our collective forms of life, and the ability of such a renewed public reason to influence the basic terms and tenets of our technological future through effective ongoing public policy.