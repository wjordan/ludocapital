\chapter{The Wealth of Computations: Bitcoin and Play Money}
\label{bitcoin}
\epigram{Only our limited idea of money is keeping us poor... \autocite{Boyle99}}
\subsection*{Introduction}
In this chapter, I examine the enduring power and novel forms of money emerging in ludocapitalism, in particular through a critical study of Bitcoin, a decentralized virtual currency that has developed from worthless play money in 2009 into a global market capitalizing upon over 10 billion USD in exchange value by 2014. Framing the currency as an intersection of modern money grounded within a material metallism and the ethereal fictions of digital play money, I argue that Bitcoin has become the catalyst of a wide-ranging referendum on what money is, can and should be in contemporary technoculture, with a discursive significance that extends far beyond the rapidly-fluctuating prices of its particular units of currency. As such, a study of the many meanings of Bitcoin provides us with an economic reflection of the multifaceted spirit of ludocapitalism grounded in an unparalleled faith in computation, as well as possible avenues for critical reflection and intervention in the form of alternative economic experimentation.

\section{Money and Modernity}
The conceptual category of money, what we understand to be money and how we value, exchange, accumulate, desire and fetishize it, both reflects and shapes the organization and ideologies of the social environments structuring everyday life. In this first section, I outline how classical modern concepts of money as the universal equivalent have reflected and shaped its corresponding ideologies of democratic nation-state governance and its liberal human subject.

\subsection*{Marx's Money Materialism}
I begin my analysis of modern money with a reading of Marx's theoretical analysis of the political-economic form of money. My claim is that this theory depends on an essentially materialist understanding of money that links the natural physical qualities and historical exchange of precious metals, particularly gold, with the social embodiment of universal value within its substance, a process that provides the necessary preconditions for capitalism to emerge. In \citetitle{MarxCapital}, \citeauthor{MarxCapital} introduces the concept of money through an abstraction from the "material substance" of increasingly complex forms of economic exchange, the final product of which comprises "the first form of appearance of capital":
\blockquote{If we abstract from the material substance of the circulation of commodities, that is, from the exchange of the various use-values, and consider only the economic forms produced by this process of circulation, we find its final result to be money: this final product of the circulation of commodities is the first form in which capital appears.…|W|e have no need to refer to the origin of capital in order to discover that the first form of appearance of capital is money. We can see it daily under our very eyes. \autocite[ch.~4]{MarxCapital}}
As the primary representation of capital, this modern form of money forms the economic subject of classical liberalism. In contrast to economic theorists such as Ricardo from whom Marx derives many technical aspects of his economic analysis, Marx frames his Hegelian dialectic of money with an oppositional goal of critical-ironic subversion: the liberal-economic worldview comprising a "very Eden of the innate rights of man," where "There alone rule Freedom, Equality, Property and Bentham" \autocite[ch.~6]{MarxCapital} is an abstraction and extension of classical political-economic theory which Marx claims misinterprets the historical formation of the capitalist mode of production as universal laws of nature. It conceals the inequality of class struggle produced by the presumptive egalitarianism of free exchange and the free market, where the purchase of labor power and ownership of the means of production generate surplus value for the capitalist. Marx's emphasis, on the other hand, is on the historical contingency of the capitalist mode of production, its dynamism and tendencies towards overproduction, expansion and crisis.

The extreme emphasis which Marx placed on his theory of money as a material history of precious metal, in opposition to contrasting theories of money which emphasize the ideal functions of money independent of any material, is a reflection and constitutive part of this underlying emphasis and overall critical-philosophical approach. In \citetitle{MarxCPE}, \citeauthor{MarxCPE} includes the following complex passage on gold which emphasizes its role as both a concrete, material, physical substance and an abstract symbol of universal wealth:
\blockcquote[ch.~2.3]{MarxCPE}{
  |I|n its simple metallic corporeality gold is money or money is real gold.…Gold is the material aspect of abstract wealth in contradistinction to commodities which only represent the independent form of exchange-value, of universal social labour and of abstract wealth.…|W|hereas the prices of commodities represent gold, the universal equivalent or abstract wealth, the use-value of gold represents the use-values of all commodities. Gold is, therefore, the material symbol of physical wealth. It is the >epitome of all things< (Boisguillebert), the compendium of social wealth. As regards its form, it is the direct incarnation of universal labour, and as regards its content the quintessence of all concrete labour. It is universal wealth in an individual form. Functioning as a medium of circulation, gold suffered all manner of injuries, it was clipped and even reduced to a purely symbolical scrap of paper. Its golden splendour is restored when it serves as money. The servant becomes the master. The mere underling becomes the god of commodities.
}
In this passage, which I read as a concise summary of Marx's statement on the function of precious metal in his theory of money, I note two key insights condensed into the "simple metallic corporeality" of gold as the material embodiment of abstract wealth. First, the "direct incarnation of universal labour" is attributed to the metal as a representation of the generally stable but fluctuating labour necessary to mine and process gold from a natural element of the earth's crust into a valuable commodity.\footnote{
  Compare to a similar passage from \citetitle{MarxCapital}: "These objects, gold and silver, just as they come out of the bowels of the earth, are forthwith the direct incarnation of all human labour. Hence the magic of money" \autocite[ch.~2.]{MarxCapital}.
}
Second, the Hegelian allusions at the end point to Marx's dialectical claim that gold historically emerged as a "universal equivalent" due to its natural material properties that made it most suitable as a generalized medium of exchange, but as a result is now being sublimated into its own symbol, replaced in practice by debased coins and "worthless" tokens of representative value issued by national governments.\footnote{
  Marx articulates this dialectic in greater detail in an earlier section of the same text: "In the same way as the exchange-value of commodities is crystallised into gold money as a result of exchange, so gold money in circulation is sublimated into its own symbol, first in the shape of worn gold coin, then in the shape of subsidiary metal coin, and finally in the shape of worthless counters, scraps of paper, mere tokens of value. But the gold coin gave rise first to metallic and then to paper substitutes only because it continued to function as a coin despite the loss of metal it incurred. It circulated not because it was worn, but it was worn to a symbol because it continued to circulate. Only in so far as in the process of circulation gold currency becomes a mere token of its own value can mere tokens of value be substituted for it" \autocite[ch.~2.2.c]{MarxCPE}.
}

Marx's dialectical approach to understanding of the significance of the symbolic representation of wealth in the form of debased coins and paper money is a crucial, often misunderstood element of his material theory of money, and a key point of contention among opposing theories still current in mainstream economic thought today. In order to draw out the significance of Marx's emphasis on gold as the material aspect of abstract wealth, I will next examine the general argument of Knapp's contrasting state theory of money, which I position as an idealism alongside Marx's materialism that together circumscribe the classical liberal-humanist discourse of modern money.

\subsection*{Metallism and Cartalism}
A broad group of early twentieth-century economic theorists including Knapp, Keynes and Schumpeter have characterized an emphasis on precious metals as the foundation of money such as Marx's theory exhibits as reflecting an outdated "metallism." Against this view, they offer the concept of "c|h|artalism,"\footnote{
  Although Knapp originally coined the concept with the spelling "Chartal," it is often spelled "cartal" in the subsequent economics literature so I adopt this latter spelling except when quoting Knapp directly.
} a modern, juridical ideal of money conceived as the distribution and management of abstract quantities of state-administered debt. The concept originates in Georg \citeauthor{Knapp1924}'s \citetitle{Knapp1924}, and although I believe that Knapp's original metallism/cartalism distinction risks reductively denouncing the material significance of precious metal in favor of government-issued paper currency as the exemplary ideal of modern money, I also find its legal-historical theory of money to be a good point of reference for identifying the intersection of economic exchange and nation-state organization within classical liberal economic theory.


With his stated purpose to "replace the metallistic view by one founded on Political Science" \autocite[viii]{Knapp1924}, Knapp begins his book with a fundamental juridical thesis: "Money is a creature of law. A theory of money must therefore deal with legal history" \autocite[1]{Knapp1924}: "Through its Courts of Law the State gives a right of action for debt" \autocite[11]{Knapp1924}. However, "Historical experience" reveals that rather than maintain debts in terms of a specific quantity of material, "The State always maintains only the relative amount of debts, while it alters the means of payment from time to time" \autocite[13]{Knapp1924}, suggesting a position where the State "meant only the name of the former unit without attaching any importance to the material of which it was imposed" \autocite[14--5]{Knapp1924}. Knapp calls such abstract, material-independent debts accounted by State ">nominal< debts" \autocite[15]{Knapp1924}, arguing that this forms a necessary precondition for the modern concept of money: "The nominality of debts and of the unit of value is a necessary premise before money can come into being" \autocite[19]{Knapp1924}; "So long as a given material is \emph{per se} a means of payment, money has not yet come into being" \autocite[25--6]{Knapp1924}. He next supplements nominal debts with the concept of "morphic means of payment," a second necessary condition for a system of money where "Our law lays it down that only pieces formed in such and such a manner are to be admitted as means of payment" \autocite[27]{Knapp1924}. The result is a "Chartalist" theory of money that denies any significance of metal in its ideal form:
\blockcquote[30--2]{Knapp1924}{
  |T|here is nothing to prevent us from giving to the morphic means of payment a validity dependent not on weight but on fiat.…The validity can depend on proclamation.…|W|e need another short name for "morphic proclamatory" means of payment, the metallic contents of which are of no importance for validity. At least they are movable objects which have in law a significance independent of their substance.…Perhaps the Latin word "Charta" can bear the sense of ticket or token.…Our means of payment have this token, or Chartal, form.
}
Schumpeter, who harbored a sustained, skeptical interest in Marx's economic analysis, labeled Marx the quintessential metallist. Though Schumpeter himself characterized the basis of Knapp's non-metallist theory as an "absurd claim" that was "in almost complete ignorance of both the literature and the logic of the subject" \autocite*[1057]{Schumpeter1954-gi}, he nonetheless adopted and extended Knapp's terminology to include "theoretical" and "practical" subcategories of metallism and cartalism, which he used to classify the positions of economic analysts according to their theories of money:
\blockcquote[274--5]{Schumpeter1954-gi}{
  By Theoretical Metallism we denote the theory that it is logically essential for money to consist of, or to be >covered< by, some commodity so that the logical source of the exchange value or purchasing power of money is the exchange value or purchasing power of that commodity, considered independently of its monetary role.…
  
  By Practical Metallism we shall denote sponsorship of a principle of monetary policy, namely, the principle that the monetary unit >should< be kept firmly linked to, and freely interchangeable with, a given quantity of some commodity. Theoretical and Practical Cartalism may best be defined by the corresponding negatives.
}
Schumpeter finds theoretical metallism common in the century after Smith, and "by nobody more implicitly than by Marx" \autocite[276]{Schumpeter1954-gi}. Schumpeter takes "for granted that theoretical metallism is untenable," as it relies upon "a confusion between the historical origin of money,…and its nature or logic---which is entirely independent of the commodity character of its material" \autocite[276]{Schumpeter1954-gi}. As I noted above, however, Marx in fact articulates this connection between the historical origin of money and its nature not as a confusion, but as an explicitly dialectical movement essential to the development of its modern logic. For this reason, I characterize metallism as a materialism, contrasted against the idealism of cartalism.

To the extent that proponents of cartalism rest on the essential premise that "modern money is state money" \autocite[77]{Tcherneva08}, Marx's metallism can be interpreted not just as a practical revolutionary resistance to state-controlled economic policy, but also as a theoretical concept of modern money that looks beyond the nation-state's juridical ordering of money as authorized means of debt payment. By asserting a material history of money in relation to the broader social totality of a global capitalism not restricted to the confines of official state policy, Marx's metallism is perhaps most significant and relevant at the margins of governmental order, where trust in the stability and security of the nation-state falters in periods of political-economic crisis or revolution. Marx's theory of money maintains its distance from the internal dynamics of state-led finance, as he carefully distinguishes an "internal sphere of circulation of commodities, which is circumscribed by the boundaries of a given community and separated from the universal circulation of the world of commodities" \autocite*[ch.~2.2.c]{MarxCPE}.

\subsection*{Toward Postmodern Money}
As Marx points toward in this concept of the universal "world of commodities" juxtaposed against the state's internal sphere, globalization has become a crucial general category in the transition from industrial capitalism to the landscape of multinational corporations characteristic of ludocapitalism. In addition to globalization, I will next touch upon two additional theoretical lines of departure leading from the classical modern monetary theories of metallism and cartalism into an analysis of money within ludocapitalism. These aspects are that of formalization and spiritualization, which I derive primarily from the work of Simmel.

First, I wish to further distinguish formalization in terms of its ideal and material elements, etherealization and rationalization. Ideal formalization describes a progressive abstraction or removal from physical embodiment, an abstraction I thematize as \emph{etherealization} following McLuhan's use of the term. Here, the cartalist paradigm of paper-based money containing value backed by the legal force of a stable and powerful government serves as the backdrop, through which the material of money becomes increasingly insignificant compared to its ideal, symbolic form. As global financial institutions and information and communications technology have continued to grow increasingly sophisticated and powerful, even the "worthless" paper form of money has steadily dissolved into increasingly ethereal forms of monetary payments such as credit card and, more recently, mobile phone and Internet-based money transfer systems.

\citeauthor{Simmel04} noted this progression in \citetitle{Simmel04}, describing a "persistent trend towards the transformation of money into a purely symbolic representative of its essential function" \autocite[191]{Simmel04}, a trend that would only intensify throughout the twentieth century. In his characteristic tone of breathless prophecy, \citeauthor{McLuhan64} cited such a transformative etherealization of physical money as a corollary to the transcendence of Marxian labor-value by the "movement of information": "As work is replaced by the sheer movement of information, money as a store of work merges with the informational forms of credit and credit card. From coin to paper currency, and from currency to credit card there is a steady progression toward commercial exchange as the movement of information itself" \autocite[137]{McLuhan64}. Such a "steady progression" towards an ideal money of pure information flow extends the cartalist ideal of a legal regime of printed money into its information-age equivalent. However, the process by which the universal labor-value constitutive of material money transitions into its digital equivalent is here left untouched, as credit card balances or bank account statements are only new symbolic representations of value, not its material embodiment: even in digital form, a contract for an ounce of gold's worth of money is equivalent in value to a real ounce of gold insofar as the debtor is trustworthy and the contract is upheld by the State's "right of action for debt," to refer back to Knapp's original cartalism.

Simmel's concept of formalization also incorporates a critical account of rationalization, eventually leading to some promising revisions to Marx's metallist position as well as a critique of the ideal juridical thesis of reified state power. Extending Marx's theory of commodity fetishism in which a social relation between people appears within the commodity as a material relation between things, \citeauthor{Simmel04} suggests that the increasing penetration of market-based exchange into everyday life tends toward a modernity in which quality is replaced by quantity and subjective social relation replaced by objective value, producing the formalized basis of a rational style of life:
\blockcquote[448--50]{Simmel04}{
  This measuring, weighing and calculating exactness of modern times,…seems to me to stand in a close causal relationship to the money economy,…|which| enforces the necessity of continuous mathematical operations in our daily transactions. The lives of many people are absorbed by such evaluating, weighing, calculating and reducing of qualitative values to quantitative ones.…Money expresses,…the purely commercial element in the commercial treatment of things, just as logic represents comprehensibility with reference to comprehensible objects. Since the abstract form that represents the immanent value of objects takes the form of arithmetical precision and thus of unequivocal rational accuracy, its characteristics must reflect upon the objects themselves.
}
Second, alongside this critique of modern intellectuality produced by the money economy, Simmel offers an alternative theory of value derived from an aesthetic concept of subjective distance which I call a \emph{spiritualization} of money. Here, the tone is hopeful rather than tragic, and hinges upon a methodological opposition to Marx's historical materialism that Simmel outlines in his preface to the second edition of his book:
\blockcquote[54]{Simmel04}{
  The attempt is made to construct a new storey beneath historical materialism such that the explanatory value of the incorporation of economic life into the causes of intellectual culture is preserved, while these economic forms themselves are recognized as the result of more profound valuations and currents of psychological or even metaphysical preconditions.\footnote{
    Notably, in a published self-advertisement for his own book Simmel used the term \emph{spirit} [Geist] to describe these currents: "I extend the claim of historical materialism, which allows all forms and contents of culture to emerge out of the prevailing economic relations, by evidence that the economic valuation and movements are, for their part, the expression of more deeply lying currents of individual and societal spirit [Geist]" \autocite[qtd.~in][526]{Frisby04}.
  }
}
In opposition to Marx's theory of value as universal labour power, Simmel offers a broader account of the forms of social relations that produce value, an argument he frames as replacing the economic concept of use-value or "utility" by a "desire for the object" \autocite[88]{Simmel04}:
\blockcquote[93--4]{Simmel04}{
  The idea,…that the essential feature of value is the socially necessary labour time objectified in it,…does not answer the question of how labour power itself became a value.…According to this theory, if price and value diverge, one contracting party exchanges a quantity of objectified labour power against a smaller quantity; but this exchange is affected by other circumstances which do not involve labour power, such as the need to satisfy urgent wants, whims, fraud, monopoly, etc.,…it is always the interrelation of demands, realized in exchange, that gives economic value to objects.
}
Simmel therefore suggests a "profound relationship between relativity and socialization" \autocite[99]{Simmel04}, arguing that "economic value,…resides exclusively in the reciprocal relationship arising between several objects on the basis of their nature" \autocite[99]{Simmel04}, a worldview that he theorizes in neo-Kantian fashion terms as the "relativity of truth" \autocite[114]{Simmel04}.  In this way, Simmel collapses the distinction between metallic money and credit, and by extension metallist and cartalist theories of money, into a unified theory of intersubjective exchange, or "the common relationship that the owner of money and the seller have to a social group---the claim of the former to a service and the trust of the latter that this claim will be honoured" \autocite[177]{Simmel04}. Finally, an additional element based in faith and belief in the religious sense gives this theory of money a distinctively spiritual element:
\blockcquote[178]{Simmel04}{
  |I|n the case of credit, of trust in someone, there is an additional element which is hard to describe: it is most clearly embodied in religious faith. When someone says that he believes in God, |it expresses| a state of mind which has nothing to do with knowledge, which is both less and more than knowledge.…Economic credit does contain an element of this supratheoretical belief, and so does the confidence that the community will assure the validity of the tokens for which we have exchanged the products of our labour in an exchange against material goods. |It| contains a further element of social-psychological quasireligious faith. The feeling of personal security that the possession of money gives is perhaps the most concentrated and pointed form and manifestation of confidence in the socio-political organization and order. The subjectivity of this process is, so to speak, a higher power of the subjectivity that creates the value of precious metals in the first place.
}
Simmel here assimilates both precious metal and credit money into a sociological theory of money as based on the relativity of truth within an exchanging community's collective trust informed by faith in its socio-political organization.

This theory of modern money as the representation in "congealed form" \autocite[175]{Simmel04} of an intersubjective spirit, combined with a critical view of the objectivity of monetary exchange producing a style of life that emphasizes formal calculation and rationalization, gestures beyond the categories of modern money established by classical political economists and their critics, arguably anticipating Weber's thesis on the spirit of capitalism.\footnote{
  Although Weber was somewhat critical of Simmel's work, \autocite{Appadurai12}'s close reading of Weber outlined in \citetitle{Appadurai12} suggests similar themes, especially contrasted against mainstream (neoclassical) economic thought. Inspired by Weber, Appadurai wants to "return to the idea of the >spirit< of capitalism" \autocite[7]{Appadurai12} in his contemporary project of a social study of finance, asking "what the link between >spirit< and >ethic< might be today" \autocite[8]{Appadurai12}. Noting Weber's unique concept of "magicality," Apparudai observes: "Today,…it is possible to identify a series of magical practices,…at the heart of global capitalism, and in particular, of the financial sectors. These practices are premised on a general, absolute and apparently transcendent faith in the market" \autocite[8--9]{Appadurai12}.
}
It also provides a basis for understanding and critiquing the changing role of money within ludocapitalism, including the understanding of economic exchange in terms of game theory and the corresponding legitimation of forms of "play money," which I turn to in the next section.

\section{Play Money}
In this section, I shift from theories of money in classical liberal political economic theory to the development of economic thought in relation to digital environments, particularly geographically-distributed economic communities connected through the Internet. In the second half of the twentieth century, the decline of nation-state economic sovereignty and the rise of multinational corporate power precipitated a paradigmatic shift in institutional organization from a society based on centralized, disciplinary power to a society based on decentralized systems of control, a correlation that \citeauthor{Deleuze95} explicitly identified in relation to contemporary shifts in national monetary policy:
\blockcquote[180]{Deleuze95}{
  Money, perhaps, best expresses the difference between the two kinds of society, since discipline was always related to molded currencies containing gold as a numerical standard, whereas control is based on floating exchange rates, modulations depending on a code setting sample percentages for various currencies.
}
As a corollary and contribution to this transitional narrative of money, I offer the concept of \emph{play money} as encompassing the forms and tensions of money within ludocapitalism. To illustrate this concept, I will begin with an account of play money within digital labor economies that spontaneously developed in virtual worlds in the early 2000s, followed by an in-depth case study of Bitcoin as a recent paradigm of money that I argue reflects the spirit of ludocapitalism more generally.

\subsection*{Play Money}
The historical usage of the term "play money" itself reflects the ambiguity within the concept of game-playing that I discussed in the first chapter. Historically, it often referred to money allocated for or obtained from "play" in the sense of gambling or gaming.\footnote{
  For example, in a British comedy from 1705: "Play-Money,…amongst People of Quality, is a sacred Thing, and not to be profan'd. The deux---'tis consecrated to their Pleasures, 'twould be Sacrilege to pay their Debts with it" \autocite{VanbrughConfed}.
}
Other times, it refers to money that is not materially "real" in some subjective sense of stability or value, such as an expired bank note or a government-issued currency that has been devalued by inflation, as contrasted against a precious metal such as gold.\footnote{
  For example, in an 1896 pamphlet defending the gold standard, an American railway authority related an anecdote about a box containing paper bills issued by various American banks around 1860, discovered twenty years later after they no longer held value: "All this was at one time thought to be real instead of play money" \autocite[9]{Kirkman1896}.
}
Finally, in contemporary usage the term most often refers to fictional or symbolic money intentionally designed to have minimal exchange value outside the magic circle of a game, such as the colorful paper bills used in a game of Monopoly.

In contemporary technoculture, play money in all three of these senses is confronted and complicated by the incorporation of game-playing into capitalism. First, as modern economic theory since game theory of von Neumann and Morgenstern has adopted models of strategic action, risk and competition through analogy to traditional gambling games, the once-clear distinction between play money earned through gambling and other earnings from strategic investment or speculation in other forms of game-like economic systems is eroding. Second (as Deleuze related in the quote above), following the collapse of the Bretton Woods system of international financial exchange in 1971, the world's most widely traded currencies are no longer linked together by fixed exchange rates negotiated in the political arena and convertible to gold. These fluctuating valuations, in addition to gold itself, are now all equally subject to currency speculation and the risk of being devalued into worthless play money. Finally, Internet-based virtual currencies have emerged that, while originating within their digital environments as purely fictional play money, nonetheless often develop observable real-world exchange rates.

Conversely, the forms of play money that arise within virtual worlds retain traces of all three senses of the term: its value is obtained from game-like conditions; their exchange values are subject to high amounts of speculation; and they are founded upon purely symbolic fictions with no previous substance or state authority grounding or backing their value. Next, I will examine the discourse of play money as it has been theorized within several academic discussions of virtual worlds, uncovering the seeds of an economic theory of ludocapitalism ambivalently linked to the modern theories of money discussed in the first section.

\subsection*{Ludocapitalism in Virtual Worlds}
Since Edward \citeauthor{Castronova01}'s widely-cited \citeyear{Castronova01} paper on the economics of virtual worlds, academic game studies researchers have taken a keen interest in the structured economies found in commercial multiplayer computer games. Castronova's paper, subtitled "A First-Hand Account of Market and Society on the Cyberian Frontier," exudes enthusiasm for virtual worlds representative of the extropian genre of techno-futurist euphoria: "To a large and growing number of people, virtual worlds are an important source of material and emotional well-being. Virtual worlds may also be the future of commerce, and perhaps of the internet itself" \autocite[3]{Castronova01}; "VWs |virtual worlds| may soon become one of the most important forums for human interaction, on a level with telephones. Moreover, in that role, they may induce widespread changes in the organization of Earth society" \autocite[37]{Castronova01}. The numerical analysis proposed by Castronova is the earliest attempt to quantify the labor market in a multiplayer commercial computer game from a basic economic perspective in terms of labor value and exchange rates. Through a survey administered to over three thousand Everquest players, including questions about the total value of all in-game virtual property and total number of avatar hours, Castronova calculated that the average avatar accumulated virtual net worth at a rate of 319 platinum pieces per hour, or \$3.42 US dollars per hour at the exchange rates found on second-hand virtual currency markets.

Following Castronova's provocative surveys, tech journalist Julian \citeauthor{Dibbell2007-dd} published a blog-turned-book, \citetitle{Dibbell2007-dd}, recording his public attempt to act as a full-time virtual entrepreneur in Ultima Online for a full year in 2003-2004, harvesting and trading virtual resources in exchange for national currency through eBay auction sales. Dibbell's project not only coins the term ludocapitalism, but also suggests the basis of an economic theory founded upon the economics of virtual currency and digital labor. His argument is tentative and ambiguous, but marks a crucial moment in a critical understanding of play money within ludocapitalism that I will expand upon through a close reading.

In the process of making (and paying taxes on) over 11 thousand dollars in profit over the course of his experiment, Dibbell was fascinated by players he encountered who would routinely spend hard-earned US dollars on secondary virtual goods markets, in exchange for virtual property and fictional gold to boost their avatars. He was equally interested in the vocational intensity of players toiling away in virtual farms and dungeons to produce the coveted items of status and power. Considering the famous "iron cage" metaphor of freedom-constraining rationalism in Weber's account of early twentieth-century industrial capitalism becoming increasingly divorced from its ascetic, Protestant origins, Dibbell asks, "If this iron cage was founded largely on an exaltation of work as everything that play is not---productive, rational, efficient---might we not find our way out of it in a countervailing exaltation of play?" \autocite[62]{Dibbell2007-dd}.

Here, Dibbell both assumes and questions the Protestant ideals of play as the inverse of its work ethic.\footnote{
  It is worth noting that this work ethic was not merely confined to personal moral opinion, but within the United States it extended throughout the legal institutions of the colonies through prohibitions on gaming. The language introducing a 1762 South Carolina "Act for the better preventing of excessive and deceitful gaming," for example, is common: "WHEREAS games and exercise should not be otherways used than as innocent and moderate recreations, and not as constant trades or callings to gain a living or make unlawful advantage thereby" \autocite[158]{Cooper1838}.
}
He finds his answer in Weber's brief comments on the American state of play at the start of the twentieth century: "In the field of its highest development, in the United States, the pursuit of wealth, stripped of its religious and ethical meaning, tends to become associated with purely mundane passions, which often actually give it the character of sport" \autocite[qtd.~in][298]{Dibbell2007-dd}. Deriving some hesitation from this dismissive attitude toward American proto-ludocapitalism, Dibbell backed away from an unqualified exaltation of play as freedom from Weber's iron cage of rationality. In his final analysis, he began to view the kind of activity taking place within virtual worlds as a contrived meaning, a saccharine layer of entertainment fiction built on top of an inescapable economic system binding the individual to wage labor:
\blockcquote[298--9]{Dibbell2007-dd}{
  Drained of the religious significance that gave it meaning, the economic system we inhabit must either bind us to its pointlessness against our wills,…or contrive new meanings for our daily grind. And what easier way is there of contriving meaningful activity than through the mechanisms of play? Add computers to the historical picture, effectively building those mechanisms into the technological foundation of the world economy, and the contriving gets so easy that it starts to look inevitable. The grind must sooner or later become a game. Call it a theory of ludocapitalism, and don't feel too obliged to take it seriously.
}
Dibbell's economic theory of ludocapitalism here contemplates the ludification of work with some much-needed ambivalence. He recognizes that the virtual world economies represent how the modern work economy is evolving to incorporate "mechanisms of play" into the structure of its "grind," its capitalist mode of production and accumulation, suggesting that the process is irreversible or "inevitable." If the separation between work and play is itself a contrived, ascetic remnant of industrial capitalism fueled by the Protestant faith, then this separation is gradually unraveling in a secular transformation into contemporary ludocapitalism where "the grind was already escaping from itself" \autocite[299]{Dibbell2007-dd}.

Dibbell's relation to Weber's condemnation of American proto-ludocapitalism is therefore somewhat conflicted. Weber lamented that the wealth-seeking sport of American proto-ludocapitalism was "stripped of its religious and ethical meaning," more of a von Neumannian game-theoretic situation avant la lettre than a Schillerian liberal-humanist play. On the one hand, Dibbell's theory of ludocapitalism welcomes "meaningful activity" produced by play inevitably becoming incorporated into the capitalist system of production. On the other hand, rather than enthusiastically embrace the productive capacity of play-fueled virtual economies, Dibbell also suggests that his year-long experience in virtual entrepreneurship did not produce the utopian transformation of work into an exaltation of play that he had longed for: his theory of ludocapitalism was "the closest I can come to saying why it was I left the business when I did" \autocite[299]{Dibbell2007-dd}.

Dibbell's ambivalence can be summarized as follows: although the discovery and analysis of virtual economies revealed significant and compelling phenomena representative of post-industrial labor, the mere presence of productively-valued output within game-like spaces, or of game-like autonomy within productive spaces, does not guarantee more meaningful work or even freedom from exploitation. On the contrary, the removal of traditional boundaries between work and play, and the liquidation of play money into global ludocapital, have produced not only new forms of production but also new forms of global exploitation. For a more recent example, in \citetitle{Dyer-Witheford09}, \citeauthor{Dyer-Witheford09} document the complex, contradictory intersections of forces of global biopower apparent in the commercial gold-farming operations prevalent in Blizzard's World of Warcraft game:
\blockcquote[149]{Dyer-Witheford09}{
  Gold-farming operations,…have their own deeply exploitative work disciplines: behind the hunter or rogue looting gold in Azeroth, there is a player who, while he or she reappropriates value from Blizzard, is her- or himself expropriated of that value by cyber-sweatshop operators and RMT |real-money trading| brokers. This workforce,…is recruited from those dispossessed by the primitive accumulation proceeding around Guangzhou, Shanghai, and Beijing---a primitive accumulation that is itself, in a bizarre circularity, partly driven by China's new position as the global center of computer production and commercial Internet activity, including MMO |massively multiplayer online| play.
}
Unquestionably opposed to any uncritical exaltation of play, \citeauthor{Dyer-Witheford09} argue that "the controversy over gold farming displays the dystopian realities of social existence so saturated by commodification that it is impossible to escape even in play" \autocite[150]{Dyer-Witheford09}, a demonstration of "how powerfully games have been subsumed by capital" \autocite[151]{Dyer-Witheford09}. With appropriately fungible secondary markets in place, mining gold in a virtual world is no more sacred, inherently meaningful, or resistant to global capitalist exploitation than any other activity of human labor.

\section{Bitcoin}
The economies of digital labor produced in Everquest, Ultima Online and World of Warcraft discussed above illustrate the contemporary phenomenon of real-world value spontaneously generated from within the rule-based constraints of digital game environments. These examples all lend support to the etherealization thesis discussed in the first section, since the virtual gold within these game-worlds is no longer physical material but merely digital quantities of transferable power within the online game environment. Considering these virtual worlds are designed, authored and owned by digital game companies and all of their virtual properties technologically monitored and enforced by centralized servers, they are corporate equivalents of money issued and controlled by the state, representing a ludocapitalist equivalent of the cartalist ideal of money.

In this section, I consider the case of Bitcoin\footnote{
  A note on capitalization: I have chosen to follow the convention of distinguishing between "Bitcoin" when referring to the decentralized network or project as a specific entity, and "bitcoin" when referring to a general unit or quantity of virtual currency. This convention is itself contested, and my choice reflects claims I make in this chapter. A writer for the Wall Street Journal rationalized his editors' decision to adopt a lowercase convention by arguing that "Bitcoin is not a single, specific Thing.…|It| is a multi-faceted, dispersed, decentralized thing. It is everywhere, under no single entity's control, like computers and cars and books" \autocite{Vigna14}. My position in this chapter is that despite the decentralized diffusion of bitcoins, Bitcoin as a software project and network of payments is nonetheless still a concrete Thing that is subject to identifiable forms of protocol-oriented control, influence and critical analysis.
}
and its diaspora of alternative "crypto-currencies" as a more recent class of decentralized virtual currency systems that represents a ludocapitalist paradigm of money that relates to, yet transcends, the endogenous economies of virtual worlds. I develop a reading of Bitcoin as a paradigmatic expression of the contemporary "spirit" of ludocapitalism, which involves a transcendent faith in pure, global computation subverting a transcendent faith in a market conventionally conceived as a political-economic network of global financial institutions.

In many ways, the emerging landscape of Bitcoin and associated crypto-currencies is comparable to the capital flows of the virtual economies that preceded its creation. First, both economies depend on a system that allows for the production of objects embodying relatively durable forms of "congealed labor time," to use Marx's term, and a mechanism for secure exchange across the Internet. In comparison to an economy such as Ultima Online that revolves around the exchange of in-game fictional "gold" and other virtual property that demands estimable quantities of in-game human labor time to accumulate, Bitcoin's economy revolves around the production and exchange of "proof-of-work" tokens representing verifiable amounts of computational effort and expense.

I find the best way to illustrate the distinction between Bitcoin and virtual currency is by way of a comment about Bitcoin by the original authority on virtual economies. In 2011, Castronova published a blog post on why he is "skeptical about Bitcoin" in contrast to his sustained enthusiasm for virtual world economies: his main concern centers on the proof-of-work creation algorithm at the heart of Bitcoin's economy:
\blockcquote{CastronovaBitcoin}{
  Whatever the actual process, |proof-of-work| is not >meaningful work< in the sense of the real world or video games. Technically, yes: It is an operation that consumes resources and results in a proof of work. But it doesn't contribute anything in a human sense to the universe.…Meaningful work is work that an ordinary human being could view as part of some quest or achievement or contribution.
}
Rather than explaining why Bitcoin's economy values less "meaningful work" than the quest or achievement-oriented forms of play-labor taking place within online game economies, however, I think that Castronova's comment instead reveals a deeper, more substantial ludocapitalist spirit at the heart of the Bitcoin economy: a collective belief, often times approximating spiritual faith, in cryptographic computing power as the most meaningful work possible in contemporary technoculture. In this sense, an interpretation of Bitcoin as play money can be viewed as an antithesis or critique of fantasy game economies in which production involves laborious tasks linked to mundane, repetitive human interactions cloaked in fantastic metaphors of combat, exploration, natural resource mining or material craftsmanship (its "grind" in Dibbell's parlance). As such a critique, Bitcoin runs parallel to Progress Quest, a satire of EverQuest's fantasy roleplaying simulation of wealth accumulation that replaces the grind of human interaction with a progress bar \autocite{ProgressQuest}. Instead of human button-clicking comprising the bulk of labor-creating wealth producing resources, inventory, and experience points, Progress Quest avatars continuously amass points, wealth and power without any human interaction at all, as the autonomous, timed advancement of a fluctuating progress bar results in one's avatar killing monsters, completing quests, and accumulating wealth and power in the game world.

I find the global adoption of decentralized currency Bitcoin to be a critical juncture in the history of money not because it fits easily within a progression towards an ideal nexus of pure information-commodity exchange, but because it establishes a new discourse of wealth built around a transcendent spirit of computation as a new organizing force in economic governance. In this section, I will interrogate the underlying materiality of this claim, one which is largely founded on a strong metallist analogy to gold yet in other ways is also unique to ludocapitalism, through an exploration of the various perspectives through which the Bitcoin economy is understood and valued by its participants. As neither a precious metal extracted from the earth nor a symbolic token of credit backed by the state, the various elements that contribute to Bitcoin's transubstantiation from play money into legitimate, valuable currency is a reflection of the contemporary spirit of ludocapitalism upheld in the collective belief of its faithful followers.

The remainder of my discussion of Bitcoin will proceed as follows:

First, I will discuss the mythical, pseudonymous foundation of Bitcoin, arguing that its technocultural origins in libertarian and cypherpunk ideology are not merely coincidental but comprise an essential meta-narrative grounding the currency's spiritual legitimacy.

Second, I will examine the "digital metallism" grounding the protocol outlined in the original Bitcoin white paper within a materialism strikingly analogous to Marx's analysis of gold in his labor theory of value. Following this, I explore the multivalent identity of Bitcoin along various hermeneutic perspectives: as a technical software project, political ideology, speculative fiction, authored text, financial asset and investment vehicle. Bitcoin cuts across all of these identifications and disciplinary boundaries, finding a material basis in what Galloway describes as a "physical logic" of protocol.

Finally, looking beyond Bitcoin economy itself to its initiation of a Foucauldian discursive practice, I read its diaspora of alternative crypto-currencies as a new, sustained method of experimental economic discourse. Although the vast number of "altcoins" largely evoke similar protocological materialities as Bitcoin, the sustained existence of such alternatives and their communities demonstrate, contra Galloway, that the material force of protocol can be critiqued and transformed through discursive means other than hypertrophic exploits of the protocol itself, encouraging us to read political and social aspects of the assumed materiality within protocol designs.

\subsection*{Satoshi's Performance of Identity}
The Bitcoin project made its public debut on October 31, 2008, when a message was posted to The Cryptography and Cryptography Policy Mailing List from "Satoshi Nakamoto (satoshi@vistomail.com)," titled "Bitcoin P2P e-cash paper." The message begins, "I've been working on a new electronic cash system that's fully peer-to-peer, with no trusted third party," followed by an abstract of a technical white paper on the Bitcoin system architecture, and a URL link to the full paper hosted at bitcoin.org \autocite{NakamotoEmail}.

Like many other Halloween costumes on parade that evening, Satoshi Nakamoto was a pseudonym, specifically crafted to present the Bitcoin system to the public behind a digital veil of secrecy. Satoshi's vistomail.com e-mail address and bitcoin.org domain registration were both established through AnonymousSpeech.com, a secure anonymous e-mail and domain hosting company based in Tokyo. With the entirety of Satoshi's communication routed through this anonymity provider, the origin of all communication associated with the identity to date remains a mystery. Despite several public investigative attempts to unmask the real person(s) behind the Satoshi name, only traces of inconclusive, circumstantial evidence have been found, and all potential suspects have publicly denied any connection to the identity.\footnote{
  Satoshi's enduring pseudonymity has attracted considerable commentary and speculation \autocites{Davis11}{Penenberg11}{LikeInAMirror13}{Peterson14}{Goodman14}.
}

However ghost-like the phantom identity of Satoshi appeared to those seeking to find the person or people behind the digital mask, the author nonetheless maintained a commanding, unifying virtual presence within the growing Bitcoin community for over two years. During that time, Satoshi closely managed Bitcoin's development, responding to mailing list and forum discussions, published periodic updates and bug fixes to the Bitcoin code repository, and collaborated with other early contributors through IRC and email. Satoshi's public communication abruptly halted weeks after WikiLeaks' momentous publication of the United States diplomatic cables in December 2010. After a range of payment processors including Visa, MasterCard, PayPal, Bank of America and Western Union all blocked WikiLeaks from receiving donations through their payment systems due to political pressure, some Bitcoin forum members started clamoring for WikiLeaks to accept Bitcoin. Satoshi disagreed, arguing that the Bitcoin network was still too young to handle such a public political stage. After an article in PC World profiling Bitcoin in the context of the WikiLeaks scandal, Satoshi wrote ominously, "WikiLeaks has kicked the hornet's nest, and the swarm is headed towards us," making his last public communication less than a day later \autocites{Nakamoto10}[see also][]{Wallace11}.

In light of this founding narrative, I view Bitcoin as a particularly self-referential performance of pseudonymous identity, one that is essential to the network's spiritual legitimacy. Insofar as the preservation of liberty through technological pseudonymity is one of the project's core ideological values, I imagine Satoshi as the network's image of authenticity, its archetypal subject or Weberian "charismatic authority." Such a subject may be less recognizable as an author accorded to traditional forms of media, but more endemic to the kind of selectively-public, digitally-mediated social relations the Bitcoin project imagines. The lasting reverberations of Satoshi's presence extend far beyond the authorship and original stewardship of the Bitcoin project, as his mysterious, untraceable identity has become an enduring foundation myth that has helped fuel further popular interest in the network. As a public performance of digital anonymity, the legend of Satoshi benevolently establishing the Bitcoin network and then vanishing without a trace has only added to the hacker-mystique narrative that sparked public media interest and contributed to its broad adoption.

Related to this identity performance, I understand Bitcoin as a political movement and ideological statement. The motivating premise behind Bitcoin's decentralized, peer-to-peer architecture is that a global, networked computing infrastructure makes for a more reliable, efficient and trustworthy monetary system than the existing network of international financial institutions. This ideal of money autonomously administered through an impartial Internet protocol meshes well with the libertarian ideal of disestablishing existing financial institutions that subordinate the abstract freedom of market exchange to specific political, national, or self-serving interests. In this vein, Bitcoin has been ideologically aligned with libertarians, crypto-anarchists and other similar groups whose politics advocate shifts in power away from existing nation-state institutional regimes toward distributed, computer-mediated systems of authority and exchange.

\subsection*{Digital Metallism}
I read the economic structure of Bitcoin's protocol outlined within Satoshi's original white paper as strongly advocating a form of what Maurer et al. have termed "digital metallism," grounded in a practical materialism running parallel to Marx's analysis of gold in his labor theory of value and commodity theory of money:
\blockcquote[13]{Maurer2013}{
  The digital metallism of Bitcoin echoes the materialism of commodity theories of money, such as those championed by Locke, the bullionists of the nineteenth-century, and the gold-standard supporters of today. And, as with Locke, this metallism is also part of a broader materialism linked to an ideology that emphasizes individual liberty and sees >sound< money as a key component of that liberty, as well as a key site for potential government intrusion.
}
The digital maxim at the material center of the Bitcoin protocol is the cryptographic concept of "proof of work," which Satoshi describes in his white paper with a political inflection as "one-CPU-one-vote" \autocite*[3]{Nakamoto2008}. Such a rule makes computational might the only form of power given voice within the economic community, granting physical computer hardware a privileged status in the emerging discourse. Cryptographic proof of work thus imagines the number-crunching machine as the ultimate equalizer of digital society, efficiently flattening an unprecedented deluge of voices into quantitative economic values according to a simple collective game of computational strength. In this way, Bitcoin replaces (but preserves through metaphor) Marx's concept of human labor in the form of mining physical gold as the universal equivalent of value with computational labor in the form of "mining" digital blocks of bitcoins.

By specifying a controlled competition for newly minted coins within the protocol, Satoshi's Bitcoin design also added a crucial psychological element to the recipe for money that previous crypto-currencies had not yet perfected: market competition, or greed. Best represented by premodern alchemy in the form of the philosophers' stone, it is the element of greed, the human desire for accumulating objectified, external power over nature and social relations, that explains money's supernatural power over the individual, as \citeauthor{MarxGrundrisse} demonstrates through an alchemical analogy in the \citetitle{MarxGrundrisse}:

\blockcquote[221--2]{MarxGrundrisse}{
  From its servile role, in which |money| appears as mere medium of circulation it suddenly changes into the lord and god of the world of commodities.…It is exactly as if, for example, the chance discovery of a stone gave me mastery over all the sciences, regardless of my individuality. The possession of money places me in exactly the same relationship towards wealth (social) as the philosophers' stone would towards the sciences.

  Money is therefore not only \emph{an} object, but is \emph{the} object of greed. It is essentially \emph{auri sacra fames} |accursed hunger for gold|.
}
Marx concludes that "Monetary greed, or mania for wealth, necessarily brings with it the decline and fall of the ancient communities" \autocite[223]{MarxGrundrisse}, an analysis precipitating his prediction of the inevitable decline and fall of the modern capitalist system. Pursued further, however, the alchemical analogy also reveals the limits of such essential claims. Although Marx states that "Money as individualized exchange value and hence as wealth incarnate was what the alchemists sought" \autocite[225]{MarxGrundrisse}, Karen \citeauthor{PinkusAlchemical} observes in her literary history of alchemy that the alchemical tradition was in fact more ambivalent: "Greed is precisely what is disavowed by those more >spiritual< or philosophical forms of alchemy, and the typical early modern alchemical treatise includes disclaimers against the use of precious metals on the market. Even if the alchemist tried to exchange his product for commodities, he would not succeed. So greed must be considered crucial to alchemy, even when---or especially when---it is denied" \autocite[10]{PinkusAlchemical}.

Similarly, the pursuit of wealth within capitalism is more ambivalent, as \citeauthor{WeberPE}'s analysis of \emph{auri sacra fames} in \citetitle{WeberPE} attests: Although "the auri sacra fames is as old as the history of man,…those who submitted to it without reserve as an uncontrolled impulse were by no means the representatives of that attitude of mind from which the specifically modern capitalistic spirit as a mass phenomenon is derived, and that is what matters" \autocite[ch.~2]{WeberPE}. His argument is that greed alone is neither a unique cause nor a sufficient explanation of modern capitalism, but rather that "the rational utilization of capital in a permanent enterprise and the rational capitalistic organization of labor" had "become dominant forces in the determination of economic activity." According to Weber, the spirit of capitalism does not necessarily encourage greed to run rampant and disintegrate, but rationalizes the individualist desire for the accumulation of wealth, positing it as an ethical, economic ideal.

Within Bitcoin's protocol, I understand the design of greed in a similar fashion to Weber's analysis: Bitcoin does not merely induce mania for wealth for the sole purpose of exploitation as in so many Ponzi schemes, but it instead rationalizes the greedy self-interest of its participants through a controlled incentive structure forming the material basis of its protocol. Bitcoin establishes this rationalization of greed through explicit analogies to gold mining grounding its digital metallism in metaphor, as explained in Satoshi's white paper describing the Bitcoin protocol:

\blockquote{
  To compensate for increasing hardware speed and varying interest in running nodes over time, the proof-of-work difficulty is determined by a moving average targeting an average number of blocks per hour. If they're generated too fast, the difficulty increases.…

  By convention, the first transaction in a block is a special transaction that starts a new coin owned by the creator of the block. This adds an incentive for nodes to support the network, and provides a way to initially distribute coins into circulation, since there is no central authority to issue them. The steady addition of a constant,…amount of new coins is analogous to gold miners expending resources to add gold to circulation. In our case, it is CPU time and electricity that is expended. \autocite*[3--4]{Nakamoto2008}
}
In a discussion following the initial public release of the Bitcoin client, Satoshi further elaborates on this analogy to gold, expressing the hope that the currency's deflationary design would, through a rising value and the element of greed, promote further adoption of the system in a "positive feedback loop":
\blockquote{|I|ndeed there is nobody to act as central bank or federal reserve to adjust the money supply as the population of users grows.…|I|n this sense, it's more typical of a precious metal. Instead of the supply changing to keep the value the same, the supply is predetermined and the value changes. As the number of users grows, the value per coin increases. It has the potential for a positive feedback loop; as users increase, the value goes up, which could attract more users to take advantage of the increasing value. \autocite*{NakamotoP2P}
}

Grounded by the cryptographic proof-of-work algorithm linking computational power to a tangible, material object combined with a carefully-designed competition for scarce, limited resources, the network produces an "alchemical recipe" for money through two key mechanisms both directly tied to individual self-interest. First, the variable mining difficulty induces a computational arms race among self-interested miners to claim the few coins trickling out of the system over time, which allows the amount of total coins in the system to be indirectly guided by the protocol's initial settings. Second, the fixed upper limit on the total bitcoin supply, set by Satoshi to 21 million coins, guarantees their scarcity, inducing a "positive feedback loop" as a second sort of controlled mania.

Since the network's launch, this positive feedback loop has produced a computational arms race that has grown to planetary proportions. In its early years, Bitcoin mining appealed to early-adopter, network-connected computer users who had machines with extra CPU cycles to spare. When the original client software was modified to take advantage of the fast floating-point processors on consumer graphics processing units, the demographics of mining expanded to include gamers and hobbyists who owned or invested in high-end aftermarket video cards that could more quickly generate the cryptographic hashes\footnote{
  A \emph{hash} is the quantity of verifiable work performed by a machine in the Bitcoin network, and \emph{gigahash/sec} is a measurement of the relative strength of an individual node in the network or the network in total. The hashing function used in Bitcoin is derived from the \citetitle{Hashcash} mechanism \autocite{Hashcash}. As the mining difficulty (the probability that any single hash calculation will yield a valid bitcoin block) is adjusted based on the total strength of the Bitcoin network, the fixed rate of bitcoins produced the network is kept relatively constant independent of its total power.
} awarded coins. More recently, as the Bitcoin network has grown into a billion-dollar economy, the hashing power arms race has shifted towards enterprising companies fabricating customized application-specific integrated circuit (ASIC) hardware, either running private Bitcoin server farms or selling/leasing the hardware to eager Bitcoin investors \autocite{Hill14}. Swedish-based KnC miner, for example, reportedly sold out \$25 million worth of dedicated Bitcoin-mining ASIC machines within two weeks in December 2013, as the global hash rate of the Bitcoin network increased from around 20 gigahash/sec in January 2013 to over 12 million gigahash/sec in January 2014.

\subsection*{Bitcoin as Speculative Investment}
Encouraged by the digital metallism embodied in its design and encouraged by its author, comparisons to gold have become commonplace among speculators and pundits commenting on the Bitcoin network, making it an attractive investment option among technologically-inclined libertarians and goldbugs. As one prominent example of this trend, in September 2013 the Winklevoss twins of early Facebook fame began shilling their Bitcoin-derived Exchange Traded Fund by hawking Bitcoin as "Gold 2.0," using the common software-versioning trope of a "2.0" release number to indicate a technologically-advanced successor of an existing system, offering tulip-manic predictions of the exchange value of a bitcoin rising over 100 times its present price \autocite{Farrell13}.

As a speculative investment, Bitcoin's financial fiction can be summarized as the counterfactual proposition that bitcoins contain value. If bitcoins are valuable, then they can be effectively exchanged for goods or services. As opposed to physical goods, or even government-issued currencies backed by a guarantee to accept national currency as payment for public taxes or debt, bitcoins have no intrinsic use value or institutional support that would provide a stable support for a corresponding exchange value. Instead, the value of a bitcoin exchanged at any moment is much more speculative, grounded only in the quasi-religious belief that another economic subject (a "greater fool") will accept the currency as valuable in the future. In this case, lofty Utopian narratives of a future in which Bitcoin becomes the primary world currency for a global digital economy provide the currency with a self-justifying narrative of enormous potential value. As one example, an early Danish Bitcoin exchanger, Lars Holdgaard, created a website to "Calculate the future value of Bitcoins." Taking into account the fixed supply of 21 million Bitcoins, if the network grows to 0.25\% of the global economy with a quarter of the coins used for transactions, then the price of a single Bitcoin would increase to \$37368. "If you believe in Bitcoin, and have a believe [sic] it can be used in 0,05-1\% of all transactions in the world, it will have a HUGE value. As an investment alone, this is a smart choice" \autocite{Holdgaard13}. Similarly, Mike Caldwell, an entrepreneur peddling "Casascius Coins," physical coins embedded with private keys linked to fixed denominations of Bitcoins, argues that the "fundamental value of a Bitcoin" is a speculative proposition: "To me, it is either worth zero, or it is worth a lot. It is either the financial revolution of the 2010's, or it isn't. If a Bitcoin is not worth a lot of money, it is worth zero. There's no middle ground" \autocite{Caldwell13}.\footnote{
  Caldwell's operation was suspended in November 2013 when the Financial Crimes Enforcement Network sent him a letter claiming that minting physical Bitcoins amounted to operating an unlicensed money transmitter business.
}

In this view, the simple value proposition in the Bitcoin network is akin to a digital version of Pascal's famous "wager," the theological argument for a rational belief in God based on the non-zero possibility of infinite reward should God indeed exist. The very likely possibility of an investment in Bitcoin ultimately becoming worthless is offset by the very unlikely but non-zero possibility of achieving unprecedented gains should the Bitcoin network manage to succeed in ultimately replacing the existing global financial institutions. In this way, Bitcoin as a speculative investment is sustained by a collective belief in the inevitability of its greater future value. The religious, even eschatological, overtones in this counterfactual proposition of ultimate value are clear.

\subsection*{Bitcoin as Enterprise}
Somewhat related to the view of Bitcoin as a speculative investment is the view of the Bitcoin as an opportunity for entrepreneurial activity or capital investment. Considering the core of Bitcoin's algorithmic ideology is a political-economic axiom equating computing power with political-economic voice, it's perhaps not too surprising that Silicon Valley has quickly become a breeding ground of organized technical development, financial investment, and general entrepreneurial enthusiasm for Bitcoin-related products and services. For example, in an essay published in the New York Times, Marc Andreessen, the co-founder of Netscape and a large venture-capital firm, equates Bitcoin to the personal computer in 1975 and the Internet in 1993 as the next big transformative technology, having already invested just under \$50 million in Bitcoin-related companies \autocite{Andreessen14}. One way to characterize the symbolic struggle over Bitcoin is as a divide between techno-capitalist faith in the liberatory potential of technological protocol for entrepreneurial "creative destruction" and an entrenched oligopoly of neoliberal financial actors aligned with government actors backing the existing global financial system. Venture capital-financed Bitcoin entrepreneurs, viewing themselves as the spiritual successors to the libertarian heritage of early commercial Internet company success stories such as Netscape and PayPal, see an opportunity to shape Bitcoin into a global currency system that would more efficiently replace monetary regimes dominated by national currencies and precious metals, particularly in light of the loss of popular faith in existing financial systems following the global financial crisis. In this long-term strategic view Bitcoin is still in its infancy, and the hyper-competitive stage of institutional growth is still so young that new entrants still stand a chance to strike it rich by staking early claims.

In his essay, Andreessen concludes, "Bitcoin offers a sweeping vista of opportunity to reimagine how the financial system can and should work in the Internet era, and a catalyst to reshape that system in ways that are more powerful for individuals and businesses alike." Although this reimagining is still in its infancy, there's good reason to remain skeptical of such sweeping, transformative visions. The entrepreneurial vision Andreessen presents for Bitcoin as a government-free, libertarian world payment system of the future masks the enormous capture of democratic voice that such a system would entail if successfully realized at such a grand scale. From a political perspective, a government of "one-CPU-one-vote" would become a pure plutocracy. By linking money's universal ability to acquire property to the universality of autonomous, ungovernable agency of the libertarian Internet imaginary, Bitcoin ideologically fuses together two facets of the transcendental techno-bourgeois subject in a translation of computational power into universal wealth. It replaces the form of money linked to the state's accumulation of administrative and political power with another form linked to the accumulation of computational and protocological power, allowing a decentralized control increasingly independent of the traditional mechanisms of national democracy, and thus more vulnerable to unchecked authoritarian influence from autonomous, powerful actors and corporations capable of making large investments in the technology sector.

\subsection*{Bitcoin as Media Story}
One of \citeauthor{McLuhan64}'s boldly prophetic pronouncements about money is that our contemporary dynamics of mass media will be reflected in new money forms: "Today, electric technology puts the very concept of money in jeopardy, as the new dynamics of human interdependence shift from fragmenting media such as printing to inclusive or mass media like the telegraph" \autocite[139]{McLuhan64}. From this perspective, I find one of the most salient components of Bitcoin's patterns of adoption within the expanding circulation of news about Bitcoin. One might even say that Bitcoin exists as a material reality of the mass media above all else.

In the United States, Bitcoin has been all over the news. According to a national poll, by December 2013 over 42\% of Americans correctly answered that Bitcoin was a virtual currency \autocite{Bloomberg13}.
Like the global spread of news of the discovery of gold in northern California in 1848, news of the rise of Bitcoin has itself been largely responsible for the succeeding cycles of wealth-grabbing euphoria, fueled by individualistic opportunism and the speculative possibility of accumulating great amounts of private wealth within a brief window of opportunity. However, vast temporal differences separate the dynamics of Bitcoin's adoption from the global onrush of prospectors toward reports of new sources of gold. Despite their material macroeconomic similarities and similar ideologies of economic autonomy from nation-state control, gold has a material history as a valuable commodity spanning thousands of years, and a stable network of global trade throughout the world. Bitcoin, by comparison, is a very new digital fiction, built around an enthusiastic but still comparatively tiny network of believers that could be extinguished as easily and quickly as it was created. One illustrative contrast is the global rush of prospectors to Northern California in the years following the discovery of gold in the region compared to the tepid response to the Bitcoin network's initial launch in January 2009 which, without any established notoriety, started out as play money with no value whatsoever. As Marx's primal form of universal money within capitalism, gold is \emph{a priori} valuable; in order for anyone to attribute any value at all to bitcoins, they must first hear and learn about it through the mass media. In this way, Bitcoin is a mass media story before it is anything else.

As a mass media story, Bitcoin expanded in cycles or stages, as the fragile network was tested by increasingly broad and diffuse layers of public support. The extreme fragility of Bitcoin's network effects and its dependency upon media exposure for its growth is most apparent in the enormous degree of influence that public communication channels and moments of broadened media exposure had over the contours of Bitcoin's early growth. Strong correlations have been found, for example, between Google and Wikipedia search volumes for Bitcoin and exchange rates for the currency.\footnote{
  For an recent quantitative analysis of these lines, see \autocite{Kristoufek13}.
}

These correlations are already evident in the first and arguably most important digital media intervention by the early Bitcoin community in July 2010, following the release of version 0.3 of the Bitcoin client. After a week of discussion in a Bitcoin Forums thread debating the appropriate wording of a concise, general-audience introduction to the Bitcoin project \autocite{Slashdot1.0}, a Bitcoin community member submitted the news contribution to the tech-news web portal Slashdot on July 11, 2010, and was soon featured on its front page \autocite{Slashdot2010}. The resulting "Slashdot effect" overloaded Bitcoin Market, the first and only automated Bitcoin exchange available at the time, bringing the trading server down for a full day \autocite{BitcoinMarket}. The next day, the Bitcoin-US Dollar exchange rate jumped tenfold (from one-half cent to five cents per Bitcoin), with over 400 new Bitcoin clients connecting to the network, producing a similar tenfold increase in the network's total computational power \autocite{BitcoinGrowingUp}.

This early media flurry was only the first bubble-like expansion of Bitcoin into a broader public consciousness. The next time Bitcoin pushed the boundaries of acceptance, its own rise in popularity became the news story propelling it along \autocite[e.g.,][]{Slashdot1Dollar}. These headlines fueled steadily increasing enthusiasm and exposure, leading to the currency's next big bubble around July 2011. This time, the correction following the bubble was much sharper and prolonged, triggered by a massive security breach at Mt. Gox, the Bitcoin exchange handling the vast majority of trading activity at the time.

The next bubble around April 2013 triggered a new news cycle that expanded the audience even further, so that a new wave of people around the world heard news of the rise of Bitcoin for the first time. This bubble was typically rationalized by news of a banking crisis in Cyprus which threatened to seize a portion of national bank deposits, where the strong backlash against the government's economic decision that would directly affecting the holdings of individual citizens presumably many look to accessible solutions for shuttling their private wealth outside the influence of their government \autocite{Bustillos13}. Around the same time, the ongoing devaluation of the Argentinan peso along with capital controls on foreign currency and precious metal exchange fueled interest in the virtual currency as a hedge against further national inflation \autocite{BitcoinsInArgentina}. With the subsequent influx of new buyers and corresponding increase in exchange value, the total market captured by Bitcoin currency exceeded a billion US dollars, making Bitcoin's popularity again its own newsworthy event. Talking heads on mainstream financial television channels and radio news discussing Bitcoin became commonplace, and some even reported a ticker displaying the Bitcoin exchange rate on MSNBC for a brief period.

A further recent wave of expansion began with a favorable in-depth documentary on Bitcoin that aired on China's state television broadcast network CCTV in May 2013 \autocite{Stacke13}. As news about Bitcoin spread around the country more news stories about Bitcoin's adoption and investment energy fed an increasing cycle, leading to search engine Baidu's decision to accept Bitcoin payments in October. By November 2013, Bitcoin exchange BTC China had quickly risen to become the largest bitcoin currency exchange by volume.

The perspective of Bitcoin as mass media story therefore provides a key component of the narrative underlying its cycles of growth and adoption that are fundamentally different from the laws governing the prices of existing globally-distributed commodities or state-governed currencies. The "viral" growth and adoption of Bitcoin has fundamentally depended upon the Internet for the global acceleration of its mass media story through several cycles, and the latter is largely responsible for the incredible dynamism of the currency's global adoption. However, as McLuhan warned, such a vital dependence upon the mass media also throws the very concept of money in jeopardy: given Bitcoin's ongoing dependency upon the novelty of its mass media story for its early stages of growth and adoption, it is therefore an open question whether Bitcoin can continue to produce itself as a form of money if or when its story has been told everywhere, and it no longer circulates as news.

\subsection*{Bitcoin as Money?}
The proposition that "Bitcoin is money" is often contested through reference to the four traditional functions of money. Introductory macroeconomic textbooks often begin a high-level discussion of these modern functions of money by citing the traditional couplet, "Money is a matter of functions four, a medium, a measure, a standard, a store."\footnote{
  See \autocite[158]{Dwivedi10} for an example. The couplet is typically unattributed; the earliest known reference I have found is \autocite[55]{Milnes1919}.
}
These four functions, which derive from neo-classical economic theory, separate money into a medium of exchange; a measure or "unit of account" of price; a standard of deferred payment (e.g., for settling debts); and a store of (long-term) value.

This separation of money into four functions comes from Stanley \citeauthor{Jevons1875}'s \citedate{Jevons1875} treatise, \citetitle{Jevons1875}. Jevons separated money into its functions as "a medium of exchange," "a common measure of value," "a standard of value," and "a store of value" \autocite[13--5]{Jevons1875}, adding: "It is in the highest degree important that the reader should discriminate carefully and constantly between the four functions which money fulfils, at least in modern societies" \autocite[16]{Jevons1875}. Although different times and places in history have delegated some of these functions to different physical media, adherents of this unified theory of money suggest that money is at its most ideal (that is, it behaves most like "money") when a single substance supports all four of these functions. In the case of Bitcoin, its function as a secure, distributed medium of exchange is its strongest virtue. However, the possibility of Bitcoin successfully fulfilling the other three functions of money have been disputed due to its volatility, as an economic research report by David \citeauthor{Yermack13} attests: "bitcoin does not behave much like a currency according to the criteria widely used by economists. Instead bitcoin resembles a speculative investment similar to the Internet stocks of the late 1990s" \autocite[2]{Yermack13}.

However, rather than evaluate according to this neoclassical paradigm, I find that identifying money strictly in terms of the unification of these four functions becomes less useful within ludocapitalism, where the emergence and cultural acceptance of various manifestations of play money within our financial institutions encourage these classical functions of money to diverge into distinct forms. Even Jevons emphasized the limitations of assuming a unified understanding of money as the basis of his four-functions separation: "We come to regard as almost necessary that union of functions which is, at the most, a matter of convenience, and may not always be desirable. We might certainly employ one substance as a medium of exchange, a second as a measure of value, a third as a standard of value, and a fourth as a store of value" \autocite*[16]{Jevons1875}. Instead, I argue that Bitcoin functions as money in relation to each of the various perspectives I have used to interpret the project's social significance to date. In this way, my understanding of Bitcoin as money is comprised of all of the multifaceted social currents of "spirit" that, as Simmel's philosophy of money argued with analogy to art, are all collectively reflected in its final economic valuation, none of which taken on its own would be necessary or sufficient.

\subsection*{Bitcoin as Discursive Practice}
My final interpretation of Bitcoin's significance leads me beyond its immediate economic valuation as money, beyond even the spirit of the particular monetary regime it establishes: I interpret Bitcoin more generally in terms of Foucault's "transdiscursive" initiation of a discursive practice. For Foucault, the initiator of a discursive practice is a subject who "produced not only their own work, but the possibility and rules of formation of other texts" \autocite*[131]{FoucaultAuthor}. The vast fields of discourse established by the works of Marx and Freud, Foucault offers for example, are much more significant than that of the work of a popular novelist. While the latter might produce a popular genre involving certain narrative themes, literary techniques or the like, what distinguishes the former is its ability to support not merely adaptations or analogies, but that it constructs a discursive stage for the introduction of concepts entirely different from its own but nonetheless within the field it initiated. Likewise, the initiation of a discursive practice is different from the founding of a new science in that future developments of a scientific practice can reconstruct its theoretical foundations according to future empirical evidence, whereas a discursive practice is "heterogeneous to its ulterior transformations" \autocite[133]{FoucaultAuthor}. In a discursive practice a "return" to the original initiation has the potential to transform our ongoing understanding of the field, as such a return can "reinforce the enigmatic link between an author and his works" \autocite[136]{FoucaultAuthor} that is not possible in scientific practice.

As a technical practice, the Bitcoin distributed network itself is incapable of such a "return" to the original work, as the public decentralized ledger makes all transactions irreversible. However, it can also be said that Bitcoin has founded its own discursive practice, one that split the ongoing discourse on money itself into pre-Bitcoin and post-Bitcoin moments. It is in this sense that, regardless of the Bitcoin network's present utility or longevity or whether the Pascalian wager of the network's early adopters ultimately succeeds or fails, its existence as an inaugurative event is neither a true nor false statement, just as the question of whether Bitcoin is or isn't neoclassical "money" is also not the dispositive question. It is also in this sense that the practice of Bitcoin is worth returning to, as it has captured something about the present technocultural moment that invites scrutiny and reflection.

In order to distinguish the general social significance of Bitcoin's discursive practice from the specific economic significance of its particular monetary network, I consider the latter in terms of Galloway's concept of \emph{protocol} as a strategy of control in decentralized networks. In order to participate in the Bitcoin network, one must implicitly accept the rules of the game embedded in the protocol accepted by all other players. In order to become a bitcoin-trading subject, one must therefore accept the entire history inscribed in the immutable, decentralized ledger of transactions recording the ownership and history of every single bitcoin created to date. In this way, control is maintained within the Bitcoin network through protocol, and the particular rules and parameters comprising Bitcoin initially established by Satoshi are collectively enforced through a consensus of continued adoption, long after the original author's absence.

Galloway's thesis of \citetitle{Galloway2004-ac} is that the decentralization of network architectures does not determine their inherent liberation, but rather engenders new forms and topologies of control and struggle that take place within the protocol itself. However, I believe that the significance of Bitcoin as a discursive practice beyond its particular network protocol demonstrates that the material force of protocol can be resisted through social means beyond than hypertrophic exploits of the protocol itself. An under-recognized facet of Galloway's protocol theory that I wish to highlight is his insistence on protocol's tendency to become reified as something more solid or material than mere social consensus:
\blockquote{
  As one learns more and more about the networks of protocological control, it becomes almost second nature to project protocol into every physical system.…But protocol is more than simply a synonym for >the rules.<…|A| better synonym for protocol might be >the practical,< or even >the sensible.< It is a physical logic that delivers two things in parallel: the solution to a problem, and the background rationale for why that solution has been selected as the best. \autocite*[244--5]{Galloway2004-ac}
}
It is only through this sort of protocological reification, grounded in a digital metallism supported by a growing community of adopters and investors, that an ethereal communications network such as Bitcoin can be viewed as grounded in a "physical logic" providing it a material basis. Galloway's materialist description of protocol here masks the everyday complexities of contemporary information capitalism: protocol adoption often has very little to do with what protocol solution is "the best" in relation to an isolated problem, and very much to do with the protocol's adoption within discursive networks. In this vein, Bitcoin has generated a simulacrum of the dynamism of information capitalism, now surrounded by hundreds of forks of the Bitcoin software project, all running their own fully-functional alternative crypto-currency networks, competing for attention, market share and exchange value within a vibrant "altcoin" diaspora.

\subsection*{Bitcoin's Diaspora}
The spread of Bitcoin among software developers and entrepreneurs, combined with the free, public distribution of its published white paper and open-source software project (without any intellectual property claims of any kind attached), has triggered an unprecedented flood of Bitcoin forks, clones, and similarly-inspired virtual currency projects.\footnote{
  See \autocite{Popper13}; as of September 2014, at least 486 unique crypto-currencies exist on the Internet \autocite{CoinMarketCap}.
}
It is in full view of these alternative currencies that Bitcoin's greatest significance as initiator of a new discursive practice of money can best be appreciated. Beyond the particular valuation of bitcoins within its original network, we can consider Bitcoin as a generalized technique in subjecting economic activity to decentralized protocological control, one whose most significant implications may have yet to be realized through one of its hundreds of experimental evolutions.

This alternate currency diaspora contains many fledgling networks that are slight modifications or even exact source-code replicas of Bitcoin, allowing participants to become early-adopters on new currency networks where everyone's balances are reset to zero. Other alternate currencies attempt to address perceived deficiencies by tweaking technical aspects of Bitcoin's algorithm, or to extend the central concepts of decentralized currency to new applications. Through a brief discussion of two of these projects, Freicoin and Ripple, I will show how Bitcoin has established not just its own decentralized currency network but also an experimental mode of economic and protocological discourse that supports a wide array of alternative deviations from the particular spirit of money Bitcoin itself envisioned.

\subsubsection*{Freicoin}
Freicoin is an alt-currency designed around a combination of Bitcoin's decentralized currency technology with the additional concept of demurrage, as introduced by the German theoretical economist, social activist, and anarchist Silvio Gesell in the early 20th century. The crux of Gesell's critique of the standard government-centralized monetary systems of his time, particularly the practice of pegging the exchange rate of official paper currencies to stores of precious metals such as gold or silver, is that the individual incentive to hoard a fixed supply of money such as gold as a commodity during periods of deflation further reduces the money in circulation, causing a deflationary spiral and general systemic crisis throughout the linked society. Gesell proposed the concept of \emph{freigeld} `free money' as a currency system implementing demurrage, a mechanism causing all issued currency to depreciate in value a small percentage over time to approximate the natural depreciation of consumer goods and encourage spending.

Freicoin is developed as an open-source software fork of Bitcoin's decentralized client, incorporating an automatic fixed-rate demurrage on all currency in the system. In addition, the initial distribution of its 100 million units of currency is also adjusted to offer only a fraction of coins to the pool of machines computing the cryptographic hashes necessary to secure the network, with 80\% of the coins going to the Freicoin Foundation, a non-profit foundation established by the developers with a mission "to promote Freicoin and support a sustainable world" \autocite{FreicoinHow}.

Despite its honorable intentions, it seems that one of Freicoin's greatest barriers to adoption parallels Keynes's critique of Gesell's original freigeld proposal: given the choice of an open market and beginning with a state of very low adoption, why would anyone choose to adopt or hold Freicoin at all, as opposed to other stores of value that would depreciate less over time?\footnote{
  Keynes read Gesell's theories with great interest, and accurately noted that the central problem with any demurrage system is the inevitable problem of substitute currencies: "Thus if currency notes were to be deprived of their liquidity-premium by the stamping system, a long series of substitutes would step into their shoes---bank-money, debts at call, foreign money, jewellery and the precious metals generally, and so forth" \autocite*[bk.~6,~ch.~23,~sec.~6]{Keynes1936}.
}
Gesell's \emph{freigeld} proposal assumed a state-issued monopoly of money, where demurrage money would be standardized as the only form of payment for taxes and other public debts, thus enforcing its broad use. Such a system could counter Keynes's critique through protecting state-issued demurrage money against other fixed assets, for example through sales or property taxes on such stores of value. Without any state power to enforce standardization, however, Friecoin can't harness any self-contained economic cycle that would enable demurrage to actually encourage spending rather than discourage use of the currency altogether in light of substitutes.

Freicoin nonetheless still represents an interesting experiment in a different kind of money system, one linked to a public organization with marked differences from Bitcoin's own techno-libertarian algorithmic ideology. This extension of Bitcoin's technology demonstrates that the concept of decentralized currency can be used to promote alternative protocological visions of the future of money.

\subsubsection*{Ripple}
\citetitle{Ripple} is another alt-currency project that attempts to differentiate itself from Bitcoin by offering a general-purpose distributed debt-accounting service rather than just a fixed currency standard. The organizational differences between Bitcoin and Ripple are most striking: Ripple is centrally organized by a for-profit corporation, Ripple Labs, Inc. (formerly OpenCoin, Inc.), founded by a serial entrepreneur and financially backed by a number of prominent Silicon Valley venture capitalist firms including Google Ventures and Andreessen Horowitz. Although the Ripple network itself is decentralized through a peer-to-peer network architecture, the protocol regulating its currency distribution differs markedly from Bitcoin's proof-of-work game controlling the dispersed production of new coins. Instead, Ripple Labs designed its protocol so that all 100 billion of the Ripple network's "ripple" currency units (XRP) begin in control of Ripple Labs, and are selectively distributed to early adopters and partners, sold to currency exchange systems, and otherwise released into circulation according to its corporate interests \autocite*{RippleDistribution}. Through such a carefully-controlled rollout, Ripple Labs's strategy draws parallels to a more conventional entrepreneurial venture, controlling (and monetizing) the network effects of a distributed system by gradually expanding access to broader populations.

Ripple's innovative currency-based business model thus conceals a subtle contradiction. On the one hand, the Ripple network architecture is still a decentralized network in Bitcoin fashion, where currency distributed throughout the network is secure, transactions are transparent on a public ledger, and none of the economic activity is directly controlled by a single authority. On the other hand, the majority ownership of the primary currency itself serves as a means of centralized, economic control of the network. As opposed to Bitcoin's currency which enters circulation through its protocol-established proof-of-work competition, Ripple's XRP currency (Ripple credits, or >ripples<) enters circulation through an opaque, institution-driven process, similar to a private company's stock options. In addition to a lack of transparency in Ripple Labs's plans for future distribution of the currency, 20 billion XRP, or 20\% of the entire economy, was initially granted directly to the founding developers on undisclosed terms. In this way, ripples are a controlled, fictional commodity like other equity securities, but rather than each currency unit representing a fixed portion of the company's future profit, the network is only indirectly linked to the company through its majority ownership of the currency.

The public vision for Ripple is a dream of advancing the technoliberal ideals of friction-free capitalism through the efficiencies of a global, digital currency. However, the project raises questions about whether such a system can be considered a decentralized currency at all, if its currency stores originate from a single controlling corporate source, with a mandate to maximize the network's value alongside its investors' wealth. Instead, the experiment of Ripple more convincingly demonstrates an innovative method of raising capital for a decentralized corporation through the circulation of virtual currency units to the public.

\subsection*{In Math We Trust}
Ripple Labs CEO Chris Larsen hopes that Ripple will become the next generation of "math-based currencies,…needed as a way to move money frictionlessly," of which Bitcoin is the paradigmatic instance \autocite{RippleInterview}. This term doesn't refer to any specific mathematical technique, formula or field, but is rather a general reference to the "real" materialism conferred by mathematical truth. Larson claims that despite its physical substance gold is also a "math-based currency," because of the durable truth that any substance containing 79 protons still counts as gold, regardless of political opinion. By this logic, in a strangely idealist inversion of metallism, a currency is "mathematical" when its system of trust and control no longer relies upon a distinct state or political authority, but is instead organized based on a decentralized consensus through an enacted protocol.\footnote{ A recent Ripple Labs white paper restates this argument: "The supply of a math-based currency is governed by the laws of mathematics. There is no human intervention beyond the creation of the protocol rules" \autocite*[9]{RipplePrimer}.} Tyler Winklevoss, manager of the Winklevoss Bitcoin Trust, endorses a similar mathematical ideology in an interview publicly announcing his venture's control of over one percent of all Bitcoins: "We have elected to put our money and our faith in a mathematical framework that is free of politics and human error" \autocite{Popper-Winklevoss}.

Such a blind faith in decentralized networks is why Galloway argues that protocol can become "dangerous" and take on "authoritarian undertones" \autocite*[245]{Galloway2004-ac}. The desire for a "math-based currency" to facilitate the global, frictionless economy of information capitalism is the latest iteration in the long Enlightenment history of discursive attempts to reduce the power of human reason to the pure formalism of number. \citeauthor{HorkheimerDoE}'s famous argument against the "Myth of Enlightenment" as ideology of instrumental reason reveals a conflation of the controlling power of prehistorical mythological symbols with the formal mathematical symbols cherished by Enlightenment ideals:
\blockcquote[19--20]{HorkheimerDoE}{
  Enlightenment pushed aside the classical demand to >think thinking,<…Mathematical procedure became a kind of ritual of thought.…The reduction of thought to a mathematical apparatus condemns the world to be its own measure. What appears as the triumph of subjectivity, the subjection of all existing things to logical formalism, is bought with the obedient subordination of reason to what is immediately at hand. To grasp existing things as such, not merely to note their abstract spatial-temporal relationships, by which they can then be seized, but, on the contrary, to think of them as surfaces, as mediated conceptual moments which are only fulfilled by revealing their social, historical, and human meaning---this whole aspiration of knowledge is abandoned. Knowledge does not consist in mere perception, classification, and calculation but precisely in the determining negation of whatever is directly at hand. Instead of such negation, mathematical formalism, whose medium, number, is the most abstract form of the immediate, arrests all thought at mere immediacy.
}
Here, \citeauthor{HorkheimerDoE}'s critique the unflinching faith in pure, mathematical reason found in both proto-fascist positivism and techno-libertarian dogma. Instead of taking things as they seem to be in a formal, mathematical, ahistorical certainty, they urge the thinking subject to relate their object to a "mediated conceptual moment" revealing their "social, historical, and human meaning." Faith in the idea of a math-based currency supposedly devoid of political affiliation is nonetheless constructed through a distinct ideology of social and historical associations, politics, meanings and metaphors together comprising a contemporary spirit of computation of which Bitcoin is the paradigmatic instance.

If it is indeed possible to recapture Bitcoin as a mediated conceptual moment beyond the compulsive drive toward the immediacy of friction-free global capital and the reduction of all value to pure, universal number, I believe it will require translating its discursive practices beyond its unified libertarian ledger of transactions, into other diverse rules of formation integrated with specific human, cultural or other ideological values. In the final section, I will interrogate the question of the unilateral rationality of wealth through Marx's comparison of love against money.

\section{Conclusions}
\subsection*{Love and Money}
In his 1844 manuscript on \citetitle{MarxPowerMoney}, \citeauthor{MarxPowerMoney} offers an early version of his commodity theory of money as the universal equivalent of exchange, emphasizing money's effect of reducing mediated social existence to an objective property relation:
\blockquote{
  Money is the procurer between man's need and the object, between his life and his means of life. But that which mediates my life for me, also mediates the existence of other people for me. For me it is the other person.…That which is for me through the medium of money---that for which I can pay (i.e., which money can buy)---that am I myself, the possessor of the money. The extent of the power of money is the extent of my power.
}
This characterization of money as the extent of man's power is later reflected in McLuhan's own maxim of the medium as the "extension of man." However, here Marx more critically and forcefully presents the alienating impact of this economic mediation in a contradictory light. The objectivity of economic exchange enters into a tension against reciprocal social relations of a more subjective (or intersubjective) humanity, as represented by the exchange of love:
\blockquote{
  Assume man to be man and his relationship to the world to be a human one: then you can exchange love only for love, trust for trust, etc. If you want to enjoy art, you must be an artistically cultivated person; if you want to exercise influence over other people, you must be a person with a stimulating and encouraging effect on other people. Every one of your relations to man and to nature must be a specific expression, corresponding to the object of your will, of your real individual life. If you love without evoking love in return---that is, if your loving as loving does not produce reciprocal love; if through a living expression of yourself as a loving person you do not make yourself a beloved one, then your love is impotent---a misfortune.
}
For Marx, the substitution of money for more reciprocal expressions of intersubjective relations threatens to distort the assumed "human" constitution of the individual subject. The substitution of money, which is one-sided, in exchange for love or any other human relations produces "a misfortune," the opposite of true wealth. Through this passage, Marx presents the view that in contrast to the power of money which mediates the existence of the other, the authenticity or legitimacy of human forms of wealth are rooted in reciprocal social relations, in the mutual recognition of the other. \citeauthor{Simmel04} recapitulates Marx's sentiment in a similar meditation on the exchange of love:
\blockcquote[80]{Simmel04}{
  It is above all the exchange of economic values that involves the notion of sacrifice. When we exchange love for love, we have no other use for its inner energy and, leaving aside any later consequences, we do not sacrifice any good.…But economic exchange,…always signifies the sacrifice of an otherwise useful good, however much eudaemonistic gain is involved.
}
These passages on the essential incompatibility between love and economic exchange present another view of Simmel's spiritualization of money, suggesting that the human needs, norms, desires and passions of an economic community, collective, class, or nation develop in tension with the objective, quantitative numeracy of money. In Bitcoin, the link between love and money is projected into an unwavering faith in the power of computation, where a techno-narcissism obsessed with the awesome power of computing infrastructure pervades its discursive production of value.

However, if we understand money not as a transcendental universal of exchange but as a symbol of wealth embodied in specific media, then there is an ambivalent potential for social meanings of money to be consciously revitalized through the production of new concepts of wealth. In Theology of Money, Philip Goodchild frames this question in critical terms as "the problem of the emancipation of evaluation" \autocite*[258]{Goodchild2009-xg}:
\blockcquote[259]{Goodchild2009-xg}{
  [I]n an age of approaching crisis and the tyranny of debt, little can be done until the spectral power of money is addressed. It is urgent, above all else, that time, attention, and devotion be committed to developing new institutions of credit that make effective evaluations once more possible.
}
The spiritualization of money taking place within contemporary ludocapitalism that I have interrogated in this chapter has not yet explicitly taken the production of new effective evaluations as its primary ethical objective. However, the particular qualities of this technoliberal spirit, as embodied within the Bitcoin software project and forks, bold investments in unprecedented economic experiments, and lofty visions of the institutional future of money diverging from national monetary policy do contain novel forms of collective deliberation and patterns of decentralized economic consensus that, I believe, have the potential to engender a diversity of effective evaluations, some of which may support radically new and humane forms of economic life.

\subsection*{Digital Alchemy}
In conclusion, I look forward to an even further intersection between new innovations digital play money inspired by Bitcoin's example and critical discourses on the production of wealth and effective evaluations of value. In this vein, economic journalist David Boyle offers the metaphor of a "new alchemy" to describe individuals and communities attempting to effect social change through enacting alternate forms of money collectively known as LETS (Local Exchange and Trading Systems). Opposed to the superficial view of alchemy as the opportunistic production of base metals into gold, Boyle's alchemy seeks to produce a new economic system capable of validating alternate visions of wealth. Systems he observes include local currencies such as the Time Dollar, which enables participants to exchange hours of informal volunteer or community service work as an idealistic instantiation of labor theories of value built around local community engagement. \citeauthor{Boyle99} summarizes the set of questions some of these new alchemists have posed through the creation of alternate currencies:

\blockcquote[200]{Boyle99}{
  How can society afford the enormous costs of looking after growing numbers of old people---especially when government budgets are being cut?
  
	How can communities defend their local economies, when local earnings are siphoned out of the area by big business or distant utilities?
  
	How can we create a more diverse and sustainable economy locally---and reduce the need for goods to be transported at heavy environmental cost?
  
	How can we create a reliable measure of value so that our local products and earnings stay valuable during inflation or worldwide currency instability?
  
	How can we rebuild communities, friendships and a sense of family so that people look after each other?
}
The Silicon Valley dream of disintermediation, reducing the dependence on financial middlemen in the name of an abstract, mathematical efficiency of a frictionless economy transcending political democracy, is not the only scenario of progress that Bitcoin makes possible. Among the hundreds of alternative digital currencies that have formed in the Bitcoin diaspora, many of them are producing truly alternate, experimental visions of political-economic reality. Insofar as each of these experiments is also envisioning the production of new forms of wealth, I see the political goals of the fringes of the Bitcoin diaspora such as Freicoin and Boyle's New Alchemy to become increasingly aligned in the future, producing what might be called a "digital alchemy" of technological-economic experimentation.

Let us continue to produce playful reconfigurations of money, in the sense of fictional constructions of wealth similar to those invented currencies at the heart of virtual game economies, particularly formations that counter a universal, global commodification of value in recognition of alternate, fanciful constructions of wealth. Play money in this sense has the potential to produce and support new, experimental forms of life under post-industrial capitalism, and a more direct engagement along these lines between the Bitcoin diaspora and that of LETS currencies would be welcome.

From a more general perspective, the Bitcoin system comprised of contrived competition among cryptographic computations, fixed allocation of scarce resources driving speculative investments and early adoption, and expanding cycles of promotion through mass media stories represents both a protocological foundation for a novel system of money, and a simulacrum of the political-economic logic of ludocapitalism. I believe that the potential for Bitcoin as a mass-mediated political model, and the ramifications of the expansion of its "one-CPU-one-vote" paradigm of technologically-mediated governance, still have yet to be fully developed in practice.
