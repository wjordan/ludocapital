\chapter{Play and Freedom}
\label{play}
\section*{Introduction}
In this chapter I develop the concept of game-playing as central to the formation of a ludocapitalist subject, a concept that has gained renewed relevance through the rise of digital play forms. From the start of the Cold War to the present, game-playing has developed into a mediating concept between the human subject and the computing imaginary in contemporary technoculture. For post-industrial knowledge workers, play can no longer be conceived as the humanist negation of the relationship between wage-labor and capital characteristic of industrial capitalism, as rhetorics of play are increasingly invoked from within hybrid processes of creative production. As an academic field of game studies has rapidly emerged in the twentieth century as the ideological and pedagogical counterpart to the digital game industry, game-playing has become visibly incorporated into capitalism as a broadly progressive movement, including organizational strategies for flexible and decentered management and new media forms of interactive mass entertainment and pedagogy.

Although the specific forms and techniques of contemporary digital play are undoubtedly novel, I recognize the tension between traditional work practices and liberatory forms of play as a recurring theme with a long history in Western thought. With the concept of the \emph{ludocapitalist} subject, I mean to emphasize the under-theorized link between an idealized concept of play and the free, autonomous human subject at the center of Enlightenment discourse---a discourse that, inverted through the Marxist tradition into an ideological instantiation of the historically-specific capitalist mode of production, also dialectically constitutes the horizon of critical thought and action. By theorizing the historical constitution of play idealism within Enlightenment thought, I form a critique of game-based metaphors that aim to structure institutional experience around the idealization of specific developmental or aesthetic qualities deemed essential to the human subject. Today, similar rhetoric surrounding digital play forms what I understand as the \emph{posthuman} ludocapitalist subject, which reinscribes elements of classical and Romantic idealizations of play within post-industrial forms of instrumental reason, producing an idealized intersection of universal human agency with the sociotechnical milieu of the computer.

Based on a critique of the ludocapitalist subject implicitly invoked by game studies, I offer a \emph{critical ludology} as an alternative disciplinary orientation for the field. My argument is structured around three interrelated claims. First, implicit in the term \emph{ludocapitalism} is my claim that the concept of game-playing is not antagonistic or external to the contemporary logic of digital capital, but that it constitutes its essential characteristic and primary productive force. The transformation of work into play no longer holds the radical humanist promise of freedom from capitalist exploitation that it once did for Utopian socialists such as Fourier; rather, play has become generalized within capitalism itself as a discursive strategy for the indirect exercise of power through therapeutic, pedagogical and other institutional forms.\footnote{ See \citeauthor{Andersen2009-yx}, \citetitle{Andersen2009-yx}.} In contemporary technoculture, the "gamification" of bureaucracy has developed a fully-rationalized model of game design into a full-fledged science of management and behavioral psychology, tasked with isolating elements of intrinsic motivation that optimize the performance and productivity of a flexible workforce according to the direct needs of post-industrial organization.

I further claim that the specific contemporary form of this fusion of play and capital runs largely parallel to the fusion of \emph{software} and capital in the computerization of society from the Cold War era to the present. Contemporary metaphors of game-playing and software were both substantially influenced by the RAND Corporation's operations research through the 1950s and 1960s, when the modern science of instrumental rationalization known as game theory developed a mutual orientation with the combinatorial power of computing machines to produce models and simulations of social behavior. In this now-paradigmatic formation, the notions of \emph{game} (idealized as an infallible poker player maximizing his outcome within a formal rule-based system) and \emph{software} (a metaphor originally extending the instrumental logic of computing hardware into the "soft" domain of human factors) became two sides of the same research program to model and organize human behavior according to a model of rationality constructed in the image of electronic computing machines.

My third claim is that play is idealized as a symbol of humanity through discourses that link play to such autonomous, value-laden concepts as learning, empowerment, socialization, creativity and beauty. Although play contains paradoxical elements in its serious negation of seriousness, I argue that the dialectic of play has historically implied a much loftier concept of humanity than the abstract form of meta-communication that Bateson observed in animals play-fighting. Recognition of objects or activities as games or play often imply humanistic ideologies of progress, development and freedom. Such ideologies are complicit in the production of a transcendental human player-subject, a figure that, for the German idealist Schiller, "carries out the will of all through the nature of the individual" \autocite*[138]{Schiller2004-if}. This is where the critique of play gains its political-ethical significance. While Schillerian idealization of play produces an aesthetic ideology capable of mediating between the work of art and the state apparatus, similar idealizations of digital play today mediate between game design and the production of technological infrastructure resulting from the institutional implementation of such "serious games."

My discussion within this chapter proceeds as follows:

First, I establish an intersection between a progressive game-playing idealism and a pervasive computational imaginary that has structured the quickly-expanding field of digital game studies. This paradigm, which Sicart identifies as "proceduralism" and associates with Bogost's theories of procedurality, is an idealistic conflation of both game and computer as orthogonal systems of rule-based rationality. Drawing upon Galloway's recent critique of Object-Oriented Ontology, I argue that this proceduralist position lacks a productive concept of political agency and, under the guise of neutrality, legitimates and reinscribes existing technocratic power formations. In response, I call for a critical ludology that contextualizes this fusion of game-playing and software as the dominant liberal-humanist rationality of our era.

Next, I argue that this pervasive position within contemporary digital game studies is founded on an idealism that runs parallel to the classical enlightenment position within German philosophy. I relate the present state of digital play rhetoric to Schiller's idealization of play as an authentic expression of the aesthetic perfection of humanity, drawing parallels between his vision of the aesthetic state as a concrete realization of play as beauty and contemporary idealistic exaltations of gamification and gameful design.

Third, I establish the contours of an approach to digital game-playing forming the basis of a critical theory of the ludocapitalist subject. Here, I integrate three related fields of analysis: social theories of game-playing ontology, critical theories of the posthuman subject, and material histories of digital game production.

Finally, I apply this integrated approach toward a comparative analysis of two quasi-game software projects, Second Life and Minecraft, both of which are inadequately understood from a formalist or proceduralist approach.

Through this analysis, I offer two theses derived from the critical ludology developed in this chapter: first, the material history of each game's production relates to the ideal player-subject cultivated by the game environment, with significant parallels between the idealized agency of developers and players; second, that each game's form of idealized free play, as an interventional deviation from mundane labor, constitutes its ideological function. My comparison of Second Life's techno-liberal model of digital entrepreneurship against Minecraft's alternative neo-Renaissance model of digital craftsmanship reveals a contradictory, contested, and ever-changing concept of game-playing as integral to contemporary ludocapitalism, where the ludic quality of software refers to an ongoing ideological discourse about the ideal (post)human values our digital media should value, encourage and produce within its player-subjects.

\section{Play and Procedurality}
As the communications comprising our everyday lives increasingly transpire within the complex domain of personal networked computing devices, the industrial culture of commercial computer games increasingly impinges upon the everyday social realities of a growing public. According to a growing number of experts in professional game design and development, the compelling power of digital games to not only entertain but also to engage, motivate, and persuade is the key to a computer-powered reformulation of our core social institutions, from schooling to corporate training, health care, advertising and political campaigning.

The bureaucratic concept of "effective procedure," epitomized since Turing by the abstract, universal logic of the digital computing machine, has become the fundamental substrate of our present age. In his exposition of "procedural rhetoric" in \citetitle{Bogost2007-ow}, game designer and media theorist Ian Bogost has stated the injunction that "we must recognize the persuasive and expressive power of procedurality" \autocite*[340]{Bogost2007-ow}, particularly as expressed by the medium of the videogame, as he cites the rise of persuasive videogames being successfully deployed across a variety of social institutions extending far beyond the entertainment industry to which it had once been restricted.

In the field of digital game studies, this "power of procedurality" that Bogost emphasizes is now often accepted as the structuring force establishing the contemporary social field, a power often construed as particularly appropriate, even exceptional, to the nascent videogame medium. In this computational-procedural worldview, social action is conceived as designing a set of rules to produce a desired outcome from an audience of game-players, whether the goal is to persuade, train, or entertain. The human agent at the center of this new social reconfiguration, the game designer, is the new professional class capable of playing with this power.

This worldview has not only been remarkably influential across both academic digital game studies and the videogame industry but, as \citeauthor{Sicart2011-dd} notes in \citetitle{Sicart2011-dd}, it has also "helped deepen the understanding of some important notions on the ontology of games." Seen as a "continuation of the formalist work laid out by the original >ludologists<" such as Juul and Frasca who helped establish the boundaries of the contemporary field of game studies, procedurality expands the ontology of such earlier formalist approaches from a narrow, ahistorical focus on explicit game rules to incorporate entire systems of social meaning into its ontological horizon, while preserving the exceptionalist argument for the power of videogames that initially justified it as a unique and privileged domain of study.\footnote{ I examine this early academic literature of formalist studies of videogames in depth in relation to the Tetris videogame in \cref{tetris}.}

However, Sicart also notes the limitations of understanding games as procedural systems that act upon a circumscribed field of meaning: by deriving the entirety of a game's meaning and significance from its rule-based essence, such a perspective risks stripping all transformative agency from the player-subject. When playing a well-designed game produced by an expert within a fixed medium to be consumed by an undifferentiated audience, the player's task is reduced to merely "actively complete the meaning suggested and guided by the rules." The result of the proceduralist perspective, then, is to validate and reinforce a modern practice of game design as the art of managing a docile, receptive, largely powerless mass audience, whose "play" is circumscribed as raw material for the game designer to manipulate through the artful application of effective procedural rhetoric.

Through his critique, Sicart challenges the field of game studies to supplement the productive contributions of proceduralist orientation to games with a critical perspective that favors attention to the ethical and political possibilities rooted in forms of free play specifically opposed to the instrumental rationality of industrial game production. My reason for focusing on this meta-discourse on games taking place in the field of digital game studies is to critique the naturalization of an ontological discourse that risks covertly reinscribing the existing hierarchies of ludocapitalist power. The intent of such a meta-analysis is to challenge the complicity of academic digital game scholarship with systems of cultural hegemony within which it finds itself playing an increasingly active and pivotal role. \citeauthor{McAllister2004-mr} places such scholarship alongside production agrees that such a disciplinary self-awareness is necessary: "analyses of the computer game complex should not only influence other critics and scholars but should also be used by them to take an active role in influencing the computer game complex itself" \autocite[49]{McAllister2004-mr}.

In order for my analysis to take an ethical and political position capable of distinguishing itself from a dominant, free-market rationality, I situate the concept of game-playing not just in relation to a game's isolated rhetorical content transmitted from designer to player, but also in relation to the human qualities of the player-subject that particular games preserve or produce through the overall material and social conditions of their production. I call this method of analysis "critical ludology," denoting an intersection of critical theory (specifically, a tradition of sociocultural critique roughly extending through Kant, Marx, the Frankfurt school and its successors within poststructuralism and cultural studies) with ludology (conceived both broadly as the study of game-playing and more narrowly as the academic/ideological wing of the contemporary computer game complex).

\subsection*{Critical Ludology}
I position critical ludology metaphorically somewhere between \emph{defragging} and \emph{deconstruction}. The computer filesystem metaphor of defragging, referring to the algorithmic analysis and reorganization of a system's diffuse, constitutive fragments, seems an appropriate metaphor for the systematic manipulation of complex unit operations of social-technical meaning articulated by the proceduralist perspective.\footnote{
  The Digital Game Research Association organized its 2013 conference under the theme "Defragging Game Studies," reflecting a certain degree of popularity and resonance of this metaphor within the field.
}
Deconstruction, a term coined by Derrida in the late 1960s that self-consciously marks a philosophical event related to the undoing, decomposing, and desedimenting of structures, is also based on a technical methodological metaphor, albeit orthogonal to defragmentation as it gestures in the opposite direction: away from unified structure and ontology, rather than towards it.

Critical ludology is an approach that refuses to consider a game as a ready-made object or system of technocultural meaning, but instead begins and ends with consideration of game-playing as an organizing mode of humanistic inquiry that extends beyond cultural products to be equally implicated in industrial and academic actors. Rather than subordinating game studies to established social-scientific methodologies, a study of games could self-referentially look to its own rich, ambiguous, historically-grounded discourse of game-playing to construct and delineate its own analytical orientation, one not merely reducible to a technical procedure of behavioral engineering. Game-playing can be researched and practiced as a distinct discursive form of its own, articulating a dialectic of human experience otherwise lost within instrumental approaches. In this way, any exceptionalism unifying the game studies field would be based not on binding game-playing to the repressive ideology of a computational-procedural worldview, but on a discursive form that represents a human-oriented play of the world, sensitive to a more politically-aligned social history.

\subsection*{The Game-Playing Object}
Contemporary interpretations of game-playing, both regarding what an instance of game-playing \emph{is} in our ontologies and definitions, and also regarding how instances of game-playing \emph{ought to be} realized through our politics and ethics, often passively reiterate common ideological refrains of our dominant capitalist mode of production. For an example, let us recall Juul's remarks from 2003 on the "Heart of Gameness": \blockquote{
  Why is there an affinity between computers and games? First of all, because games are a transmedial phenomenon.…|G|ames are not tied to a specific set of material devices, but to the computational processing of data. Secondly, because the well-defined character of game rules means that computers can process them. It is then one of the stranger ironies of human history, that the games played and developed over thousands of years have turned out to fit the modern digital computer so well. \autocite*{Juul2003-sx}
  }
The "irony" Juul observes here is rather a tautological consequence of the post-Fordist lens implicitly used to frame his object: games, defined as procedural systems ontologically tied to the "computational processing of data," will certainly "fit the modern digital computer so well."

This situation makes a materialist, historical contextualization of those concepts that structure our game-playing objects an important counterbalancing critique of their supposedly neutral deployment. Revisiting the historical development of concepts of game-playing is therefore a key step in aligning a critical ludology with a materialism that resists the imposition of ahistorical structural metaphors, and that allows us to self-consciously examine its normative relation to cultural hegemony.

Here, I find the critique of play that Derrida set forth in his famous response to the structuralist anthropology of Levi-Strauss in \citetitle{Derrida1978-ix} to be worth a closer look. \citeauthor{Derrida1978-ix} described Levi-Strauss's references to the concept of game-playing throughout his work as "always caught up in a tension.…Tension with history, first of all" \autocite[367]{Derrida1978-ix}. Levi-Strauss's "structuralist" moment, Derrida argues, risks "falling back into an ahistoricism of a classical type" which "compels a neutralization of time and history":
\blockcquote[368]{Derrida1978-ix}{
  The appearance of a new structure, of an original system, always comes about,…by a rupture with its past, its origin, and its cause. Therefore one can describe what is peculiar to the structural organization only by not taking into account, in the very moment of this description, its past conditions: by omitting to posit the problem of the transition from one structure to another, by putting history between brackets.}
Theories that posit a single, ahistorical structure to their object of study are unable to account for the historical and ideological limitations of their mode of analysis. This leads to the inevitable conceit that such structure, as Derrida quotes Levi-Strauss on language, "could only have been born in one fell swoop" at the dawn of humanity.

While this first component of Derrida's argument speaks directly to the ahistoricity of ludology's early formalisms (I would also include some of the field's precursors such as philosopher of sport Bernard Suits's game-playing ontologies in this group), the next point is more subtle. Noting the "tension between play and presence" in structuralism, Derrida contrasts the structuralist "ethic of nostalgia for origins" with a "Nietzschean affirmation,…of the play of the world and of the innocence of becoming, the affirmation of a world of signs without fault, without truth, and without origin" \autocite[369]{Derrida1978-ix}. He observes that these "two interpretations of interpretation, of structure, of sign, of play" are irreducible to each other: there is not "any question of \emph{choosing}," but we should rather "try to conceive of the common ground" \autocite[369--70]{Derrida1978-ix}. This dialectical understanding of play thus carves out an ethical position: If the former interpretation risks the mythical imposition of dogmatic structure without recourse to social history, the latter risks foreclosing the possibility of ethical-political action by reducing the play of the world to the will to power.

It is in light of this second interpretation that I read Galloway's trenchant critiques of the broader ontological project of Object-Oriented Ontology, the philosophical position underlying Bogost's proceduralist mode of videogame criticism. In an essay, \citetitle{Galloway2013-ht}, \citeauthor{Galloway2013-ht} focuses on the recent resurgence of realism within continental philosophy in figures such as Latour and De Landa and extending to the speculative realism of Meillasoux and Harman and the object-oriented ontology of Bryant and Bogost, asking: "Why, within the current renaissance of research in continental philosophy, is there a coincidence between the structure of ontological systems and the structure of the most highly evolved technologies of post-Fordist capitalism?" \autocite[347]{Galloway2013-ht}. His analysis draws a distinction between the claims of the new philosophical realism and the tradition of materialist critical theory since Kant and Marx. With respect to this distinction along the "task of the political," Galloway argues that "Realism is an unaligned politics.…By contrast, materialism is an aligned politics" \autocite[365--6]{Galloway2013-ht}. Unaligned political projects such as realism, he argues, are "unencumbered by the moral law" and "exist as mercenaries, often jumping the gap between friend and enemy" \autocite[365]{Galloway2013-ht}. In contrast, materialist projects, "aligned" with "something like an absolute moral sphere (history, the social totality)," are "tethered to a moral yardstick and equipped with an ethical mechanic able to pursue it" \autocite[365-6]{Galloway2013-ht}.

The unaligned politics of object-oriented ontology belies its implicit correlation with post-Fordist capitalism, which inevitably emerges through a pragmatist adoption of concepts oriented to the dominant mode of production. Such a correlation is present in the very etymological history of "Object-Oriented Ontology," as Latour recounted in 2005: "A few years ago, computer scientists invented the marvelous expression of >object-oriented< software to describe a new way to program their computers. We wish to use this metaphor to ask the question: >What would an object-oriented democracy look like?<" \autocite[14--5]{Latour2005-vc}. \citeauthor{Galloway2013-ht} responds in his essay: "But these democracies already exist.…Their democracy has little relation to the rule of the people, only the rule of the market" \autocite[363]{Galloway2013-ht}.\footnote{
   Galloway further remarks upon Latour and Harman's disingenuous notion of an anti-humanist "democracy of objects" in a blog post: "It is actually an anti-democratization, in two ways. First, because it removes the point of decision from people (the demos) to the object world at large.…And second because it allows certain objects to have more natural >gravity< than others, thus in essence letting their >votes< count double or triple" \autocite{Galloway2012-da}.
  }
This political neutralization thus inherent in OOO would be merely an irrelevant (but harmless) theoretical exercise, if it were possible for such a "pure ontology" to exist in isolation from the \emph{ought} of embedded ethical practice. However, such a categorical is/ought distinction is as suspect as the human/nonhuman distinction the anti-correlationist agenda hopes to break down, and Galloway situates this point of contention along realist/materialist lines: in contrast to the realist belief that ontology could and should not be politically aligned, materialists believe in a necessary unity of theory and practice, that ontology is inseparable from its material sociopolitical existence.\footnote{
  This unity is often expressed through chiasmus, e.g., by Kant: "Thoughts without content are empty; intuitions without concepts are blind" \autocite*[86]{Kant2008-hh}. Marx often employs formal antimetabole: "It is not enough for thought to strive for realization, reality must itself strive towards thought" \autocite*[Introduction]{Marx1970-hw}.
}
\citeauthor{Galloway2013-ht} sides with the materialist position, that "the uncoupling of the ontological realm from the political realm is not entirely neutral," but is rather "an ideological strategy bent unwittingly or not on the elimination of competing discourses" \autocite[357]{Galloway2013-ht}.

Indeed, it is within the application of such object-oriented philosophies within the game industry, and by extension to the growing industry-aligned "gamification" of other social institutions, that the application of object-oriented engineering should be taken to task on account of its unaligned politics. In particular, I view \citeauthor{Bogost2006-ec}'s proceduralism in terms of his own recent reflections on his philosophical position, as an "extension beyond first principles, into the practice of metaphysics itself" of object-oriented ontology, as a "pragmatic" or "applied speculative realism, an object-oriented engineering" \autocite*[674--7]{Bogost2012-wr}.

In order to move beyond such ethical or ontological impasses in the "object" of game-playing, I wish to frame game-playing not as a structurally-determined, ahistorical concept that ventriloquizes information capitalism but rather as the nexus of a discursive struggle rife with social tension, ambiguity, and an open-ended yet historically-bounded meaning. In the next section, I paint a broader picture of game-playing by recovering a key moment in the concept's modern history, where the humanist intersection of aesthetics and politics is played out in a discourse whose traces are still at work in our present-day game-playing concept. This moment is found in the Enlightenment concept of \emph{spiel }[game/play] which occupies a crucial but oft-overlooked place in Kant's aesthetic philosophy, and is subsequently elevated to a secular symbol of humanity in Schiller's own aesthetic writings.

\section{Aesthetics of game-playing}
A history of the game-playing concept, "if written, would prove to be virtually coextensive with Western thought" \autocite[8]{Wilson1990-eu}. Game-playing is said to have existed since antiquity, with scholars observing elements of game-playing in ancient civilizations, animals, insects, and even the universe itself, where play "could be seen as one of many forms of evolving, emerging, self-organizing, complex dynamic systems" \autocite[251]{Brown1998}. The association of play with some of the core institutions of Western society is easily recognized in the Greek etymology of the associated words \emph{paideia} (education/culture), \emph{paidia} (play/game/pastime/sport), and \emph{pais} (child). Accordingly, play functions as a pivotal concept in Plato's Republic. As \citeauthor{Spariosu1991} reads in his study of Hellenic thought, Plato's concept of play is both agonistic and educational, linked to both the mimetic function of art and the dialectic of logo-rational argument, and always precariously shifting between the serious and not-serious: "Plato is no doubt a serious man, but he needs play in order to remain serious. Through play he can have his cake and eat it too, for the dialogic form allows him to say what he cannot say" \autocite[192]{Spariosu1991}.

\subsection*{Play and Idealism}
Within Kant's critical system, the concept of \emph{spiel} plays a key role in his \citetitle{Kant1987-coj} concerning aesthetic judgments of taste. For Kant, "pure" judgments of taste by which we declare something to be beautiful must be disinterested, so that the feeling of pleasure produced by the cognition of an object corresponds not to the particular circumstances or an individual's personal interest but to the possibility of its universal communicability.\footnote{
  This Kantian idea of a necessary universal communicability of human reason survives in modern critical philosophy through e.g., Habermas's universal pragmatics, where intersubjective consensus-formation depends on counterfactual presuppositions as a necessary pre-linguistic condition for the possibility of communicative understanding and action.
  }
\emph{Spiel} is the term that represents a spontaneous, harmonious relation of the imagination to the understanding in pure judgments of taste, which, being analogous to or symbolic of our moral law, forms the crucial link between the domains of pure and practical reason that comprised the first two Critiques: "The spontaneity in the play [\emph{Spiele}] of the cognitive powers, whose harmony with each other contains the basis of this pleasure, makes that concept of purposiveness suitable for mediating the connection of the domain of the concept of nature with that of the concept of freedom, as regards freedom's consequences, inasmuch as this harmony also promotes the mind's receptivity to moral feeling" \autocite[37-8]{Kant1987-coj}. Kant further qualifies this relation as one of "free play" [\emph{freien Spiele}], because it is only when the play of the imagination is freed from any particular rule of cognition that such play is universally communicable to cognition in general:
\blockcquote[62]{Kant1987-coj}{
  If a presentation by which an object is given is, in general, to become cognition, we need imagination to combine the manifold of intuition, and understanding to provide the unity of the concept uniting the [component] presentations. This state of free play of the cognitive powers, accompanying a presentation by which an object is given, must be universally communicable; for cognition, the determination of the object with which given presentations are to harmonize (in any subject whatever) is the only way of presenting that holds for everyone.
  }
Schiller raised Kant's conception of free play to an aesthetic ideal of humanity's enlightenment in his adaptation of Kant's aesthetics in Letters upon the Aesthetic Education of Man. For Schiller, a renowned poet and playwright himself, play was much grander than a mere educational technique to be carefully controlled and attributed primarily to children, as it was predominantly identified in the Greek tradition. Rather, play became for him the founding principle through which the idea of humanity, and its corresponding >aesthetic state< of political organization, is made possible. This marks a significant transformation both from the Greek concept of play and Kant's own abstraction and, as \citeauthor{DeMan1996-ks} comments in \citetitle{DeMan1996-ks}, it is all but impossible to deny or escape the resulting influence of Schiller's paradigmatic aesthetic ideology on our modern, liberal institutions of knowledge: "Whatever writing we do, whatever way we have of talking about art, whatever way we have of teaching, whatever justification we give ourselves for teaching, whatever the standards are and the values by means of which we teach, they are more than ever and profoundly Schillerian" \autocite[142]{DeMan1996-ks}.

According an interpretive tradition initiated by de Man and extended by Warminski, Schiller offered a mis-reading of Kant predicated upon a psychological humanization of Kant's transcendental critiques, allowing Schiller to make a much more ambitious, universal statement of play as a symbol of humanity and model for political action than Kant's work ever suggested. Schiller's play concept is advanced through a symptomatic misappropriation of Kant's aesthetic of the sublime---one that "can be taken as the idealist operation," according to \citeauthor{Warminski2001-kd}, of "setting up a sublime problematic and >solving< it by recourse to the beautiful" \autocite[970]{Warminski2001-kd}. Schiller posits a binary distinction between two opposing drives/instincts [\emph{trieb}]---sensual [\emph{stofftrieb}] and formal [\emph{formtrieb}]---as a sublime problematic, one that can only be reconciled as the result of an impossibly infinite operation. These two drives circumscribe the totality of human existence: "it is these two impulses that exhaust the conception of humanity" \autocite*[67]{Schiller2004-if}. The problematic of the concept of humanity so defined, then, is the seemingly impossible reconciliation of two opposed instincts within a single, unified, mediating concept: "This reciprocal relation of both impulses is, admittedly, a problem of the reason, which Man will be able to solve fully only in the perfection of his being. It is in the truest sense of the term the idea of his humanity, and consequently something infinite to which he can approximate ever nearer in the course of time, without ever reaching it" \autocite[73]{Schiller2004-if}. The solution to this problem is the play-drive [\emph{spieltrieb}], which is this "reciprocal relation."

As Warminski observes, this "problem of the reason" to which Schiller offers the solution of the play-drive is precisely the task of the mathematical sublime presented by Kant---the faculty of reason's attempt to comprehend the absolutely large in its totality in a single intuition, a task resulting in aesthetic judgment only through its tragic failure. However, a "sleight-of-hand" \autocite*[966]{Warminski2001-kd} occurs in Schiller's version---the fulfillment of the problematic of the sublime is transformed into an object of beauty, one easily associated with wordly objects, figures, and common-sense concepts, and solved by recourse to play. Although Kant's concept of play was central in aesthetic judgments of taste in objects of beauty, it had no place in Kant's concept of the sublime, for which "it seems to be seriousness, rather than play, in the imagination's activity" \autocite*[98]{Kant1987-coj}. Warminski notes that Schiller's linking play to the sublime rather than the beautiful is no mere mistake, since the entire aesthetic ideology hinges upon a sleight-of-hand enabling a solution amounting to the ideal of humanity directly reducible to practical knowledge and moral action, forming the basis of his aesthetic education of man.

Schiller's key concept of the play-drive, and its precarious relation to that of the human and its empirical nature, becomes the conduit through which the concept of humanity is able to reach the transcendental status it achieves in its ideal perfection. Schiller claims that humanity is able to attain such an ideal perfection only when it becomes most playful: "For, to declare it once and for all, Man plays only when he is in the full sense of the word a man, \emph{and he is only wholly Man when he is playing}" \autocite*[80]{Schiller2004-if}. Beyond serving as a model of humanity, play can thus be viewed as a symbol for the system of total social organization upon which Schiller's aesthetic state is founded.

Schiller's rhetoric of play is powerful and emancipatory, and has had an enduring affinity with subsequent developments in critical theory, particularly in the "aesthetic dimension" of Marcuse's left-radicalism that fueled the French New Left in the 1960s. In \citetitle{Marcuse1974-oy}, Marcuse's reading of Kant's aesthetic theory closely followed Schiller's interpretation, finding that "the aesthetic reconciliation implies strengthening sensuousness as against the tyranny of reason and, ultimately, even calls for the liberation of sensuousness from the repressive domination of reason" \autocite[179]{Marcuse1974-oy}. The freedom Schiller finds in the play-instinct is translated by Marcuse into the liberation of the repressive order found in the de-sublimation of reason and the transformation of labor into play.

\subsection*{Digital Gamefulness}
Today, many play-infused social movements gaining currency in our ludocapitalist era continue to draw upon similar aesthetic notions of game-playing, mobilizing concepts such as "serious games" and "gamification" around the aesthetic value of game-playing in popular culture to serve predetermined political or other more mundane material purposes. Such movements attempt to both essentialize and revalue a universal, progressive concept of particular kinds of game-playing in order to legitimate prescribed reorganizations of cultural and intellectual work. As a corporate game designer and futurist thinker, McGonigal is both representative of this shared worldview of contemporary game design practice as a powerful tool for social change, and unique in the extent of her idealistic enthusiasm for a more positive future engineered and optimized in its image. In \citetitle{McGonigal2011}, \citeauthor{McGonigal2011} tells the story of the progressive work of game designers as the world's greatest "happiness engineers," and their unprecedented rise to power and status as the vanguard of the growing digital games industry. Through their historically unparalleled ability to create flexible, fun approaches to reorganizing society's modern institutions and develop novel, collaborative, interactive solutions to complex global problems, McGonigal claims that the class of game designers she represents has the unique potential, if not a social and ethical mandate, to leverage the power of the games they design to remake the world for the better: "Life is hard, and games make it better. Organizing large groups of people is also hard---and games make it easier" \autocite[Conclusion]{McGonigal2011}. In particular, the progressive work of game designers McGonigal champions avoid engaging political questions of cultural struggle, critique or messy revolutions, and are rather a matter of objectively engineering products and services that provide corporate-friendly happiness from within our existing industrial organization: "The [commercial game] industry has consistently proven itself, and it will continue to be, our single best research laboratory for discovering new ways to reliably and efficiently engineer optimal human happiness" \autocite[Conclusion]{McGonigal2011}.

As one of the more popular exaltations of modern game design, McGonigal's ludic message is a compelling example of the persuasive, universal appeal of game-playing idealism today. Through a transvaluation of the public value of games combined with a populist message, e.g.: "We are living in a world full of games and gamers" \autocite[Introduction]{McGonigal2011}, McGonigal constructs a ludic Utopia, imagining the future modeled after an aesthetic ideal of game-playing:
\blockcquote[Introduction]{McGonigal2011}{
  What if we decided to use everything we know about game design to fix what's wrong with reality? What if we started to live our lives like gamers, lead our real businesses and communities like game designers, and think about solving real-world problems like computer and video game theorists? Imagine a near future in which most of the real world works more like a game. But is it even possible to create this future? Would it be a reality we would be happier to live in? Would it make the world a better place?
}
In her vision of a new, more game-like reality principle to replace everything that's wrong in our broken reality, McGonigal's argument unwittingly parallels that of German idealism, a ludic Utopia echoing Schiller's vision of an aesthetic state, grounded in an experience of game-playing fundamental to human nature, that can make the ideal society a reality that "carries out the will of all through the nature of the individual" \autocite[138]{Schiller2004-if}.

Despite the potential for such game-playing idealism to inspire revolutionary politics or a challenge to a repressive status quo through art, its ready mobilization of aesthetics can harbor similar dangers of an unaligned politics that we saw in the previous discussion of realist philosophy. In contrast to the realist disavowal of any humanistic moral ground to which a politics can be linked, an idealist position explicitly links humanist aesthetics to political action; however, the idealist's symbol of humanity is often a mythical figure or programmatic psychology, rather than a more democratic or dialogic morality tethered to human history, or a more diverse concept of cultural or social totality. For Schiller, the totality of human nature is contained in the symbol of beauty found in Greek culture from which modern man, wounded by the fragmentation of the individual in modern culture, had fallen.\footnote{
  Jung comments on Schiller's aesthetics that "only an incorrigible idealist and optimist could conceive the >totality< of human nature as simply >beautiful.<…From this conceptual immaturity and inadequacy,…it is not at all clear how this mediatory state shall be established" \autocite[161]{Jung1923}.
  }
For \citeauthor{McGonigal2011}, this totality is contained in a positivist psychological science of happiness, which posits a set of universal, apolitical and ahistorical "genuine human needs that the real world is currently unable to satisfy" \autocite[Introduction]{McGonigal2011}.

Summing up the last two sections, we can distinguish a materialist critical ludology from the dangerous tendencies of an object-oriented realism represented by Bogost, and a play-idealism represented by Schiller and McGonigal. With these distinctions in mind, in the next section I will offer some positive indications as to what a materialist ludology might look like by focusing on the conditions of digital play in contemporary technoculture, and offer some suggestions as to how to move the development of a ludic posthuman subject forward.

\section{Digital play and the Ludic Posthuman}
Earlier I briefly discussed the limitation of game-playing ontologies, in that any such essential ontology of games reflects an implicit aesthetic judgment privileging particular forms of activity among others. If we assert that games require quantifiable outcomes or a goal, for example, our perspective would marginalize works that leave open the interpretation of player intention or evaluation of outcomes. If our concept of game requires a system of well-defined rules, then we would fail to recognize playful activities that emphasize improvisation, ambiguity or indeterminacy. If we stress the voluntary nature of games or the player's necessary emotional attachment, we might suppress discourses on addiction or compulsive gaming, or coercive or professional play. If we emphasize an essential distinction between work and play, or stress the safety of games and their insulation from real-world consequences, we might turn a blind eye to social practices which attach material or social rewards or consequences to game-playing performance, or which incorporate games into productive labor practices or market economies of exchange. If we stress the element of competition or conflict in games, then we might marginalize those games in which cooperation, coordination or creative expression are instead emphasized (such as the games of DeKoven's New Games movement).

On the other hand, a deconstructive delimiting of ontology is not to say that an equally prevalent "anything goes" approach to game-playing discourse would be any more productive. As Wilson notes, \blockquote{
  in some critical discourse play and game concepts seem to behave like magic motifs in traditional folk literature in that, like an endless sausage, an unstinting goose, or an unemptiable bowl, they not only dominate the other elements in the scene but are ontologically inexhaustible, swallowing, like black holes, all other analytic lexica. Once one has the concepts of play and game firmly in hand, it might appear unnecessary to talk about anything else and, for that matter, anything else may be talked of in precisely those terms. Play and game can fill the conceptual horizon. \autocite*[7]{Wilson1990-eu}
}

In 2006, this problem of a debilitating game studies pluralism had become a central concern of the digital game studies community. In the inaugural issue of the journal \emph{Games and Culture}, Patrick Crogan criticized what he saw as a naive pluralism taking shape, in which "different conceptions of the object of study operate in the various disciplinary and regional configurations of academic communities interested in computer games," a method that "tends towards Babel and not toward a synthesizing perspective on what underlies this diversity" \autocite*[73]{Crogan2006}. In order to resist such neutral orientations to contemporary technoculture that emerge through such an uncritical embrace of pluralism, Crogan argued, the question that animates the study of computer games must be posed as the question of the nature of computer games as part of "life" in contemporary technoculture: \blockcquote[76]{Crogan2006}{
  |T|he thinking of technocultural forms, including all those emanating from today's defining technology, must also always be led to an interrogation of technoculture, culture, technology, and >life< today and into the future. For every thing >we< make---computer games themselves and the research we do about them (which also >makes< them)---is an answer to the question of >life.<
}
In his own anthropological response to the field's ontological crisis, \citeauthor{Malaby2007} offered an understanding of games as grounded in human practice and fundamentally processual (while not essentially procedural), providing his own definition of game-playing as "a semibounded and socially legitimate domain of contrived contingency that generates interpretable outcomes" \autocite[96]{Malaby2007}. He links social self-understanding to the type of activity recognized as play, hedging on any further essentialist delineation. This emphasis shifts the central question of play from an ontological tug-of-war towards observing the local processes by which particular play-forms become socially legitimate among its players. To study a game as a game is merely to recognize a particular "form of life" as socially legitimate.\footnote{
  The expression "form of life" [\emph{lebensform}] comes from \citeauthor{wittgenstein2001}'s discussion of language-games in \citetitle{wittgenstein2001}: "To imagine a language means to imagine a |\emph{lebensform}|" \autocite[7\textsuperscript{e}]{wittgenstein2001},  which I find echoed in Crogan's and Malaby's positions.
}

A critical ludology that takes game-playing as an organizing metaphor for its own knowledge-practice, then, should reflect on both the grounding of knowledge and the exhibition of moral and political freedom in the concept of humanity enacted through its material. When we define, discuss or play games, we are forming the conditions through which our discourse is able to observe and construct the type of activity such games make possible. As both Schiller's aesthetic state and McGonigal's re-engineered reality demonstrate, ideals of game-playing are no frivolous matter and can evoke an unparalleled enthusiasm, or terror, particularly from those whose marginalized voices or activities are not recognized as normative, legitimate forms of play matching particular ideals.

\subsection*{Posthuman Play Aesthetics}
How, then, can we articulate the forms of life underlying digital game-playing without falling back upon neutral forms of technocultural validation? In addition to historical contextualization, we might contribute to an aesthetic of digital play neither in relation to a fixed Greek ideal of human beauty nor a mass-psychology of computer-engineered happiness, but to a "posthuman" subject (with perhaps greater emphasis on "human" than on "post-") that is something more socially conscious, politically capable, and materially diverse. Based upon the work of critical media theorists including Poster and Hayles, I argue that a posthuman subject includes a critique of the computational imaginary alongside the idealized concept of play.

In his earlier work on potential avenues of critical reflection for poststructuralist ethics and politics in the emerging milieu he identified in the title of his work \citetitle{Poster1990}, \citeauthor{Poster1990} problematized the burgeoning field of computer science's foundation on the computing machine as both subject and object: \blockcquote[147]{Poster1990}{
  The computer stands as the referent object to the discourse of Computer Science. As such it is in the position of the imaginary, the mirror of this science's false recognition and is invested with great signifying power, inscribed with transcendent status. I mean by this that Computer Science is to some degree dependent on computers in the way a child is dependent on its mother. The computer scientist cannot escape the relation to the computer; his or her identity is bound up with the computer. As the field of Computer Science develops, constituting the computer scientist in ever new ways through disciplinary practices, the relation to the computer remains one of misrecognition. Since Computer Science found its first identity through its relation to the computer, that identity remains part of the disciplinary protocol of the field, even if the actual object, the computer, changes significantly, even unrecognizably, in the course of the years.
}
Poster proceeds to characterize Computer Science in terms of its ideological function: \blockcquote[148]{Poster1990}{
  Computer Science then is then a discourse at the border of words and things, a dangerous discipline because it is founded on the confusion between the scientist and his or her object. The identity of the scientist and the computer are so close that a mirror effect may very easily come into play: the scientist projects intelligent subjectivity onto the computer and the computer then becomes the criterion by which to define intelligence, judge the scientist, outline the essence of humanity.…The imaginary foundation of computer science is,…essentialized as a closed discourse whose domain is spirit.
}
A posthuman figure more capable of resisting instrumental reason must be formed from an ensemble of the social conditions that produced not only the closed, transcendent discourse of the computer, but also its formative historical and social context that provide the conditions for an alternate discourse or critique. This position is occupied no longer by the classical liberal humanist subject that Schiller found in the Greek ideals of beauty, but rather by what Poster calls in his later work a "humachine": a fluid, social assemblage of humans and machines, unassailable to the subject/object distinction attributed to the relation between human and machine-as-tool that was possible in earlier media eras \autocite{Poster2006}. Far from being determined by universal forces of technological progress, however, Poster's concept of the humachine is still linked to history, culture and social movements, imbued with the potential for ethical-political action which demands critical attention and recognition in order to develop the corresponding public institutions to support them.

Hayles narrates a similar critical moment in contemporary technoculture through the more conventional term, "posthuman":
\blockquote{
  I understand human and posthuman to be historically specific constructions that emerge from different configurations of embodiment, technology, and culture. My reference point for the human is the tradition of liberal humanism; the posthuman appears when computation rather than possessive individualism is taken as the ground of being, a move that allows the posthuman to be seamlessly articulated with intelligent machines. \autocite*[34]{Hayles1999-de}
}
Although her vision of the posthuman ambivalently contains both terror (the "post-" prefix signifying the threat, possible but not inevitable, of the End of Man) and excitement (e.g., in new human-machine configurations that could alleviate problems with our inherited liberal humanism and its privileging of the disembodied subject), Hayles's ethical aim is for contingent posthuman becoming to be deliberated and crafted into a sustainable narrative that ensures our collective survival without reproducing structures of domination and oppression:
\blockcquote[5]{Hayles1999-de}{
  I view the present moment as a critical juncture when interventions might be made to keep disembodiment from being rewritten, once again, into prevailing concepts of subjectivity.…If my nightmare is a culture inhabited by posthumans who regard their bodies as fashion accessories rather than the ground of being, my dream is a version of the posthuman that embraces the possibilities of information technologies without being seduced by fantasies of unlimited power and disembodied immortality, that recognizes and celebrates finitude as a condition of human being, and that understands human life is embedded in a material world of great complexity, one on which we depend for our continued survival.
}
Critical concepts of the posthuman subject offered by Poster and Hayles both embrace new technological configurations while disputing any essential claim of computation as the ground of being, instead remaining receptive to alternate modes of literary thought, cultural tensions, and social struggle. I link these figures of the posthuman subject to game-playing activity by focusing on the material history of production and validation of forms of play that are either encouraged or suppressed in social-technical environments.

\citeauthor{Kline2003} offer such a promising historical-materialist model of the dominant dynamics of contemporary post-fordist, postmodern information capitalism in \citetitle{Kline2003}. Within a Marx-inspired circuit of capital production, commodification and consumption, they posit three interpenetrating cycles (or subcircuits) of cultural, technological, and marketing activity that characterize the contemporary field of production in the mediatized, global marketplace. Each of these cycles involves its own dynamic circulation of capital, channels and networks of communication and feedback, evolution of technical forms and processes, and emergent contradictions and crises. The circuit of technology is constituted by a relation between programmers and users structured by the medium of computing platforms; the cultural circuit is constituted by designers relating to players through the medium of games; and the marketing circuit is constituted by marketers and consumers relating through the medium of commodities. In this multifaceted model, the production of free play that would constitute an emancipatory posthuman potential is constrained by the more dominant construction of a gamer-subject interpellated by the overlapping circuits of capital. Here, the gamer is already preconfigured as a "player" within the cultural circuit of meanings prescribed by a professionally-designed procedural fiction, legally protected by international copyright and trademark registrations; as a "user" within a proprietary technological platform tailored toward the efficient, one-way distribution of authorized content in exchange for payment (or for the exchange of other monitored and monetizable value, as in the case of social network activity for example); and finally as a "consumer" within mediated distribution channels saturated with advertising, sales and cross promotions encouraging more frenzied and friction-free habits of consumption.

A critical ludology fits well alongside \citeauthor{Kline2003}'s analytical framework. The free play that is the focus of such a critical perspective is located outside the hegemonic flows of technocultural capital, or found within its moments of crisis or contradiction. Such play resists the instrumental rationality of capital accumulation and promotes democratic freedom for the great masses of gamer-subjects, whose forms of life are never freely at play, but are firmly constituted within the matrix of technical platforms and flows of capital. As opposed to the capital-intensive flow of commodified creativity produced by the commercial videogame industry, the type of free play described by its critique is to be found in aberrations or distortions of these cycles of ludocapital.

It is tempting to applaud an expanding sphere of popular game-playing activity promoted by industrial forms of game production, expanding the social impact of its creative energies to the corporate arena of capitalism through gamification and serious play initiatives. However, such game forms, motivated by a politically-neutralized, discourse of games and a focus on the instrumental deployment of procedural rhetoric within corporate-controlled digital media platforms, are already overdetermined by the institutions underwriting their production. We must look for the ethical possibility of posthuman free play elsewhere.

Paying attention to this notion of free play, in the next section I juxtapose brief inquiries into the aesthetics of Second Life and Minecraft, both game-like software projects but each also located outside the circuits of industrial entertainment, that offer glimpses of free play in their alternate and contrasting visions of posthuman subjectivity.

\section{Case Studies}
\subsection*{Homo Lindens: Second Life and Virtual Entrepreneurship}
Second life is an open-ended, three-dimensional virtual world environment initially launched in 2003 that has generated extensive interest from the academic community. Public interest in the project peaked sharply around 2007 \autocite{Google-sl}.
Although similar in some aspects to other persistent-world multiplayer games such as World of Warcraft and Everquest, Second Life was novel in two fundamental respects. First, Second Life lacked both an explicit goal or any hierarchical representations of power, wealth or status, instead encouraging its users, hailed as >residents< by its development studio Linden Lab, to interpret Second Life as its name implies, an alternative living space of everyday experience. Second, the world of Second Life explicitly facilitated works of end-user creativity, entrepreneurship and commercial exchange through experimental innovations in end user licensing agreements, content-creation tools and interfaces, and regulated virtual currency exchange markets.

Malaby observes that the impact of game culture on our society has truly risen to prominence as a post-bureaucratic response to traditional organizational forms. Games, he suggests, are a source of organizational (dis)order that have proven a useful trope for the meta-management of complexity, a trope that affects not only the product of game development but also the production process itself. The particular aesthetics of free play that Second Life embodies is therefore most visible only when the virtual world is placed alongside the real-world corporate culture which owns, manages and produces it. In \citetitle{Malaby2011-my}, an ethnographic study of the parallel construction of Second Life alongside the dynamic, playful organization of Linden Lab, \citeauthor{Malaby2011-my} does just that, linking the "dual projects of Second life and Linden Lab as sites for individual, autonomous creativity for whom technology was a handmaiden" \autocite[78]{Malaby2011-my}. Alongside the production of a virtual space in which a certain model of an individual creative subject was facilitated through technology, he observes a parallel ideology underlying the organization of Linden Lab itself. With this duality in mind, he develops his observations into the critical notion of >technoliberalism,< denoting a dependency on technological tools for solving social and policy concerns: "Faith in the tool-making tool of computer programming practice served as the go-to practical means by which a public policy problem could be answered" \autocite[78]{Malaby2011-my}.

Second Life's website enthusiastically embraces a rhetoric of freedom and autonomy: "Enter a world with infinite possibilities and live a life without boundaries, guided only by your imagination" \autocite{SecondLife}. This rhetoric of technologically-facilitated individual and autonomous creative freedom, at work within both Second Life and Linden Lab, masked the unspoken, pervasive systems of implicit control Malaby saw within Linden Lab's internal decision-making practices as well as in the virtual world's construction. For example, espousing an office rhetoric of flat organization devoid of hierarchy, everyone's opinion at Linden Lab was encouraged and said to be given equal treatment. However, in the end according to Malaby, decisions were made according to the most >obvious< decisions based on what was >cool<---an unspoken dynamic of cultural capital that ran along informal but identifiable lines of power, typically culminating in the personal affinities of the company's CEO. In a similar fashion, the consumption-oriented, free-market capitalism promoted within Second Life was itself cultivated through programming, a technological mechanism of regulation that was fixed and unquestionable except through appeal to the engineers in charge of the code.

The boundaries of this form of free control are made particularly evident within the Second Life world through two primary tensions, which combined demonstrate the aesthetic orientation of the project in general. First, Linden Lab found itself in the contradictory situation of promoting open-ended entrepreneurial activity within its virtual world, while making various efforts to control the brand and message of this activity when it reflected poorly upon the service or conflicted with its public message. Second Life's culture of private property ownership and marketplace-oriented transactions of virtual goods was an innovation that resulted from Linden Lab's deliberate shaping of the legal and computational architecture of its virtual world. Linden Lab was progressive in its stated policy not to claim ownership of the creative works of individuals within their platform, and it could be said that this policy comprised one of the driving missions of the project from the start. Indeed, it was a bold and unprecedented move to treat creative activity within its virtual world with similar intellectual property protections as in the real world.

It is only against the backdrop of games understood as privatized, regulated spaces of overdetermined entertainment, refusing to grant any autonomy to its player-consumer audiences, that the uniqueness and constructedness of the legal protections that Linden Lab granted its >residents< can be recognized. World of Warcraft, for example, consistently asserts its contractual ownership over any creative activity conducted within its game environments. It is notable that this playful concession of rights to its players makes Second Life something other than a >game.< No longer hermetically sealed within a self-contained, rigorously policed ludic fiction, Second Life's >residents< were not only permitted but encouraged to construct hybrid social and professional identities, transferable between creative work performed in-world and the larger economic activity of the >First Life< surrounding it.

However, constraints upon this novel entrepreneurial freedom were not far below the surface. Commenting on the predominance of consumer fashion in Second Life, Malaby notes that "potential new users are told that they can >enjoy being whatever they want in Second Life,< but for most of them this seems to involve buying clothes and other items that thousands of others have bought as well" \autocite*[114]{Malaby2011-my}. Entering Second Life in early 2013, my own experience confirmed impressions of an electronic shopping mall. Virtual clothing store buildings were the most common in-world attraction, where displays of the latest designer clothing and accessories hung on the walls, available for purchase with a couple button clicks and a few hundred Linden dollars. For more conventional (and efficient) e-commerce shopping, the Second Life website prominently featured a marketplace with over two million virtual items listed for purchase. Digital rights management was coded as an inviolable, >natural< law of the virtual world itself, allowing every object for sale to be individually marked with automatically-enforced permissions on >copy,< >modify< and/or >transfer< operations, technically limiting the ability of users to creatively reuse items beyond their intended purpose as commodity objects. This commodified consumerism was "an inescapable value written into Second Life," Malaby notes, with its system of intellectual property rights a "core attribute of Second Life" that "so easily serves both the ideal of empowered creation and the ideal of consumption" \autocite[115]{Malaby2011-my}.

The second primary tension of Second Life's free control is found in its system architecture: despite a rhetoric of openness that allowed it to gain a great deal of traction among academic, corporate and educational institutions, the physical, real-world location of Second Life, as a networked grid of computing servers that mapped to two-dimensional coordinates in the Second Life world, was closely guarded by Linden Lab as a private resource and a key component of a business model that involved selling fixed portions of virtual real estate to universities, corporations, and high-profile investors. The price of land that only Linden Lab had the legitimate authority to create by fiat is the primary sovereign force in its virtual world. The proprietary development of Second Life's servers meant that development of technical capabilities within the virtual world were limited to Linden Lab's engineers, the >gods< of Second Life, and their feature priorities.

Although efforts to free the boundless creative potential of Second Life's proprietary metaverse from its sovereign benefactors have been ongoing in projects such as OpenSimulator, such open-source development efforts have yet to develop a critical mass around their message, as the appeal of a network that aims to interoperate with Second Life but with a more open, flexible or customizable technical architecture seems to fall flat. I think the problem is that such efforts are unable to venture far enough from the commodified entrepreneurial ethos of Second Life as a reference point of the system's design. With Second Life's construction of the producer-consumer relationship as its constraining posthuman vision, such attempts to create an interoperable open-source server have not engaged the small community satisfied with the official, corporate-controlled Second Life experience, and have been equally uninspiring to those seeking radically different models of networked computer-mediated interaction, collaboration, perception and construction from what Second Life has been able to offer.

With Second Life, Linden Lab presented an experimental vision of digital liberty with an aesthetic of free, creative labor mediated through a loosely-regulated market of virtual goods. Next, I contrast this vision of posthuman subjectivity against the more popular, gamer-oriented phenomenon of Minecraft, and the very different aesthetic of free play in creative production it enacts.

\subsection*{Minecraft: Neo-Renaissance Craftsmanship}
Minecraft is a software project more well-known among mass-market digital gamers than Second Life, though like the latter, its open-ended aesthetic also doesn't overtly prescribe a unilateral goal upon its players and therefore presents an anomaly to formal game ontologies. Minecraft was originally conceived in 2009 by an individual game developer, Markus Persson, as a side project while employed for another game company. As early beta releases of Minecraft became hugely popular, Persson quit his day job and founded a small company, Mojang, to develop Minecraft and other independent projects. As of 2013, Minecraft has become one of the best-selling computer games of all time.\footnote{
  On 15 September 2014, Microsoft announced that it would be acquiring Mojang and its Minecraft franchise for \$2.5 billion \autocite*{Microsoft-2014}.
}
In this section, I will describe how Minecraft's combination of amateur creation-oriented interaction mechanics, intentional lack of virtual economy, and open-authorship development ethos combined to create an aesthetic of neo-Renaissance craftsmanship distinct from the entrepreneurial model of Second Life.

\subsubsection*{Mine and Craft}
Drawing broad inspiration and specific game mechanic conventions from a mix of various commercial and "indie"\footnote{
  "Indie game" is a term loosely distinguishing a cultural product less dependent upon dominant production organizations and mechanisms. The term has been gaining currency within the game industry in recent years as a result of rapid transformations in game production models (such as digital distribution and crowdfunding), having displaced the term "independent game" around 2009. However, much like the "independent films" actually produced by conglomerate-owned subsidiaries, the "indie game" can also merely indicate minor distinctions along audience, distribution, funding/budget, or stylistic lines rather than indicate a complete financial or cultural independence from industry conventions. See \citeauthor{Parker2013}'s \citetitle{Parker2013} for a survey of the growing literature on indie games.
}
game influences, the core gameplay of Minecraft involves navigating an abstract, brown-haired human avatar through rustic, three-dimensional landscapes and dark underground dungeons, interacting with the environment's assortment of static blocks and dynamic creatures, and producing and collecting a limited inventory of resources and items arranged in a balanced hierarchy. The game exhibits a somewhat standard set of conventional action-fantasy/role-playing gaming mechanics: health points and combat, food and hunger levels, graded tiers of weapons, armor and durability, experience points and levels, and magical statistical modifiers in the form of potions and enchantments.

Two key game mechanics set Minecraft apart from most canonical "dungeon crawlers" of the action-fantasy genre, aptly comprising the game's title: mining and crafting. Both of these mechanics derive from the game world's striking visual aesthetic comprised of large unit cubes or "voxels," each block half the height of the player and texture-mapped with simple, low-resolution images. Every block that comprises the terrain can be destroyed or "mined" with tools linked to simple narrative fictions aiding the process (an axe can be used for breaking wood, pickaxe for stone, shovel for dirt, etc). The most valuable resources used to create advanced equipment are rare metals such as iron and diamonds, found randomly deposited throughout the underground landscape. In order to find these treasures, the player must carve out large, winding caverns through the underground landscape, search for and collect treasures, and then return to a home base to deposit loot into permanent storage, upgrade equipment, and repair and expand the base.

To mine, a structural block is broken with a single mouse button press. The resource can then be collected in the player's inventory, and this resource can be used to place a new block anywhere in the environment (adjacent to another block) with another single mouse click. This mine-collect-craft feedback loop, where mined resources are then repurposed to create functional structures ranging from simple shelters to elaborate central outposts, comprises what might be seen as the essence of Minecraft's gameplay. This feedback loop of explore, collect and create is mapped to a series of simple narrative fictions, with recognizable mechanics derived from fantasy computer game conventions such as progressing through a hierarchy of increasingly-powerful items, and discovering and collecting rare and special resources. However, in the process of engaging such simple, standard conventions, more open-ended, creative possibilities of inhabiting Minecraft's world present themselves.

As bases and cave tunnels expand, the player must creatively organize their spatial activity through increasingly creative constructions. At first, these might be simple, prescribed creations such as an enclosed shelter to keep out randomly-spawned monsters, or a second floor to maximize the use of space. As such functional creations expand in complexity, they can take on an individual character of self-expression. Through video recordings shared on Youtube and through special >creative< game servers, players have constructed and showcased enormous, elaborate dwellings built up from the basic blocks, and on wikis and forums, players share their blueprints for elaborate mechanisms of automatic resource production. On this scale, Minecraft can begin to be viewed as a creative tool with similarities to Second Life's own building-construction interface, though designed as an intuitive tool for the amateur craftsman rather than complex design software for the professional entrepreneur. Blocks become three-dimensional units of Lego-like construction, the infinitely-generated terrain becomes a spatial canvas, and the player's avatar becomes an intuitive first-person interface, where selections and modifications are performed simply by pointing, building or destroying.

\subsubsection*{Creative Mode}
Second, beyond the game mechanics, Minecraft also cultivates its ethos of the amateur craftsman through a complete absence of any functional virtual economy in its default environment, in stark contrast to Second Life and most other commercial virtual worlds. This lack is intentional, and can be attributed to two aspects: an open server model where the server application was made public so anyone could easily create and manage their own privately-run game environment, and a >creative mode< gameplay option that enables the avatar to fly quickly through the virtual space and bypass most of the restrictive, time-consuming mechanics present in the >adventure< mode such as resource requirements for creating blocks and items or limited travel speeds. Both of these aspects facilitated the removal or modification of any artificial in-game constraints, so that nothing in the Minecraft experience was bounded by a player's labor that could be exploited in a virtual economy of any significance.

In contrast to the academic enthusiasm for and interest in isolated, game-specific "play-labor" economies in online multiplayer games such as World of Warcraft, I believe there is a distinct freedom to be found in software that structurally eliminates the pernicious effects of human labor captured within artificially-regulated systems under rhetorics of play as entertainment. Within Minecraft, it's still possible for third parties to create their own managed, isolated economies (by running a customized server with the appropriate "mod" software installed), but such micro-economies are more recognizably fictional (and harder to integrate into globalized systems of ludocapitalist exploitation, as in the gold-farming operations conducted in World of Warcraft) since they are voluntarily adopted rather than unavoidably endured by all players for the sake of preserving a game's authorial integrity. In other words: rather than ethically construing intentionally bypassing resource or spatial constraints as >cheating,< Minecraft's game-playing aesthetics instead presents the player with freedom to voluntarily enter and exit those constraints at will as one of the values consistent with its system.\footnote{
  As \citeauthor{Consalvo2007-yx} describes in \citetitle{Consalvo2007-yx}, the discourse of cheating is a social-technical, value-laden negotiation of power and agency among players and industry actors in and around game-playing media, and different attitudes towards cheating can reflect various nuanced ethical relations to >gaming capital< \autocite[2]{Consalvo2007-yx}.
  }

\subsubsection*{Public Authorship}
Minecraft exhibits a novel model of cultural production I see as \emph{public authorship}, a model in which an engaged audience is not only authorized but is actively encouraged to openly interact with an authored, fictional world as a literary starting point for their own interpretations, modifications, and creations. I recognize two distinct sides of such public authorship, both of which represent a break from prior models of commercial game development: a public orientation of the author's own iterative development process, and a public orientation of the audience's own software modifications and expressions of derivative works.

Minecraft's development history begins with a moment of amateur appropriation, as a self-acknowledged >clone< of another recently-released amateur game, Infiniminer: on 13 May 2009 (weeks after Infinimer's public release on April 29), Persson posted a video to YouTube titled \citetitle{Persson-cave}: "This is a very early test of an Infiniminer clone I'm working on. It will have more resource management and materials, if I ever get around to finishing it" \autocite{Persson-cave}. Minecraft was subsequently developed in full public view, with a first public prototype released less than a week after the first video, and tech demos regularly appearing on YouTube. Starting in December 2009, Persson started making all ongoing in-development ("indev") builds of the game available to players who placed preorders. Between the two-year development process from first prototype to the release of Minecraft 1.0 in November 2011, Minecraft had already become popular enough from preorder sales to guarantee the game's success.

The unlikely amount of enthusiasm within the player community for modifying the game software was largely unanticipated by Mojang, but it has become one of the most interesting and prolific aspects of the game and is in many ways inseparable from the Mojang-authored object itself. Today, much of the ongoing creative production surrounding Minecraft is developed and supported through an enormous, informal "mod" community producing software that extends and rewrites the game's operation. The unintended nature of the mod community is demonstrated whenever Minecraft version updates are released that break the functionality of existing mods, and it is ultimately left up to the community to adapt its offerings. (Mojang has a more organized API under development.) Aside from software modifications, the Minecraft community's public orientation is also exhibited through the vast proliferation of user-generated Minecraft content on YouTube, encouraged by Mojang as a form of free promotion. In his study of \citetitle{Lastowka2012-89e}, \citeauthor{Lastowka2012-89e} identifies this community-generated content as a fundamental component of Minecraft's production logic: "Players use Minecraft's software as a locus for generating their own creative content both in the game and outside of it.…Their creations work in lieu of traditional advertising by popularizing the game with new users and adding to the game's value" \autocite[10]{Lastowka2012-89e}.

Despite this open orientation of public authorship, Minecraft is still a traditionally-authored product of today's information economy, packaged and distributed as proprietary software, with payment verified through server-side digital rights management required to play. However, beyond this conceit, Mojang marks a radical departure from mainstream practices of cultural property management through a hands-off approach to its creative property. So long as its players pay for its software, Mojang has encouraged the vibrant proliferation of code modifications, hacks and derived content, allowing its active community of players to continue to define the game's future public image and feature designs. While there are obvious options available (both technical and legal) that would more closely manage the Minecraft property to prevent or discourage derivative works or more strictly enforce unauthorized copying, Mojang has, if anything, gestured in the other direction. Persson has spoken out against excessive trademark litigation, donated money to the Electronic Frontier Foundation in support of patent reforms, and is a member of the Swedish Pirate Party. At the 2011 Game Developers Conference, he publicly encouraged players to pirate Minecraft if they couldn't afford the game: "Piracy is not theft. If you steal a car, the original is lost. If you copy a game, there are simply more of them in the world. There is no such thing as a >lost sale<" \autocite[qtd. in][]{Thier-2012}.

As one of best-selling computer games of all time, Minecraft has the luxury not to be as concerned about any "lost sale" as other moderately-successful creators struggling to make ends meet. This tension between Minecraft's amateur ethos and commercial success poses the question: does Minecraft's precarious popularity outline a reproducible, sustainable, model of anarcho-ludic aesthetic organization that captures some novel, playful elements of our posthuman, "Web 2.0" moment of digital media? Or is it merely an indie aberration, its commercialization representing a "selling out" all too familiar to subcultural production, a veiled betrayal of the amateur gamer community that provided it with design feedback, publicity, even direct inspiration (e.g., Infiniminer)? Persson personally struggled with this tension throughout the latter phase of Minecraft's development process, leaving Mojang's business operations and eventually Minecraft itself to other employees, returning to tinker with new creations. To the extent that it resisted its own commercial trappings, Minecraft can be viewed less as a digital game-product to be consumed and more a digital playground, a space of free play less recognizable as a singularly-authored game than as a flexible component of open-ended cultural possibility.

To summarize, I characterize the model of free play that Minecraft exemplarizes as an expression of an idealized model of craftsmanship derived from a Renaissance view of work. This model, sharing broad affinities with such entities as Lego building blocks, O'Reilly Media's \emph{Make} magazine, and Stewart Brand's \emph{Whole Earth Catalog}, has developed into a compelling rhetoric of play in recent decades, which I interpret as a popular response to an increasing professionalization of technical expertise in modern society. In his \citedate{Mills1951} work \citetitle{Mills1951}, \citeauthor{Mills1951} summarized this ideal model of craftsmanship as involving six major features:
\blockcquote[220]{Mills1951}{
  There is no ulterior motive in work other than the product being made and the processes of its creation.…The details of daily work are meaningful because they are not detached in the worker's mind from the product of the work.…The worker is free to control his own working action.…The craftsman is thus able to learn from his work; and to use and develop his capacities and skills in its prosecution.…There is no split of work and play, or work and culture.…The craftsman's way of livelihood determines and infuses his entire mode of living.
}
Mills laments that "none of these aspects are now relevant to modern work experience" of mid-twentieth century American white-collar workers, describing it as "an anachronism" to be upheld as an "explicit ideal" against which the conditions of modern work can be negatively contrasted \autocite[224]{Mills1951}. Within contemporary ludocapitalism, however, this very ideal of craftsmanship is being reintroduced into digital cultures through a rhetoric of play viewed not as the refusal of or freedom from work, or as begrudgingly coexisting alongside work (as in a labor/leisure divide, or craftsmanship as a "hobby"), but as an ideal of authentic creative production from which new posthuman forms of life might emerge. Minecraft was both formed by and is an exemplary expression of this cultural ideal, as evident in the game's title ("-craft") and by \citeauthor{Persson-cave}'s own public struggles to maintain an authentic, autonomous relation to his own craft following Minecraft's massive success: "Turns out, what I love doing is making games. Not hyping games or trying to sell a lot of copies. I just want to experiment and develop and think and tinker and tweak.…So that's what I'm going to do" \autocite*{Persson2013-todo}.

\section{Conclusions}
Although neither Second Life nor Minecraft is a sufficient or complete answer to Crogan's question of the nature of computer games as part of >life< in contemporary technoculture, I believe my comparison illuminates the role of the concept of game-playing in framing posthuman subjectivity. Through ensuring intellectual property protections to its >residents< and giving them tools to make and sell their own virtual goods, Linden Lab's Second Life presents an aesthetic model of the technoliberal, entrepreneurial subject that it simultaneously bred within its own corporate culture. Through its public web-based development process, sandbox-style gameplay radically open to player modifications, and its rejection of the anti-piracy rhetoric of the game industry, Mojang's Minecraft presents an aesthetic model of a neo-Renaissance communion of player-craftsmen, simultaneously playing in and playing with an openly-shared, collective culture. I have also pointed out how each model contains its own contradictions consistent with its own ideological model---Second Life in its God-like control over the legal and technical limitations of its virtual economy; Minecraft in its unabashed cloning of Infiniminer's unique visual aesthetic and game mechanics and in its enormous commercial capitalization of an amateur player community. Each software project gestures towards an aesthetic presentation of posthuman subjectivity encouraging a degree of individual autonomy that many other games, and many other ways of talking about games, lack entirely.

The critical ludology attentive to game-playing aesthetics that I elaborated in this chapter leads us away from the violence of essentialist classifications of game objects: not toward a black hole of realist speculation where everything is the play of the world, nor toward an idealism where play is a narrowly-specific form of essentially human experience, but toward a thinking of free play as presenting the ethical and political goals we hope to achieve through the precarious balance of necessity and freedom in the game-playing forms we create, observe and inhabit.

As I began to touch upon through the examples in this chapter, one of the central facets of our posthuman environment that structures our notion of what a digital game object is and how it should be produced and consumed within our society, a facet undergoing regular transformation and facing mounting political debate, is the liberal concept of property and its contemporary expansion into a global regime of intellectual property. My next chapter will examine the legal discourse and social tensions surrounding this constitutive value of the ludocapitalist paradigm more closely through the ironic socio-legal corporate history of Tetris, a popular game now counted among the world's most valuable and vigorously protected intellectual property brands.