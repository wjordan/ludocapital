% This is a template for Ph.D. dissertations in the UCI format.
% 
% All fonts, including those for sub- and superscripts, must be 10
% points or larger.  Recommended sizes are 14-point for chapter
% headings, 12-point for the main body of text and figure/table
% titles, and 10-point for footnotes, sub- and super-scripts, and text
% in figures and tables.
%
% Notes: Add short title to figures, sections, via square brackets,
% e.g. \section[short]{long}.
%
\documentclass[12pt,fleqn]{ucithesis}

% A few common packages
\usepackage{amsmath}
\usepackage{amsthm}
\usepackage{array}
\usepackage{graphicx}
\usepackage{natbib}
\usepackage{relsize}

% Some other useful packages
\usepackage{caption}
\usepackage{subcaption}  % \begin{subfigure}...\end{subfigure} within figure
\usepackage{multirow}
\usepackage{tabularx}

% plainpages=false fixes the "duplicate ignored" error with page counters
% Set pdfborder to 0 0 0 to disable colored borders around PDF hyperlinks
\usepackage[plainpages=false,pdfborder={0 0 0}]{hyperref}

% Uncomment the following two lines to use the algorithm package,
% which provides an algorithm environment similar to figure and table
% ("\begin{algorithm}...\end{algorithm}"). A list of algorithms will
% automatically be added in the preliminary pages. Note that you
% probably want a package for the actual code to go with this (e.g.,
% algorithmic).
%\usepackage{algorithm}
%\renewcommand{\listalgorithmname}{\protect\centering\protect\Large LIST OF ALGORITHMS}

% Uncomment the following line to enable Unicode support. This will allow you
% to enter non-ASCII characters (such as accented characters) directly without
% having to use LaTeX's awkward escape syntax (e.g., \'{e})
% NOTE: You may have to install the ucs.sty package for this to work. See:
% http://www.unruh.de/DniQ/latex/unicode/
%\usepackage[utf8x]{inputenc}

% Uncomment the following to avoid "widowing", where page breaks cause
% single lines of paragraphs to float onto the next page (this is not
% a UCI requirement but more of an aesthetic choice).
%\widowpenalty=10000
%\clubpenalty=10000

% Modify or extend these at will.
\newtheorem{theorem}{\textsc{Theorem}}[chapter]
\newtheorem{definition}{\textsc{Definition}}[chapter]
\newtheorem{example}{\textsc{Example}}[chapter]

% Macros for title, author, abstract, etc.
\thesistitle{Title of the Thesis}

\degreename{Doctor of Philosophy}

% Use the wording given in the official list of degrees awarded by UCI:
% http://www.rgs.uci.edu/grad/academic/degrees_offered.htm
\degreefield{Computer Science}

% Your name as it appears on official UCI records.
\authorname{Your name}

% Use the full name of each committee member.
\committeechair{Professor A}
\othercommitteemembers
{
  Professor B\\
  Professor C
}

\degreeyear{2012}

\copyrightdeclaration
{
  {\copyright} {\Degreeyear} \Authorname
}

% If you have previously published parts of your manuscript, you must list the
% copyright holders; see Section 3.2 of the UCI Thesis and Dissertation Manual.
% Otherwise, this section may be omitted.
% \prepublishedcopyrightdeclaration
% {
% 	Chapter 4 {\copyright} 2003 Springer-Verlag \\
% 	Portion of Chapter 5 {\copyright} 1999 John Wiley \& Sons, Inc. \\
% 	All other materials {\copyright} {\Degreeyear} \Authorname
% }

% The dedication page is optional.
\dedications
{
  (Optional dedication page)
  
  To ...
}

\acknowledgments
{
  I would like to thank...
  
  (You must acknowledge grants and other funding assistance. 
  
  You may also acknowledge the contributions of professors and
  friends.
  
  You also need to acknowledge any publishers of your previous
  work who have given you permission to incorporate that work
  into your dissertation. See Section 3.2 of the UCI Thesis and
  Dissertation Manual.)
}


% Some custom commands for your list of publications and software.
\newcommand{\mypubentry}[3]{
  \begin{tabular*}{1\textwidth}{@{\extracolsep{\fill}}p{4.5in}r}
    \textbf{#1} & \textbf{#2} \\ 
    \multicolumn{2}{@{\extracolsep{\fill}}p{.95\textwidth}}{#3}\vspace{6pt} \\
  \end{tabular*}
}
\newcommand{\mysoftentry}[3]{
  \begin{tabular*}{1\textwidth}{@{\extracolsep{\fill}}lr}
    \textbf{#1} & \url{#2} \\
    \multicolumn{2}{@{\extracolsep{\fill}}p{.95\textwidth}}
    {\emph{#3}}\vspace{-6pt} \\
  \end{tabular*}
}

% Include, at minimum, a listing of your degrees and educational
% achievements with dates and the school where the degrees were
% earned. This should include the degree currently being
% attained. Other than that it's mostly up to you what to include here
% and how to format it, below is just an example.
\curriculumvitae
{

\textbf{EDUCATION}
  
  \begin{tabular*}{1\textwidth}{@{\extracolsep{\fill}}lr}
    \textbf{Doctor of Philosophy in Computer Science} & \textbf{2012} \\
    \vspace{6pt}
    University name & \emph{City, State} \\
    \textbf{Bachelor of Science in Computational Sciences} & \textbf{2007} \\
    \vspace{6pt}
    Another university name & \emph{City, State} \\
  \end{tabular*}

\vspace{12pt}
\textbf{RESEARCH EXPERIENCE}

  \begin{tabular*}{1\textwidth}{@{\extracolsep{\fill}}lr}
    \textbf{Graduate Research Assistant} & \textbf{2007--2012} \\
    \vspace{6pt}
    University of California, Irvine & \emph{Irvine, California} \\
  \end{tabular*}

\vspace{12pt}
\textbf{TEACHING EXPERIENCE}

  \begin{tabular*}{1\textwidth}{@{\extracolsep{\fill}}lr}
    \textbf{Teaching Assistant} & \textbf{2009--2010} \\
    \vspace{6pt}
    University name & \emph{City, State} \\
  \end{tabular*}

\pagebreak

\textbf{REFEREED JOURNAL PUBLICATIONS}

  \mypubentry{Ground-breaking article}{2012}{Journal name}

\vspace{12pt}
\textbf{REFEREED CONFERENCE PUBLICATIONS}

  \mypubentry{Awesome paper}{Jun 2011}{Conference name}
  \mypubentry{Another awesome paper}{Aug 2012}{Conference name}

\vspace{12pt}
\textbf{SOFTWARE}

  \mysoftentry{Magical tool}{http://your.url.here/}
  {C++ algorithm that solves TSP in polynomial time.}

}

% The abstract should not be over 350 words, although that's
% supposedly somewhat of a soft constraint.
\thesisabstract
{
  The abstract of your contribution goes here.
}


%%% Local Variables: ***
%%% mode: latex ***
%%% TeX-master: "thesis.tex" ***
%%% End: ***


% Add PDF document info fields
\hypersetup{
	pdftitle={\Thesistitle},
	pdfauthor={\Authorname},
	pdfsubject={\Degreefield},
}

% Uncomment the following to have numbered subsubsections (by default
% numbering goes only to subsections).
%\setcounter{secnumdepth}{4}


% Set this to only select a subset of the includes directives below.
% Very handy to speed up compilation if you're working on a certain
% part of your thesis. It conserves page numbers, references, etc.
% even for non-included files.
%\includeonly{chapter1}

\begin{document}

% Preliminary pages are always loaded (TOC, CV, etc.)
\preliminarypages

% Include the different components of your thesis, in separate files.
% Using \include allows you to set \includeonly above.
\chapter{Introduction}

This is an example using the \LaTeX{} template for UCI theses and
dissertation documents \cite{uci-thesis-latex}. Figure
\ref{fig:sourcecode} is just for illustration purposes, as is Table
\ref{tab:coordinates}.

\begin{figure}
\begin{verbatim}
#include <iostream>
int main(int argc, char** argv) {
  std::cout << "Hello World." << std::endl;
  return 0;
}
\end{verbatim}
  \caption{Example source code.}
  \label{fig:sourcecode}
\end{figure}

\section{Background}

Lorem ipsum dolor sit amet, consectetur adipisicing elit, sed do
eiusmod tempor incididunt ut labore et dolore magna aliqua. Ut enim ad
minim veniam, quis nostrud exercitation ullamco laboris nisi ut
aliquip ex ea commodo consequat. Duis aute irure dolor in
reprehenderit in voluptate velit esse cillum dolore eu fugiat nulla
pariatur. Excepteur sint occaecat cupidatat non proident, sunt in
culpa qui officia deserunt mollit anim id est laborum.

\begin{table}
  \centering
  \begin{tabular}{|rr|r|}
    \hline
    $x$ & $y$ & $z$ \\
    \hline
    14 & 12 & -2 \\
    0 & 33 & -25 \\
    -3 & 11 & 22 \\
    4 & 4 & 6 \\
    \hline
  \end{tabular}
  \caption{Example coordinates.}
  \label{tab:coordinates}
\end{table}

Lorem ipsum dolor sit amet, consectetur adipisicing elit, sed do
eiusmod tempor incididunt ut labore et dolore magna aliqua. Ut enim ad
minim veniam, quis nostrud exercitation ullamco laboris nisi ut
aliquip ex ea commodo consequat. Duis aute irure dolor in
reprehenderit in voluptate velit esse cillum dolore eu fugiat nulla
pariatur. Excepteur sint occaecat cupidatat non proident, sunt in
culpa qui officia deserunt mollit anim id est laborum.


%%% Local Variables: ***
%%% mode: latex ***
%%% TeX-master: "thesis.tex" ***
%%% End: ***

%\chapter{Specters of Play: Hauntology of Tetris®}
\label{tetris}
\epigram{
  Tetris enslaved my brain. At night, geometric shapes fell in the darkness as I lay on loaned tatami floor space. Days, I sat on a lavender suede sofa and played Tetris furiously. During rare jaunts from the house, I visually fit cars and trees and people together. Dubiously hunting a job and a house, I was still there two months later, still jobless, still playing. \autocite{Goldsmith1994}
}
\epigram{
  A spectre is both visible and invisible, both phenomenal and nonphenomenal: a trace that marks the present with its absence in advance. The spectral logic is de facto a deconstructive logic. It is in the element of haunting that deconstruction finds the place most hospitable to it, at the heart of the living present, in the quickest heartbeat of the philosophical. Like the work of mourning, in a sense, which produces spectrality, and like \emph{all} work produces spectrality. \autocite[117]{Derrida2002}
}

\section*{Introduction}
In this chapter, I present the commodity form of the videogame as a key aspect of the political-economic structure of ludocapitalism through a case study of Tetris.

First, I start from the existing academic canonization of Tetris as a paradigmatic object among videogame formalists, and develop an allegorical reading emphasizing the game's visual expression of spatial-temporal mechanics, which produces a human cognition of algorithmic space comprising a cognitive mapping functioning as a symbol of the computerization of everyday life.

Second, I extend this structural reading of Tetris's construction of game-space through a social-historical interpretation of the game's construction of commodity-space within digital capitalism, as a vigorously marketed and litigated billion-dollar brand. Here, I look at the habit-forming psychological character of Tetris play as a Benjaminian aura exceeding the ontological boundaries of the commodity form. I argue that the growth of the software industry and its expansion of intellectual property law across time and space have coerced basic components of intersubjective experience, such as the abstract operational rules of popular videogames like Tetris, into exemplary post-Fordist commodity forms.

Third, through a close reading of a recent lawsuit The Tetris Company won against an independently-produced variation of Tetris, I argue that the novel legal arguments justifying protection of the videogame commodity idealize "fun" while suppressing any latent "function" in the game object, revealing a tension between market-enabling conditions of property ownership and communicative conditions of intersubjectivity. This tension is predominantly \emph{spectral} in the sense that Derrida examines in his reading of Marx's analysis of the commodity form.

Finally, I conclude by discussing substantive critiques of the modern intellectual property doctrine's liberal-humanist foundations from the field of critical legal studies, along with Derrida's thematic orientation of \emph{hauntology} as deconstructive logic of ontological form, to form an ethical-political position that a posthumanist game criticism could adopt toward the ludocapitalist commodity form that Tetris represents.

\subsection*{The Ideal-Type Commodity Form}
A central component of the liberal enlightenment subject is the figure of the possessive, individual property holder, of which Locke's labor theory of property is the canonical expression and Marx's social theory of value the classical critique. Locke held that the "great and chief end" of government is "the preservation of property," justifying an individual's right to an exclusive claim upon land and other natural resources according to his proportional application of human labor to the state of nature. In \notecite{Marx-surplusvalue}\citetitle{Marx-surplusvalue}, Marx recognized Locke's "classical expression of bourgeois society's ideas of right" as the "basis for all the ideas of the whole of subsequent English political economy" \autocite*[XX-1293a;~pt.~1,~addendum,~sec.~4]{Marx-surplusvalue}. This comprised a primary target of Marx's critical work, which instead conceived the modern institution of property as reflecting particular, historically-specific social relations integral to the profit cycle of industrial capitalism.

Marx's twofold theory of value extended beyond the specific capitalist social structure of his age, providing the basis for future transitional models of capitalism's global structure. I use one such model to situate the political-economic structure of ludocapitalism alongside the era known as "post-Fordism," distinguished both from the previous Fordist era as well as from the classical form of industrial capitalism. From the mid-1970s, a cluster of Marxist-influenced social theorists known as the Regulation School theorized the emergence of a distinct "regime of accumulation" around the early twentieth century, marked by an increased emphasis on the marketing and distribution of mass-produced household goods to an expanding consumer society. They identified this distinct regime as "Fordism," exemplified by the Ford Motor Company's paradigmatic stewardship of the American automobile industry through a combination of tactics including a large-scale, vertically-integrated corporate structure, Taylorist scientific management and assembly-line factories, and mass media marketing to an increasingly homogenous consumer middle class.\footnote{
  \citeauthor{Aglietta1980}'s \citetitle{Aglietta1980} is viewed as the foundational text of the Regulation School; see also \citeauthor{Boyer1990}'s \citetitle{Boyer1990} for an overview of the field's subsequent literature. \citeauthor{Gramsci1930}'s notes from the 1930s on \citetitle{Gramsci1930} provided the basis for recognizing Fordism as a paradigm for the era.
}

While the Regulationists claimed that Fordism peaked around 1960 and has since entered a period of decline or crisis, debate has continued through recent decades around how best to distinguish the present "neo-" or "post-Fordist" regime of accumulation from its predecessor. In his study of consumer culture, Martyn Lee presents a Regulationist-influenced method of distinguishing the key characteristics of Fordist and post-Fordist periods through a study of the particular qualities and characteristics of the commodity form within each era. Starting from a Marx-influenced premise that the "real significance of the commodity" is that it tends to "reflect the whole social organization of capitalism at any historical and geographical point in its development" \autocite*[112]{Lee2003}, Lee develops Weber's methodological concept of the "ideal type" to propose that for a specific regime of accumulation, a distinctive "aggregation of individual commodities which appear to share certain recognisable material and non-material characteristics" can be identified as the "ideal-type commodity form of the regime of accumulation" \autocite[119]{Lee2003}. Lee's method of analysis implies a strong correlation between a commodity form and a regime of accumulation, such that "the particular structural order of the spatial and temporal dimensions of a regime of accumulation,…is accordingly objectified in the major commodities of the period" \autocite[124]{Lee2003}.

Lee describes "standardised housing and the car" \autocite*[129]{Lee2003} as two paradigmatic commodities to emerge in the Fordist regime, combining to construct an ideal-type spatial and temporal experience of everyday consumer life characterized by "fixity, permanence, and sheer physical presence" \autocite[130]{Lee2003}. While the modern house provided a "stable space for mass consumption" as a standardized container for the vast array of consumer goods entering the household, the car complemented and mobilized the household by linking it to the "vast social network of consumption and welfare" including "schools, shopping centers, and the new leisure complexes" \autocite[129-30]{Lee2003} all within driving distance of the home. As exemplary of Fordism's ideal-type commodity form, the car and standardized housing thus stabilized and regulated the spatio-temporal dimensions and rhythms of everyday middle-class life, subjecting it to a relatively homogeneous assortment of household goods and car-accessible social landmarks distributed throughout the modern American landscape.

For post-Fordism, Lee refers to "high-tech commodities," "information" commodities and "cultural services and events" \autocite[128]{Lee2003} as examples of an ideal-type commodity form characterized by the "fluidisation of consumption" and the "freeing up of the previously static and relatively fixed spatial and temporal dimensions of social life" \autocite[133]{Lee2003}. However, \citeauthor{Kline2003}, adopting Lee's model but finding his account of contemporary commodity forms inconclusive, instead propose that the videogame best represents the ideal-type commodity form for the present era. Videogames, they argue, "embody the new forces of production, consumption, and communication with which capital is once again attempting to force itself beyond its own limits to commodity life with new scope and intensity" \autocite*[76]{Kline2003}.

In this chapter, I affirm and extend \citeauthor{Kline2003}'s provocation by presenting Tetris as a paradigmatic example of the videogame as ideal-type commodity form in post-Fordist capitalism. The central position of Tetris within the academic videogame canon as representing something abstractly essential to the videogame's experiential form has been well established within the subfield of game studies known as \emph{ludology}, which I will cover in the next section.

\section{Tetris as Symbolic Game Object}

\subsection*{A Ludological Touchstone}
To link my analysis of Tetris in particular to the videogame as ideal-type commodity form for post-Fordism in general, I depend upon a "touchstone" metaphor in two interrelated senses. First, I think of Tetris as a mechanism of comparative perception that \citeauthor{Arnold1909-qu} once employed "for detecting the presence or absence of high poetic quality": the literary critic would "have always in one's mind lines and expressions of the great masters, and,…apply them as a touchstone to other poetry" \autocite*[10]{Arnold1909-qu}. This sense of Tetris as a great work at the center of the videogame canon has been justified through the game's widespread popularity and commercial success, as well as the "prototypical," genre-defining status it had asserted within the subsequent casual game and matching-tile sub-categories of commercial software entertainment.\footnote{
  See \citeauthor{Juul2009}'s \citetitle{Juul2009}: "It could also be argued that the 1985 Tetris was the first casual game" \autocite[27]{Juul2009}; Tetris is at the top of a "family tree of the history of matching tile games.…Tetris was an extremely successful game that spawned a number of imitators" \autocite[86--7]{Juul2009}; "For a long period of time, matching tile games were considered derivatives of Tetris, which was given the status of a prototype game" \autocite[98]{Juul2009}.
  }
Second, I consider Tetris in the same way \citeauthor{Lee2003} describes the correlation between the contemporary commodity form and the material-cultural reproduction of everyday life: "the commodity presented itself as a vital touchstone, at once being the focus of national economic prosperity as well as providing an important material and symbolic resource by which ordinary people could, both materially and culturally, reproduce their life" \autocite[x]{Lee2003}. In this way, I frame my analysis of Tetris as the construction of a ludological touchstone: by interpreting the "meaning" of Tetris and investing this particular instance of the videogame's commodity form with symbolic significance, I aim to present a cultural symbol of ludocapitalism itself.

I begin my analysis of the meaning of Tetris at the moment that Tetris was first canonized within academia as a central symbolic object of contention, the "ludology vs. narratology" schism that occurred in the infant years of computer game studies from 1997 to 2004. In 1997, \citeauthor{Murray1997}'s \citetitle{Murray1997} subordinated games to stories within a broader neo-Aristotelian aesthetics of the digital storytelling medium, advancing the view that "A game is a kind of abstract storytelling.…Every game, electronic or otherwise, can be experienced as a symbolic drama" \autocite[142]{Murray1997}. Reiterating the anthropological view of games as "ritual actions allowing us to symbolically enact the patterns that give meaning to our lives" \autocite[143]{Murray1997}, Murray offered a unique reading of Tetris in advancing the position that beyond symbolic drama, "Games can also be read as texts that offer interpretations of experience":
\blockquote{
  Even a game with no verbal content, like Tetris,…has clear dramatic content. In Tetris,…success means just being able to keep up with the flow. This game is a perfect enactment of the overtasked lives of Americans in the 1990s---of the constant bombardment of tasks that demand our attention and that we must somehow fit into our overcrowded schedules and clear off our desks in order to make room for the next onslaught.…Tetris allows us to symbolically experience agency over our lives. It is a kind of rain dance for the postmodern psyche, meant to allow us to enact control over things outside our power. \autocite[143--4]{Murray1997}\footnote{
    In a later essay, Murray reiterated this position, again including Tetris as an example: "Games are always stories, even abstract games such as checkers or Tetris, which are about winning or losing, casting the player as the opponent-battling or environment-battling hero" \autocite*[2]{Murray2004}.
  }}
I find this reading of Tetris, as an allegory for the psychology of everyday life within American post-Fordist capitalism, to be intriguing but conflicted. Murray's aesthetics, structured by a neo-Aristotelian view of interactive drama and strongly influenced by the humanist subjectivity of the Victorian novel, aim at a creative maximization of narrative expression epitomized by a "half-hacker, half-bard" figure \autocite*[9]{Murray1997}. Such an ideal figure tends to elusively conceal the significance of abstract games like Tetris, preferring at times to praise a more nostalgic, traditional liberal-humanist notion of storytelling that "helps us understand the world and what it means to be human" \autocite[26]{Murray1997}. Framing the "potential for compelling computer stories" on a progressive "road from puzzle gaming to an expressive narrative art" \autocite[53]{Murray1997}, Murray's progressive desire is to "move the established game industry far past the lucrative shoot-'em-ups and puzzle mazes" on the market today, imploring "more sophisticated developers" to "make stories that have more dramatic resonance and human import to them, stories that,…mean something" \autocite[54]{Murray1997}.

As a play of "irregularly shaped objects,…relentless activity, misfits and tight couplings, order and chaos, crowding and clearing," Tetris doesn't amount to much of significance for Murray: "while we experience the game as being about skill acquisition, we are drawn to it by the implicit expressive content of the dance" \autocite[144]{Murray1997}. The content of this >rain dance< is merely a primitive precursor to more authentic expression that Murray claims lies elsewhere, in a more fully-developed videogame medium: "The violence and simplistic story structure of computer skill games are therefore a good place to examine the possibilities for building upon the intrinsic symbolic content of gaming to make more expressive narrative forms" \autocite[144-5]{Murray1997}.

Murray's reading of Tetris drew strong criticism from a handful of formalist gamer-theorists who were just beginning to organize ludology as a scholarly field of game studies formed around an essentialist study of games "as games," set apart from the perceived colonizing attempts of other fields to >claim< computer games for their own \autocite{Aarseth2001-y1}. In the years following Murray's interpretation, a handful of formalist responses worked to reclaim Tetris as a paradigmatic abstract game, an event horizon of the story-game divide representing the essence of what the field of ludology could claim as fundamentally a "game" and nothing else:

\blockquote{
  A game like Myst has the quality of being representable in a traditional medium like the newspaper.…But they're usually the worst games. A game like Tetris, on the other hand,…looks dull in the paper, it has no story. And imagine a narrative as abstract as Tetris. This would be out of the question. Stories need human or anthropomorphic characters. Games don't. \autocite{Juul1998}
  }
  
\begin{quoting}
    |A| session of Tetris can hardly be recognized as narrative, mostly because of a lack of characters. However, some narratology authors claim that even a cooking recipe is narrative, so maybe a session of Tetris could be it, too. \autocite{Frasca1999}
\end{quoting}
\blockquote{
  Instead of studying the actual game [of Tetris] Murray tries to interpret its supposed content, or better yet, project her favourite content on it; consequently we don't learn anything of the features that make Tetris a game. The explanation for this interpretative violence seems to be equally horrid: the determination to find or forge a story at any cost, as games can't be games because if they were, they apparently couldn't be studied at all. \autocite{Eskelinen2001}
  }
  
\begin{quoting}
  Games are games, a rich and extremely diverse family of practices, but fundamentally, they are games.…Games are not >textual,<…games are not intertextual either; games are self-contained.…In Tetris, I do not stop to ponder what those bricks are really supposed to be made of. \autocite[47-8]{Aarseth2004}
\end{quoting}
Murray offered her own critical summary of myopic formalism in the developing field, while acknowledging the adoption of Tetris as its paradigm: "The paradigmatic game for [ludology] is Tetris. According to the formalist view Tetris can only be understood as [an] abstract pattern of counters, rules, and player action, and the pattern means nothing beyond itself, and every game can be understood as if it were equally abstract" \autocite*{Murray2005}.

My particular interest in this debate is in its emphasis on Tetris, and particularly the "meaning" of Tetris, as a recurring symbol of the conflict. While the game fundamentalists upheld Tetris as their paradigmatic game comprised of formal, rule-based qualities and little else, they left the significance of such a paradigm as largely self-evident based on the game's commercial success and the booming commercial videogame industry. I extend their analysis in this direction more explicitly, reframing the fundamental distinctions between game and narrative that the ludologists were grappling with not as a pure, universal philosophical abstraction but rather as a novel commodity form reflecting certain general characteristics of our present capitalist age.

Amid the gallons of ink spilled over this ludological schism in the following decade, I find a productive middle-ground in Ian Bogost's attempt to bridge the divide of this "classic conflict between narration and simulation" \autocite*[99]{Bogost2006-ec}. His approach reframes the "meaning" of Tetris in terms of a more general cultural model:
\blockquote{
  The problem with the Murray/Eskelinen approach to abstract puzzle games is that one wants the game to function only narratively, the other wants it to function only formally. Neither is exactly right without the other. The problem seems to be this: the "meaning" of an abstract puzzle game lies in a gap between its mechanics and its dynamics, rather than in one or the other. \autocite*{Bogost2009-gj}
}
Bogost describes this "gap" between mechanics and dynamics as a "simulation", defined as "the gap between the rule-based representation of a source system and a user's subjectivity" \autocite*[107]{Bogost2006-ec}. According to this view, Murray's allegorical reading of Tetris is "entirely reasonable," as it "accounts for a biased, subjective response in the player," and "takes into account a larger system that the game represents in smaller part, the function of the unit-operational rules of the simulation, and a subjective response to the simulation that embeds an ideology" \autocite[101]{Bogost2006-ec}.\footnote{
  See also \citeauthor{Begy2010-zz}'s \citetitle{Begy2010-zz}, which evaluates and expands upon Bogost's simulation-gap analysis of Murray's Tetris reading through a terminology of "experiential metaphors" inspired by Lakoff and Johnson's work.
  }
In this model it is a "larger system" of culture, a system at a higher level than the formal analysis of rules in isolation, that is the real product of the interpretation of "meaning" in the game object.

Nonetheless, despite being "reasonable," Murray's interpretation of Tetris as "a perfect enactment of the overtasked lives of Americans in the 1990s" is a generic reading that could equally apply to many sorts of cultural artifacts, and does not attend to any specific material qualities that mark Tetris as distinct from other forms of time-sensitive stress-inducing activity in human history. An interpretation of Tetris paradigmatically representing the videogame as a commodity form with unique material qualities requires both a closer mechanical and a broader social-historical reading of the videogame object's allegorical construction of digital space, which I will develop further in the following sections.

\subsection*{Digital gamespace}
Like other videogames, Tetris allegorizes the computer's graphical display as a virtual space through an analogy to real, physical space. The experience of space produced by Tetris, however, is much different from experiences of physical space in other paradigmatic games such as Pong or Spacewar. The mechanisms of movement in Tetris are limited to discrete keyboard inputs rather than, for example, the continuous, analog potentiometer knobs used to control Pong paddles. The playing field of Tetris is comprised of a two-dimensional grid of 20 units by 10 units, much more limited in number than Spacewar's vast 1024 by 1024 coordinate landscape. To use a distinction that \citeauthor{DeleuzeGuattariATP} applied in their comparative reading of "The >smooth< space of Go, as against the >striated< space of chess" \autocite[353]{DeleuzeGuattariATP}, Tetris allegorizes a contained, Cartesian gamespace that is distinctly >striated< as opposed to other >smooth< approximations of physical worlds.

The computer-graphical representation of striated space as a two-dimensional, cellular grid has its predecessor in von Neumann's two-dimensional models of cellular automata, which he used to theorize connections between McCulloch and Pitts's neural network model and Turing's one-dimensional paper-tape based universal computing machine, producing constructive simulations of self-reproducing automata. However, it was the promotion of more recreational models that brought this concept to the attention of the broader public, through Martin Gardner's popular "Mathematical Games" column that ran in Scientific American from 1956 to 1986. A collection of combinatorial puzzles and other abstract, mathematical recreations, Gardner's column was responsible for introducing the notion of a striated gamespace to the public consciousness through its 1970 article, \citetitle{Gardner-1970}, as well as an ongoing popularization of \citetitle{Gardner-1965} puzzles that served as the primary inspiration for Tetris.

In his \citedate{Golomb1965} book, \citetitle{Golomb1965}, Solomon \citeauthor{Golomb1965} noted the particular significance of such puzzles to an increasingly combinatorially-dependent scientific landscape: "the ever-increasing importance of digital computers in modern technology has revived a widespread interest in combinatorial analysis, a subject that also has had important applications in such modern scientific fields as circuit design, coded communications, traffic control, crystallography, and probability theory" \autocite*[44]{Golomb1965}. The discrete, combinatorial spatial puzzles involving tetrominoes and other polyominoes indeed make it possible to engage in complex mathematical reasoning and problem-solving through analogies to the physical sensory experience of space, but the relation also works in the other direction as well: as with Conway's Game of Life, the same allegorical relation makes such spatial puzzles serve as the analogical, physical experience of our mental representation of digital gamespace, allowing us to experience the mathematical relations expressed through such games as an intuitive, emotional response.

Once the metaphor of tetrominoes as physical objects occupying combinatorial space is established, a whole set of additional material metaphors follows. The uniformly-sized tetromino objects and their all-or-nothing, occupation of grid-space resembles a dense, solid, manufactured material such as brick, and the involuntary, downward movement of the active piece evokes the pull of an algorithmic gravity on a solid object. The vivid space evokes within the player's imagination the feeling of real, dense, space-occupying objects flung through the air, of the downward force of gravity, and of a pile of blocks accumulating at the bottom of a rigid well.\footnote{
  Begy rightly speculates that "if Tetris were inverted such that the blocks rose from the bottom towards the top of the screen it would create a very different experiential gestalt" \autocite*[84]{Begy2010-zz}.
}
In his defense of the formal analysis of abstract games, Aarseth claims: "In Tetris, I do not stop to ponder what those bricks are really supposed to be made of" \autocite*[48]{Aarseth2004}. However, such a claim reveals that even in such formalist analysis, a certain interpretation of the text must have already taken place: the discrete spatial units comprising the Tetris pieces as solid "bricks" already belies a chain of intertextual interaction metaphors extending the game's algorithm beyond a self-contained digital formalism into a human representation of physical, spatial experience. This combined set of interaction metaphors for manipulating solid shapes draws strong parallels to what \citeauthor{Shneiderman82} famously described in \citeyear{Shneiderman82} as "direct manipulation" in relation to the now-ubiquitous desktop computing metaphor: "representation of the object of interest, rapid incremental reversible actions and physical action instead of complex syntax" \autocite[237]{Shneiderman82}. Direct manipulation is a perfectly appropriate description of the combined spatial and interaction metaphors used in Tetris.

Despite such physical metaphors of direct manipulation, I find some distinct differences between the digital space of Tetris and the intuitive experience of continuous, physical space. Even though the unit "bricks" and tetromino "pieces" are intuitively physical, we can only move the active block exactly one unit space to the right or to the left, not anywhere in between. In this way, I view Tetris as the ideal, abstract "allegorithm" of the computer game itself, relating sign to algorithmic function.\footnote{
  \citeauthor{Galloway2006-pp} and \citeauthor{Wark2007-ya} have both advanced this concept of allegorithm in critical media theory: "To play the game means to play the code of the game. To win means to know the system. And thus to interpret a game means to interpret its algorithm (to discover its parallel >allegorithm<). So today there is a twin transformation: from the modern cinema to the contemporary video game, but also from traditional allegory to what I am calling horizontal or >control< allegory" \autocite[90-1]{Galloway2006-pp}; "Allegory is about the relation of sign to sign; allegorithm is about the relation of sign to number" \autocite[par.~041]{Wark2007-ya}; "The gamer discovers a relationship between appearances and algorithm in the game, which is a double of the relation between appearances and a putative algorithm in gamespace---that's allegorithm. But there is always a gap between the intuitively knowable algorithm of the game and the passing, uneven, unfair semblance of an algorithm in the everyday life of gamespace---this is the form that allegory now takes" \autocite[par.~031]{Wark2007-ya}.
  }
Tetris provides even the most uninitiated game player an intuitive, aesthetic, emotional access to the visual, digital logic of the computing machine from which it is produced. To play the game is to interpret and internalize its algorithm, not expressed as a mathematical formula or verbal narrative, but intuitively linked to basic human sensorimotor functions. The aesthetic function of Tetris thus familiarizes the mechanical, unfeeling, digital, virtual space of the computing machine through the basic spatial metaphors with which we are all already familiar.

I see its logic of digital space reflected back upon Tetris in an uncanny doubling effect: whereas the computer works to apply a grid of intelligibility onto the aspects of society its programs attempt to model, simulate, archive or control, Tetris applies a grid of intelligibility onto the machine itself---translating its opaque procedural mechanism and discrete logic into a form more directly discernible by human sensorimotor processing. Although the effect produces a sort of false consciousness (the player only controls the machine to the limited extent that it was designed to be played), it is precisely this illusion of agency in relation to an imagined computer that makes the simulation meaningful, even pleasurable, as Ted \citeauthor{Friedman2005} notes: "the pleasures of a simulation game come from inhabiting an unfamiliar, alien mental state: from learning to think like a computer" \autocite*[136]{Friedman2005}.

This psychological experience produced by the simulation game, of inhabiting an alien, algorithmic mental state, can be characterized as an emotion akin to the feeling of the sublime described in Kant's \citetitle{Kant1987-coj}. In Kant's model, an experience of the sublime is produced when the mind's faculty of imagination is overwhelmed by its inability to comprehend the multiplicity of an object in a single intuition. Bogost offers a reflection on Tetris in comparison to two iPhone puzzle games along these lines, in a reading of the experience of abstract puzzle games through an application of Kant's mathematical sublime:

\blockquote{
  One might find a similar mathematical sublimity at work in Tetris, after all. Each block alters the topology of the playfield, the player must alter that topology to continue the game, and chance dictates what pieces might be available to consummate the geometrical promises made earlier.
  
  But Drop7 and Orbital differ from Tetris in an important way: they are turn-based, not continuous. The player must always intervene to make the next move, offering an opportunity to reflect on the enormity of the task, a requirement of sublimity.…In Tetris, the method of play disrupts access to the sublime. \autocite*{Bogost2009-gj}
  }
While I find Bogost's connection between abstract puzzle games and the mathematical sublime illuminating, I fail to see why the temporal dynamic of Tetris "disrupts access to the sublime" found in other abstract turn-based games, as there is in fact no requirement that the subject pause to "reflect" on the task in Kant's model.\footnote{
  Within Kant's system, aesthetic judgments of the sublime are one category of >reflective judgment,< which is perhaps misinterpreted here to imply meditative practice. Rather, Kant deploys this term in a specific sense, distinguished against >determinate judgment,< to mean the construction of a universal concept from a particular.
}
In the case of Tetris, such an experience of the sublime would be enhanced, not disrupted, by the temporal dynamics of its continuous loop of algorithmic interaction.

My reading of Tetris as pure simulation, studied in terms of its spatio-temporal operations alone and their cognitive effect on an idealized human player-subject, largely followed Bogost's unit-operational, simulation-gap model of game criticism. While I believe that such a reading adequately reconciles the interpretive schism surrounding Tetris within the game studies field, the meaning it produces still limits itself to an internal account of the particular spatial experience that uniquely defines the game as a self-contained system, rather than an external account of the game's historically-specific function within particular models of culture and society.\footnote{
  As one example of a Tetris study limited to such a narrow form of analysis, \citeauthor{Post2009}'s \citetitle{Post2009} offers an orthodox structuralist approach: >To bridge the divide between ludology and narratology, that is, to reconcile narrativity and interactivity, we need paradoxically where Barthes in 1966 called for, a >structural analysis of narrative<< \autocite[36]{Post2009}.
  }
In particular, describing the relation between the videogame and its subject as one of reception and response to a "simulation" brackets off any traces of its outward commodity form, idealizing it as a perfect cultural transaction between a preconstituted game designer-producer and player-consumer. In the following sections, I will expand this limited, idealist model of a game's operation into a social-historical account of how Tetris in particular, and videogames in general, function as a commodity form within post-Fordist capitalism, an account that will also provide my reading with its political relevance.

\subsection*{Simulation as commodity}
So far, my description of Tetris-space has remained within the confines of understanding the game as a simulation, and describing experience in terms of an allegorical relation to an idealized space of algorithmic computation. An attempt at discerning the meaning of Tetris as a canonical, paradigmatic videogame object merely in terms of its simulation of digital gamespace still refuses an adequate description of the relation of digital gamespace to the global system as a whole, and risks normatively reifying the broader social relations of the computer game complex which legitimate, define and protect the digital object as a given entity in our social world. As I extend my interpretation of Tetris into a social history, I intend to reconcile the allegorical function of the game's internal construction of space and time with its function as the complex, ubiquitous, legally-protected, billion-dollar brand burned into the psychological unconscious of its world of player-consumer subjects.

As I transition into social-historical analysis, I imagine the particular way in which a simulation relates to its broader world-system, with its varied political resonances, as a form of cognitive mapping. \citeauthor{Friedman2005} notes in his study of videogames as simulations, "Simulations may be the best opportunity to create what Fredric Jameson calls >an aesthetic of cognitive mapping: a pedagogical political culture which seeks to endow the individual subject with some new heightened sense of its place in the global system<" \autocite[141]{Friedman2005}. Jameson poses this same problematic in the context of "museum space," reading Hans Haacke's conceptual art as a project that expands into a politically-oriented cognitive mapping through a critique of its own institutional structure:
\blockquote{
  |T|he work of Hans Haacke, for example,…redirects the deconstruction of perceptual categories specifically onto the framing institutions themselves.…|I|n Haacke it is not merely with museum space that we come to rest, but rather the museum itself, as an institution, opens up into its network of trustees, their affiliations with multinational corporations, and finally the global system of late capitalism proper, such that what used to be the limited and Kantian project of a restricted conceptual art expands into the very ambition of cognitive mapping itself.…|T|he spatializing tendencies,…become overt and inescapable in the uneasy gestalt alternation between a "work of art" that abolishes itself to disclose the museum structure which contains it and one that expands its authority to include not merely that institutional structure but the institutional totality in which it is itself subsumed. \autocite*[158]{Jameson1990-zi}
}
Jameson reads Haacke's work as a cognitive mapping in this political sense because of the self-referential attention to its own commodity status, which expands the observer's perception outward towards the institutional totality of global capitalism within which it is embedded. In a similar vein, I will follow the expansion of Tetris into the spaces of global commodified creativity characteristic of contemporary post-Fordist forms of digital entertainment. However, unlike Haacke's work which is designed to transparently disclose the "institutional totality in which it is itself subsumed," I see Tetris more like Andy Warhol's Campbell's soup cans, in that it does not autonomously function as a critical or political statement but rather it has the potential to function as a symptom of its environment through a critical reading. A critical-historical, spectral reading of the commodity-space of Tetris, one that goes against the grain of the object to recover the repressed, smoothed-over controversies haunting its hard-fought security and legitimacy, can trace the contours of this symptomatic object, producing a cognitive mapping that could recover its political potential. It is with this aim that I begin my account of the history of Tetris in the next section.

\section{Tetris as Cultural Commodity}
\subsection*{Two Images of Tetris}
To begin my social history of Tetris as a commodity form, I will first juxtapose two luminous injections of Tetris's cognitive mapping into the real space of everyday life to demonstrate the variety of expressions and distributions that are possible in order to emphasize the contingency of any particular approach.

First, a curious lamp made its debut at Toy Fair 2012 in New York, appearing in online retail outlets in time for the year's holiday shopping season \autocite{Tetris2012-toy}. The lamp is comprised of seven brightly-colored, detachable tetromino\footnote{
  Back-derived from Golomb's \emph{polyomino}, \emph{tetromino} is the mathematical term for the geometric shape comprised of four orthogonally-connected cells.
}
shapes, each with a flat, fixed depth. When each piece is stacked on top of the base they light up, powered through electrical current sent through each piece's metallic edges. Anyone who has ever played Tetris on a computer, videogame console or mobile phone will immediately find the lamp's blocky, colorful visual design distinctly familiar, as the seven one-sided tetrominoes have secured a place in the optical unconscious of our global, computerized society.

What makes this lamp particularly notable is not merely its clever shape or functional design, but the logos and fanfare identifying it as an "Official Tetris\texttrademark\space Product" \autocite{Tetris2012-lamp}. The lamp was produced by Paladone, a UK-based gift supplier, as part of a partnership arranged by LicensingWorks!, a merchandise licensing company which represents The Tetris Company, the exclusive licensee of Tetris Holding, LLC, the corporate entity in charge of all intellectual property rights related to Tetris. Through this chain of sublicensing agreements, along with a similarly complex lineage of copyright and trademark assignments dating back to Russian computer engineer Alexey Pajitnov's original game design prototypes around 1985, an unbroken chain of public authorization extends to this official Tetris-brand lamp on display during the 2012 shopping season. This authorization carries with it an implicit injunction: without such a licensing arrangement to make authorized use of Tetris Holding's guarded intellectual property, it would be impossible to market a similar product in public for long before receiving a cease-and-desist letter from Tetris Holding with charges of trademark and trade dress infringement, and threats of long and costly litigation unless the product is removed and all copies destroyed immediately.

That same year, another Tetris-inspired work lit up the space of everyday life with a very different aesthetic. In a spectacular form of public Tetris craft, several successful installations of fully functional Tetris games on the face of tall buildings have been created using computer-controlled lights emanating from a grid of windows as unit squares of the game's display. The most recent implementation, installed on the face of MIT's Green Building in April 2012, has been called "The Holy Grail of hacks" by MIT students \autocite[8--9]{Pourian2012}. Aligning this aesthetic performance with a political position, the students who produced the hack subsequently published "MITris," their Java source code for "a game similar to Tetris using the windows of the MIT Green Building as pixels," under an open-source license on a public code hosting service, with their stated goal "to inspire the world at large to create interesting games, visualizations, or just about anything" \autocite{mitrisdev2012}.

In stark contrast to the authorized marketing of Tetris-branded commercial products, a wealth of Tetris craft persists, despite the controlled spread of officially licensed merchandise. Due to its spatial simplicity and formal distinctiveness, Tetris-inspired visual expressions can be easily crafted from all sorts of different media, including chalk outlines, waffles, ice cubes, birthday cakes, nail polish, even marching bands, to name just a few. Indeed, to many, the production of public, unbranded Tetris craft represents the epitome of a hacker culture that embraces an open spirit of free play, averse (or at least indifferent) to proprietary, commercial products.

A similar Green Building Tetris hack was originally planned in 1995 by a group of MIT students including Vadim Gerasimov, who as a teenager worked with Pajitnov to co-author the very first published version of Tetris.\footnote{
  Although Gerasimov's plans for a Green Building Tetris hack were not realized at the time, he built a playable software prototype in Java which he eventually published free of charge on his personal website \autocite*{Gerasimov-green}.
}
Although the first Tetris prototype was initially completed by Pajitnov sometime around mid-1985 on an Elektronika-60 computer,\footnote{ The date of Tetris's "creation" has undergone some interesting historical revision. Prior to 2009, The Tetris Company had always consistently promoted Tetris as "created in 1985 by Russian scientist Alexey Pajitnov" \autocite{Tetris-dot-com}. However, on June 2, 2009 (the first day of the annual Electronic Entertainment Expo), as part of a large and quite successful "25th anniversary" media campaign run by a PR firm, The Tetris Company announced for the first time that Tetris's "birthday" was now officially June 6, 1984 \autocite{TetrisPR}.} Gerasimov rewrote the game for the IBM-PC platform himself within "a few days," according to his own recollection of events \autocite*{Gerasimov-tetris}. It was copies of this latter version that the team distributed for free on floppy diskettes to friends, leading to the game's popularity throughout Moscow and Eastern Europe on the path to its subsequent global fame. Gerasimov wrote that he "worked on Tetris just for fun" and received nothing in return, periodically distributing updates to the program over the next couple of years. Several years after Tetris was first released, Pajitnov reportedly "stopped by [Gerasimov's] home and asked [him] to urgently sign a paper >to get lots of money for us from game companies.<" Since then, Gerasimov's name "disappeared from all newly released versions of Tetris and all official documents" and has been stricken from official Tetris history. When a US copyright registration was eventually filed for the IBM-PC game Gerasimov himself wrote, Pajitnov was listed as the sole legal author.

In juxtaposing these two contemporary Tetris-inspired works, I suggest that the significance of Tetris in contemporary digital culture is represented in the space between these two images represented by Pajitnov and Gerasimov: between the corporate-sponsored "Official Tetris\texttrademark{} product" lamp endorsed by Pajitnov's Tetris Holding, and the Tetris Green Building hack first conceived by Gerasimov and finally realized in the public MITris performance and open-source code release. Generalizing this relation, we can say that the contemporary form of new media objects is held together by a tension: here, a tension between the Tetris game's abstract, formal simplicity and reproducibility, and the various intellectual property protections from which its legitimate heirs fashion a commercial game object and product brand.

More generally, this tension within the videogame object is distinctively spectral, in the sense that Derrida applies to this term in \citetitle{Derrida1994-ii}. The videogame object is situated between enduring material elements, present in time and place that are accorded an author and provided legitimacy and protection, and spectral elements that circulate and propagate freely in a common, shared, public orality. Such spectral elements are never present in the object itself as perceived by the subject; rather, the object is haunted by their ghostly presence. What these specters represent, if they can be said to manifest anything as such, are the repressed histories and controversies of the property they haunt, of the ghostly alterity and play of ideas that speaks along the boundaries of a proprietary object that could have been or might yet become its other.

I will return to discuss Derrida's hauntology toward the end of this chapter (indeed, the theme of the specter haunts my entire analysis of the commodity form). In the next section, I develop Walter Benjamin's concept of \emph{aura} in relation to the habit-forming and image-persisting qualities of Tetris play, as a discourse that exceeds the ontological boundaries of the commodity form. As such, I read it as a preliminary or alternative hauntology.

\subsection*{Videogame's Aura}
The structural conditions of Tetris's present-day commodity form bear some relation to, but also mark a significant transformation from, those conditions concerning Benjamin's analysis of the mass-produced work of art in the Fordist era. In this section, I read Benjamin's analysis of aura as a critical examination of that which exceeds the work of art's ontological construction of "presence" through physical property relations. Aura relates to the subject's unconscious perception and interiorization of the object as something not captured by the strictly economic analysis of exchange, yet can nonetheless produce a powerful effect. As his analysis of aura took place in an era when the economics of intellectual property were not yet codified into law and as thoroughly constitutive of global economic production as they are today, I believe the kinds of phenomenological tensions Benjamin identified in his concept are even more relevant to the ideal-type post-Fordist commodity form than that of his own era.

In his celebrated, enigmatic essay, "The Work of Art in the Age of Mechanical Reproduction," \citeauthor{Benjamin1969-ed} laments the decline through mechanical reproduction techniques of a work's "aura," viewed as the "cult value of the work of art" \autocite[243n5]{Benjamin1969-ed}, or a "unique phenomenon of a distance however close it may be" \autocite[222]{Benjamin1969-ed}. Although an art object's material form may be perfectly copied through mechanical techniques, the unique presence of the original and its subsequent authenticity may not:
\blockquote{
  Even the most perfect reproduction of a work of art is lacking in one element: its presence in time and space, its unique existence at the place where it happens to be. This unique existence of the work of art determined the history to which it was subject throughout the time of its existence. This includes the changes which it may suffered in physical condition over the years as well as the various changes of ownership.…The presence of the original is the prerequisite to the concept of authenticity. \autocite[220]{Benjamin1969-ed}\footnote{
  The footnote to this passage is also relevant: "Of course, the history of a work of art encompasses more than this. The history of the >Mona Lisa,< for instance, encompasses the kind and number of its copies made in the 17th, 18th, and 19th centuries" \autocite[243n1]{Benjamin1969-ed}.
  }
}

I interpret this analysis as a prototypical analysis of socio-economic distortions within the commodity form that would only later become fully institutionalized within the post-Fordist concept of intellectual property. In his age, visual media designed for mass reproduction such as film and photography were just beginning to establish a mass public, prompting drastic transformations within explosive new organizations of culture and politics. This new form of mechanically-reproduced media "detaches the object from the domain of tradition" \autocite[221]{Benjamin1969-ed}, and "emancipates the work of art from its parasitical dependence on ritual,…the total function of art is reversed. Instead of being based on ritual, it begins to be based on another practice---politics" \autocite[224]{Benjamin1969-ed}. This process of reversal produced the "growing proletarianization of modern man and the increasing formation of masses" \autocite[241]{Benjamin1969-ed}, resulting in the dialectical crisis of Communism's politicization of art and repudiation of property relations opposed to Fascism's aestheticization of politics and upholding of property.

Within bourgeois cultural production more generally, Benjamin also noted that this process resulted in a shift in from "cult value" or "use value" to "exhibition value" or exchange value, a shift from the aesthetic appreciation of the work of art to the political economy of the mass-produced cultural commodity. In this analysis, the valuation of a uniquely-produced work of art, originally based in a ritualistic appreciation of the singular object, is displaced through industrial production into secular, economic terms as an exchange value normalized and regulated by a strengthening juridical regime of property rights. In American film production, for example, this regime is capitalized upon by the monumental rise of the Hollywood studio conglomerates and the personality cult of celebrity. While for the traditional work of art "the quality of its presence is always depreciated" by mechanical reproduction degrading its authenticity \autocite[221]{Benjamin1969-ed}, the quality of the cultural product designed for mass reproducibility, by contrast, is instead secured through the emerging legal regime of intellectual property rights necessary to secure its mass audience.

However, I find Benjamin's multivalent concept of aura to be not merely reducible to an economic analysis of prototypical intellectual property relations within the cultural commodity, as it also serves as a social critique of its ontology. As a medium of perception, aura exceeds the valuation of a cultural commodity within pure exchange, and it is this critical function that I wish to highlight in the concept. In his essay "On Some Motifs in Baudelaire", Benjamin remarks that "to perceive the aura of an object we look at means to invest it with the ability to look at us in return", citing Proust's notion of "memoire involontaire" alongside Valery's "characterization of perception in dreams as aural": "To say, >here I see such and such an object< does not establish an equation between me and the object.…In dreams, however, there is an equation. The things I see, see me just as much as I see them" \autocite[188-9]{Benjamin1969-ed}. Aura here gains the valence of an "optical unconscious," as a set of associations that distinctly bleed beyond the memoire volontaire that consciously perceives the object in its recognized form. In this sense, aura mediates an uncontrollable, unconscious desire that haunts the presence of the commodity-object.\footnote{
  For more on how Benjamin's concept of aura shifts between multiple valences, see \cite{Hansen2008-ej}.
}

Relating Benjamin's perceptual metaphors to Tetris, I find a heightened significance in the auratic phenomenon known as the "Tetris effect" \autocite{Goldsmith1994}. After intense play, players often report visions of falling tetromino-shaped blocks persisting in their dreams. In controlled experiments, even amnesiacs recall such dreams and retain their Tetris-playing ability in subsequent plays, despite no conscious memory of their previous experiences \autocite{Stickgold2000-mv}. In this quite literal sense, Tetris produces an aura through the interactive experience of play, an aura which inserts itself into the optical unconscious that orients future perceptions.

\citeauthor{Benjamin1969-ed} compares, but distinguishes between, the aura of an object of perception from the "trace" of a utilitarian object: "If we designate as aura the associations which, at home in the \emph{memoire involontaire}, tend to cluster around the object of a perception, then its analogue in the case of a utilitarian object is the experience which has left traces of the practiced hand" \autocite[186]{Benjamin1969-ed}; and the emerging techniques of mechanical reproduction "represent important achievements of a society in which practice is in decline" \autocite[186]{Benjamin1969-ed}. The medium of the videogame (for which Tetris is our paradigmatic example) can be characterized differently from the audiovisual work of passive perception, as a hybrid, utilitarian object of perception both observed and enacted. In contrast to the pure audiovisual work, reproduction of the videogame object encourages the formation of a "practiced hand," not merely its decline.

In sum, Benjamin's concept of aura evokes not only the alluring mystique of an inaccessible, authentic art object always removed from the commodity form in presence, but also an uncontrollable force that both invades the subject's optical unconscious and leaves traces of a practiced hand, depositing itself into memoire involontaire and haunting the object's presence as an element exceeding its commodity form.\footnote{
  This dual function of aura relates to \citeauthor{Huizinga1971}'s quasi-mystical account of the >magic circle< that not only bounds a game in distinct time and space but also imbues the play experience with meaning and significance, a play-function that bleeds beyond the game's boundaries to comprise a meaningful cultural component and civilizing function of society.
}
As Pajitnov describes his creation, "Tetris is some song which you sing and sing inside yourself and can't stop" \autocite[qtd. in][]{Goldsmith1994}.

The distinctive character of Benjamin's aura is a simultaneous interiorization of the perceptual experience of a commodity, and commodification of interiority, features which have only grown importance in contemporary conditions of post-Fordism. Over the last half-century, a series of international trade agreements have expanded and standardized the notion of intellectual property from a loosely-connected set of industry-specific production arrangements to a unified, global regime of intellectual property protections. The 1980s was a crucial decade of this proliferation which saw the economic restructuring and eventual collapse of the Soviet Union, one of the largest remaining ideological barriers to the global capitalist expansion of property relations. This decade also saw the most dramatic expansion of property rights into the realm of the idea, prompted in large part by the formation of the computer software and videogame industry. Benjamin's critical analysis of aura therefore acts as a premonition of the forms of branding and intellectual property protection of mass-produced digital artifacts that define the post-Fordist commodity experience. In the next section, I relate how the specific business-legal history of Tetris was shaped by specific post-Fordist conditions of commodification including branding, licensing and litigation.

\subsection*{Branding Tetris}
Pop-culture journalist David \citeauthor{Sheff1993} has documented the early history of Tetris in Game Over, where we read that the game was largely commodified through its eventual corporate integration and consolidation into an elaborate, and distinctly American, narrative of intellectual property license-oriented technological entrepreneurialism. Sometime around 1985--6, a director of the Computer Center where Pajitnov worked send a copy of the first published IBM-PC prototype of Tetris developed by Gerasimov to the director of SZKI, the Institute of Computer Science in Budapest. There, the game was >discovered< by the west in 1986, when the director of a UK software company, Andromeda, saw the IBM-PC version running on a terminal during a visit to SZKI and, realizing the game's untapped commercial potential, began to negotiate licenses with Soviet government administrators in an attempt to commercialize the game in the West. From 1987 to 1989, a complex web of localized and platform-specific rights to the Tetris intellectual property were negotiated through Elorg (Elektronorgtechnica, the Soviet Ministry of Software and Hardware Export that centrally managed digital technology transfer for the Soviet Union), and tenaciously fought over by several computer game companies. This web of what Sheff called a "tangled family tree" \autocite[310]{Sheff1993} of licenses and sublicenses covered every computing platform from arcade machines to personal computers to Nintendo's Game Boy handheld console, in every developed geographical market from Japan to the United Kingdom to the United States.

This business-legal history largely determined the resulting structure of the game's commodification. Early on, the product's geographical focus moved from the Soviet Union to the United States, which provided the growing Tetris enterprise with a much more commercial-friendly legal protection and firmly established system of intellectual property protection than the Soviet regime could offer. June 2, 1989 was the "first use in commerce" date on Elorg's official US Patent and Trademark Office registration for the \textsc{TETRIS} mark, which marked the use of the Tetris brand as a key legal playing piece within an ongoing turf war between Nintendo and Atari Games in the American courts. On June 21, a U.S. District judge awarded Nintendo a preliminary injunction against its rival, resolving the trademark sublicensing dispute over two competing Tetris versions both released on the Nintendo Entertainment System earlier that year, which ordered several hundred thousand of Atari's game cartridges to be recalled from stores and destroyed.

Through the early 1990s, Tetris was canonized as one of the most popular successes of the videogame medium, thanks in large part to its licensing deals with Nintendo that secured it a privileged spot within its proprietary computing platforms and videogame marketing channels that saturated the generation's digital entertainment. In the years since its initial conception, Tetris had evolved from a rudimentary prototype virally propagating through Moscow into an industry dominating, internationally-licensed property---a bundle of copyright, trademark and trade dress property claims, framed by growing consumer recognition and appeal bolstered by marketing budgets of companies such as Nintendo that stood to profit from commodifying the entertaining game in the form of licensed, packaged software for their entertainment systems.

The next phase of Tetris commodification began in 1995, when Pajitnov's ten-year licensing agreement with (post-Soviet privatized) Elorg expired, and Pajitnov teamed up with marketing-savvy business partner Henk Rogers to form The Tetris Company, a venture entirely dedicated to consolidating and exploiting the Tetris brand in future products. With a vision to become "the Coca-Cola of computer games" \autocite{Edge1999}, The Tetris Company set out to standardize the Tetris brand around a new iconic logo, a definitive design document (originally titled "What is Tetris?", subsequently called the "Tetris Guideline") standardizing details such as game mechanics and color styles for all subsequent Tetris-licensed products, and delivering legal threats and lawsuits to any games similar in name or design that encroached upon its share of the thriving casual puzzle-game market. In a 1999 magazine interview, Rogers explained the company's rationale behind its tenacious litigation tactics: "Intellectual property rights allow companies to invest real money in the development of new product. Look at any country where they don't have intellectual property rights. You don't find any interesting intellectual property being created there" \autocite{Edge1999}.\footnote{
  The interviewer goes on to note the irony: "Well, except for Tetris in the USSR" \autocite{Edge1999}.
}

From 1995 to the present day, The Tetris Company has flexibly adapted its core property to every new consumer technology platform---from the early mobile phone platforms tightly regulated by telecommunications companies such as AT\&T to the more competitive smartphone marketplaces administered by Google and Apple, from downloadable game software sold on Internet websites to always-connected Facebook games peddling virtual goods through in-game currencies. It engages in a precarious balancing act, paradigmatic of contemporary cultural commodities, of an open but authoritatively-controlled message: tacitly embracing and promoting emergent, Tetris-inspired crafts through public-engaged channels such as its corporate Facebook page, while simultaneously suppressing any Tetris-derived innovations successful or popular enough to compete in any of the brand's active markets. The resulting spatial distribution is an amateur, ephemeral, actively-forgotten culture alive with popular but ultimately unproductive activity around the edges of discourse, with an enduring, stable, branded, professional product at the center.

Haacke, the artist Jameson cited in his discussion of cognitive mapping, recognized the constitutive role of intellectual property in shaping cultural production and incorporated critical commentary on these issues in his creative work. \citeauthor{Coombe1998-yv} mentions examples of the legal threats Haacke had received Mobil Oil and Philip Morris Co. regarding his work, "in one instance incorporating the threatening letters he received, together with an explanation of the legal defense of fair use, directly into his art" \autocite[74]{Coombe1998-yv}. It is with a similar aim of critical commentary on the transformation of the commodity form in contemporary life that I will next discuss how recent transformations in intellectual property law have shaped our understanding of the cybernetic commodity as object of property, and how the arguments put forth in a recent Tetris intellectual property lawsuit provides the most legible mapping of this commodity form.

\section{The Idealization of Property in the Digital Age}
\subsection*{Cybernetic Property}
In his translation of Benjamin's media analysis of mechanical reproduction into "the age of cybernetic systems," \citeauthor{Nichols1988} describes a qualitative shift from physical, mechanical reproduction to logical, cybernetic simulation as the defining factor of the age:
\blockcquote[33]{Nichols1988}{
  The chip replaces the copy. Just as the mechanical reproduction of copies revealed the power of industrial capitalism to reorganize and reassemble the world around us, rendering it as commodity art, the automated intelligence of chips reveals the power of postindustrial capitalism to simulate and replace the world around us, rendering not only its exterior realm but also its interior ones of consciousness, intelligence, thought and intersubjectivity as commodity experience. The chip is pure surface, pure simulation of thought.
}
Nichols presents the critical question as that of control: "The ideal simulation would be a perfect replica, now controlled by whomever controls the algorithms of simulation.…Who designs and controls these greater systems and for what purpose becomes a question of central importance" \autocite[34--5]{Nichols1988}.

The juridical-legal dimensions of property are a determining component of any "commodity experience" in contemporary cultural production. They are also its most opaque, socially-technically complex elements, where the boundaries of proprietary versus common conceptual elements are both widely misunderstood and feared by both producers and consumers, and are also undergoing rapid transformation and continuous public debate. An opaque, subjective doctrine of legal interpretation separates borrowing from copying, inspiration from appropriation, creation from clone. The struggle for creative control over commodity experience that takes place within this juridical-political sphere is "clearly a central area of conflict and one in which some of the basic changes in our conception of the human/computer, reality/simulation metaphors get fought out" \autocite[38]{Nichols1988}, and the legal reconceptualizations of copyright and patent law that enabled the commodification of algorithmic simulation to take place was merely "the process by which a dominant ideology seeks to preserve itself in the face of historical change" \autocite[38]{Nichols1988}. The cell and the computer, linked through the cybernetic metaphor, were both officially commodified in 1980 through the extension of property rights to products of genetic engineering (\emph{Diamond v. Chakrabarty}) and software engineering (in the Computer Software Copyright Act of 1980), intentionally promoting the development of protected products and services based on these new technologies.\footnote{
  In \citeyear{apple-franklin}, an appellate court in \citetitle{apple-franklin} wrote: "We believe that the 1980 amendments reflect Congress' receptivity to new technology and its desire to encourage, through the copyright laws, continued imagination and creativity in computer programming" \autocite*[1253--4]{apple-franklin}.
}

It is the juridical commodification of the videogame medium, however, that represents the normalization of cybernetic simulations as creative works of art. Nichols therefore looks to the precedent-setting 1982 copyright infringement case of \emph{Atari v. North American Phillips}, where the working out of the legal doctrine in copyright law that distinguishes between unprotected general idea versus concrete, protected expression in comparing the video game Pac-Man against K.C. Munchkin "lends insight into the degree of difference between mechanical reproduction and cybernetic systems perceived by the United States judicial system" \autocite[41--2]{Nichols1988}.\footnote{
  This case is also briefly discussed in \citeauthor{Siva2001}'s \citetitle{Siva2001} \autocite[168--70]{Siva2001}.
}
In its final analysis, the Seventh Circuit Court based its decision through analogy to an essentially visual notion of "aesthetic appeal":
\blockcquote[qtd. in][42]{Nichols1988}{
  Video-games, unlike an artist's painting or even other audio visual works, appeal to an audience that is fairly undiscriminating insofar as their concern about more subtle differences in artistic expression. The main attraction of a game such as Pac-Man lies in the stimulation provided by the intensity of the competition. A person who is entranced by the play of the game, >would be disposed to overlook< many of the minor differences in detail and >regard their aesthetic appeal as the same.<
}

The quotations in this judgment refer to \emph{Peter Pan Fabrics v. Martin Weiner Corp.}, a 1960 copyright infringement case involving onamental designs printed upon cloth. In the \emph{Pac-Man} case, however, the observer has become a player ("a person who is entranced by the play of the game"), and the visual aesthetic has become an experiential relation ("stimulation provided by the intensity of the competition"). As Nichols comments, this is a distinct transformation, marking a paradigmatic moment in the commodification of cybernetic systems: "The fetishization of the image as object of desire transforms into a fetishization of a process as object of desire. This throws as much emphasis on the mental state of the participant as on the exact visual qualities of the representation" \autocite[43]{Nichols1988}.

The 1982 judgment on the \emph{Pac-Man} case provided enough legal legitimation for the videogame industry's cultural output to be commodified in the form of copyrighted intellectual property. The basis and consequences of the transformation of consumer culture that largely shaped the practice of game design is, even thirty years later, still quite ambiguous and largely untested. If the creative expression of procedure was  The ambiguity in the court system is partly explained by the extremely high legal costs required to follow any intellectual property dispute through to a final judgment, combined with the risk-averse business models of any company large enough to bear the costs.

\subsection*{Tetris v. Xio}
In 2012, a lawsuit was concluded between Tetris Holding, LLC and Xio Interactive, Inc., in which the judge decided in favor of Tetris Holding on counts of copyright and trade dress infringement \autocite{tetris-xio}.\footnote{
  For less critical law review articles summarizing this case, see \autocites{Lampros13}{Casillas13}.
}
Xio was a small start-up business formed by two recent college graduates selling an iPhone variant of Tetris named \emph{Mino} on Apple's App Store marketplace. Tetris Holding argued that Mino infringed the copyright in its Tetris game by copying fourteen distinct, expressive elements, of which the following seven were all present in the original IBM-PC prototype:
\begin{quoting}
  \begin{itemize}
    \item[1.] {
      Seven Tetrimino\footnote{
        "Tetrimino\texttrademark" is a common-law trademark that The Tetris Company uses to refer to its tetromino-shaped playing pieces.
      } playing pieces made up of four equally-sized square [sic] joined at their sides;…}
\item[3.]{ The bright, distinct colors used for each of the Tetrimino pieces;}
\item[4.]{ A tall, rectangular playfield (or matrix), 10 blocks wide and 20 blocks tall;}
\item[5.]{ The appearance of Tetriminos moving from the top of the playfield to its bottom;}
\item[6.]{ The way the Tetrimino pieces appear to move and rotate in the playfield;}
\item[7.]{ The small display near the playfield that shows the next playing piece to appear in the playfield;…}
\item[11.]{ When a horizontal line fills across the playfield with blocks, the line disappears, and the remaining pieces appear to consolidate downward.\footnote{\interfootnotelinepenalty=1000
  Each of the remaining seven expressive elements first appeared in published versions of Tetris not copyrighted by The Tetris Company (as noted below), and would likely have been filtered from the comparison, had Xio known and included these historical details in its case:
    \begin{itemize}
      \item[2.] { The visual delineation of individual blocks that comprise each Tetrimino piece and the display of their borders [Tetris (PC), Spectrum Holobyte, 1987]; }
      \item[8.] { The particular starting orientation of the Tetriminos, both at the top of the screen and as shown in the >next piece< display [Tetris (ZX Spectrum), Andromeda, 1987];}
      \item[9.] { The display of a >shadow< piece beneath the Tetriminos as they fall [Tetris: The Grand Master (Arcade), Arika, 1998];}
      \item[10.] { The color change when the Tetriminos enter lock-down mode [Tetris (NES), Tengen, 1989];}
      \item[12.] { The appearance of individual blocks automatically filling in the playfield from the bottom to the top when the game is over [Tetris (PC), Spectrum Holobyte, 1987];}
      \item[13.] { The display of >garbage lines< with at least one missing block in random order [Tetris (Arcade), Tengen, 1988]; and}
      \item[14.] { The screen layout in multiplayer versions with the player's matrix appearing most prominently on the screen and the opponents' matrixes appearing smaller than the player's matrix and to the side of the player's matrix [TetriNET (PC), 1997].}
    \end{itemize}
  I will leave these elements aside for the remainder of my analysis.
} \autocite[397--8]{tetris-xio}}
\end{itemize}
\end{quoting}
Tetris Holding also argued that Mino infringed its trade dress consisting of "the brightly-colored Tetriminos, which are formed by four equally-sized, delineated blocks, and the long vertical rectangle playfield, which is higher than wide" \autocite[415]{tetris-xio}.

Xio's central argument in its defense pointed to the long-standing doctrine that copyright does not protect rules, game mechanics and functional elements "is a basic tenet of intellectual property law" \autocite[6]{xio-summary}, as summarized in an factsheet titled "Copyright Registration of Games" circulated by the U.S. Copyright Office, which states:
\blockquote{
  Copyright does not protect the idea for a game, its name or title, or the method or methods for playing it. Nor does copyright protect any idea, system, method, device, or trademark material involved in developing, merchandising, or playing a game. Once a game has been made public, nothing in the copyright law prevents others from developing another game based on similar principles. Copyright protects only the particular manner of an author's expression in literary, artistic, or musical form. \autocite[qtd. in][6-7]{xio-summary}
  }

Xio argued that all of the "expressive elements" Tetris Holding accused it of infringing were actually "functional---either because they are a limitation and/or affordance of the game or because they otherwise play a functional role in the game" \autocite[42]{xio-summary}. Any requirement to modify or remove any of those elements would have prevented them from developing their own game based on similar principles following the conventional, familiar rules of Tetris. Changing these elements too much would effectively change the work into an altogether different, unfamiliar, or harder to play game. If Xio were to alter the dimensions of Mino's playfield or the shape of its playing pieces, for instance, it would result in much different gameplay, limiting its appeal to those players who were already familiar with the standard gameplay that Tetris had already established in its large player population. It would be as disrupting to conventional play as a computer chess game that was not allowed to display an 8x8 square, or the knight piece's L-shaped movement.

The counter-argument presented by Tetris Holding (and affirmed by the District Court judge) was that none of the elements should be considered functional rules at all, because a game could still be designed with the same >function< (that is, it could still "function perfectly well" as a game, an essentially useless, non-purposive object of entertainment) with an unlimited number of different gameplay design possibilities: "Tetris is an entertaining videogame, not a >useful article< that must look a certain way. Tetris' expression serves no utilitarian purpose, and is not the >rules< of the game" \autocite[2]{tetris-reply}.\footnote{
  Tetris Holding repeats this argument in several variations: "Tetris is a fanciful entertainment product that does not have to look the way it does" \autocite[8]{tetris-reply}; "Tetris is not a utilitarian product that has to perform a particular function---rather, it is a fanciful puzzle game created for entertainment and which could have had any myriad of designs" \autocite[11]{tetris-reply}.
}
The crux of this argument is the claim that, in the judge's own words, "Tetris Holding's design choices were essentially >arbitrary flourishes< and were in no way related to the reason the game works or functions" \autocite[416]{tetris-xio}.

Xio's legal defense was largely unprepared to counter this bold line of argumentation---no previously decided cases had ever questioned the premise that a particular game employs functional or utilitarian elements in the context of copyright or patent law. It was largely assumed, much to the chagrin of the relatively small board and card game industry throughout the 20th century, that the abstract game mechanics necessary for a game to be played were necessarily a part of its uncopyrightable system, as opposed to any ornamental decorations which would be appreciated and legally protected in terms of their respective linear media. However, in \emph{Tetris v. Xio} the procedural system as a form of creative expression had become such a commonplace that the assumption that an "entertaining videogame" has no utilitarian "function" in the context of copyright law was accepted without hesitation.

\subsection*{The Fun(ction) of Game Mechanics}
Reading this initial controversy around the function of game mechanics more generally, I identify a tension upon which the constitution of the videogame commodity is founded: the game's extrinsic function in the commodity-space is in tension with the intrinsic aesthetic or entertainment value within its game-space. In an ideal model of the videogame as object of pure commercial leisure, one in line with the judge's assumptions in this case, the game-commodity has no practical function beyond providing fleeting entertainment. The object is transacted as a commodity on the open market, passively and happily consumed by the user without any lasting effect other than the empty experience of "fun," and the process would repeat itself with every subsequent play transaction. As pure recreation, the game object is completely emptied of any recognizable "function," and thus every aspect of its design would be protected by copyright as pure creative expression.

However, as suggested by the experience of aura as exceeding the commodity form discussed earlier, this fiction of the non-functional game object is never securely fashioned. Tension emerges between the fiction of Tetris as "just a game", as an entirely arbitrary design, with fanciful flourishes of expression that could have been equally produced otherwise with the same absence of effect, and the productive impulse to characterize the game as a recognizably useful device in its specific materiality. Within the cultural-scientific realm, Tetris is often characterized as something more significant and productive: it is touted for its educational value as a digital training apparatus and brain-booster (in scientific studies funded by the Tetris licensor),\footnote{
  See \autocites{Haier1992-vh}{Haier2009-wr}. This latter study was funded by, and Haier was employed as a paid consultant for, Blue Planet Software, the company holding exclusive rights to Tetris.
}
and praised for its therapeutic value as a neuro-psychological relief for trauma flashbacks\footnote{ See \autocites{Holmes2009-ir}.} and lazy-eye,\footnote{ See \autocite{Li2013-cq}.} and even the mathematical formalization of its algorithm has drawn some scholarly interest.\footnote{ See \autocite{Demaine2003}.}

The legal narrative that Tetris Holding uses to protect its product integrates two orthogonal dimensions of intellectual property law, trademark/trade dress and copyright, that work together to form its object as a hybrid brand-product commodity. Although these two dimensions are comprised of distinct legal doctrines and thus require distinct methods of analysis, in this case both arguments similarly boil down to this tension between fun and function, or between arbitrary signifier/expression versus useful signified/content. First, the combination of the \emph{Tetris} trademark and its trade dress consisting of "brightly-colored" tetromino pieces designate not only a distinctive source of a specific brand of game, but the generic, unprotectible name and behavior of the game itself. Second, the algorithm determining the shape, movement and behavior of the game pieces is not just an arbitrary, copyrightable audiovisual expression of an underlying system, but constitute the formal, unprotectible rules of the game-play itself. In order to secure legal property protection for Tetris as a commodity, the judge chose to indicate the first side of this tension and suppress the second.

\subsection*{Trademark and Trade Dress}
The commercial identity of Tetris depends on signifying a paradoxical combination of both brand and product, denoting simultaneously the singular source and its generic object of entertainment. As codified in the Lanham Act in the United States and similarly enforced through international trade agreements throughout most of the first world, trademark law is primarily designed to protect only the words and marks that designate a specific brand, not a generic product. If trademark law granted a company exclusive control over a widely-used name for a product, this would unfairly restrict the ability of competitors to freely use the name to identify their objects in public.

Although the \emph{Tetris v. Xio} case did not concern any rights over the word-mark \textsc{TETRIS}, the word still plays a crucial role in the struggle over the object it designates that warrants some analysis.\footnote{
  The threat of Tetris Holding's trademark claims were still palpable in this case, as Xio's tetromino-stacking game was originally code-named "TetraNet" until they changed it to "Mino" in order to reduce its legal risk.
}
As a sequence of English letters, a proper name and a protected mark of trade, the origin of the word "Tetris" dates back to the creation and distribution of the similarly-titled game. Pajitnov formed the sign from a portmanteau of "tetra" and "tennis," compactly signifying two of the game's fundamental components: tetrominoes and kinetic recreation.

In accordance with the post-facto commodification of the game, the word originally appeared as the signifying name of a newly invented game, not the brand name of a commercial product. Spreading "like a wildfire" through Moscow and Eastern Europe \autocite[301]{Sheff1993}, the game of Tetris (as executable machine code), word of the game Tetris (as the interesting, original concept of a tetromino-stacking computer game), and the word \emph{Tetris} (as a spoken and written word, program filename and intro screen title) all spread in conjunction. As hackers and hobbyists produced their own copies, ports, variations and translations of Tetris on other systems and devices, many adopted the name \emph{Tetris} for their derived works, in order to maintain ties to the game rules that were being cited, adapted or remixed.

As already evident in these specters of freely-copied executable code and unofficial derivative works that mark the game's repressed prehistory, there is a disconnect between how the public generally understands a name like Tetris to refer to the abstract rules or play of a particular form of game regardless of its brand-name packaging, and how the trademark-function of the word \emph{Tetris} ideally distinguishes a specific source or manufacturer of the game from any other versions of the game. The difficulty in distinguishing the Tetris brand from the game product through an unwieldy moniker such as "Tetris-brand tetromino-stacking computer puzzle game" demonstrates this disconnect.

Complicating this is the fact that The Tetris Company has actively policed public use of the word in its descriptive sense when affixed to unlicensed products. Software programs with "Tetris," "Tetri-," or even just "-tris" in their titles (including one iPhone game literally named "Tris") have been routinely sent cease and desist letters, and many have even been formally sued for trademark infringement, to the point that wary developers learn to avoid using any related word at all in their tetromino-stacking games for fear of litigious reproach \autocites[see][]{DeMocker1998}{Witherspoon2008}.

The motivation for such policing comes from the legal doctrine of trademark dilution, which claims that a brand's property right requires its proprietor to prevent others from "diluting" the "goodwill" of the brand by associating it with inferior or unauthorized products. In theory, such a doctrine is intended to protect a brand's arbitrary sign from misuse; however, in practice, the doctrine tends to legitimize the displacement of the generic product identifier by a trademarked brand name in the consumers' consciousness, effectively monopolizing the market for the product. In the case of Tetris, since consumers know the game by no other name, it is impossible for them to find it other than by searching for The Tetris Company's policed trademark, effectively eliminating any competing products.

Coombe has noted that the doctrine of dilution gained increasing legal acceptance in the 1970s and 1980s along with the increasingly hegemonic commodification of consumer consciousness:
\blockquote{
  The trademark owner is invested with authorship and paternity; seen to invest >sweat of the brow< to >create< value in a mark, he is then legitimately able to >reap what he has sown.< The imaginations of consumers become the field in which the owner sows his seed---a receptive and nurturing space for parturition---but consumers are not acknowledged as active and generative agents in the procreation of meaning. \autocite*[71]{Coombe1998-yv}
}
The protection of words firmly embedded in consumer's largely-unconscious habits of product identification, then, acts as a legally-sanctioned form of >mind control<: "By controlling the sign, trademark holders are enabled to control its connotations and potentially curtail many forms of social commentary" \autocite[73]{Coombe1998-yv}.

Supported by enforcement of dilution claims, public perceptions of a trademark's authorizing significance often become a self-enforcing, self-fulfilling prophecy. As long as the public recognizes a difference between >official< Tetris-licensed games and unauthorized >clones< not named Tetris, while conflating the mark of authenticity with the mark of identification, public opinion is doubly leveraged to not just mis-recognize, but actively silence, unauthorized derivative productions through their lack of access to the only proper name that could address them. These illegitimate copies are always merely Tetris-\emph{like}---not authorized to be Tetris®, and yet, only ever recognizable as its invisible Other.

Under the doctrine of dilution, there is a clear economic incentive to combining product and brand into a single sign, as it leverages the legal system to prevent competitors from being able to siphon away consumers with better versions of a similar product. However, the expanding legal doctrine of dilution is kept in check by the competing doctrine of genericness. This doctrine can cause a trademark to be officially released into the public domain, a fate that claimed such former brand names as \emph{thermos}, \emph{escalator}, and \emph{aspirin}, once the public unequivocally recognized them as product.\footnote{
  This same fate loomed over the Monopoly brand in a trademark lawsuit where the game's name was judged to be generic in 1982. Parker Brothers managed a settlement in 1985, before the judgment to officially cancel its valuable trademark was formally concluded \autocites{Hollie1983-nc}[120--5]{Orbanes2006-tn}.
}
Likewise, the Tetris brand is perpetually haunted by the generic meaning of the game's name in the public consciousness, which threatens to release the name into the public domain at any moment. In order to protect and enforce the state-granted monopoly inherent in the property-function of its authorized name, Tetris Holding must assert a distinction between the distinguished significance of its brand name and the generic game algorithm to which it unmistakably refers. The name of the game, stripped of its logos, uniform colors and distinctive packaging down to the bare algorithm also known as nothing other than \emph{Tetris}, thus threatens to dissolve the corporation's linguistic boundaries erected upon the same name.

Trade dress is related to trademark in that that it recognizes rights to control the usage of a sign associated with a brand, but trade dress protects a more general "look and feel" of a product rather than a word mark or logo. In Tetris v. Xio, Tetris Holding successfully argued that Mino infringed upon its trade dress rights in "the brightly-colored Tetriminos, which are formed by four equally-sized, delineated blocks, and the long vertical rectangle playfield, which is higher than wide" \autocite[415]{tetris-xio}. Although the discussion of trade dress was less central to the case's judgment than that of copyright infringement and the arguments regarding the functionality of the game ran largely parallel in both discussions, it is important to note that through trade dress protections, Tetris Holding is able to doubly secure the protected status of its game as simultaneously brand and creative expression.

\subsection*{Copyright: Algorithm as object}
The economy of granting monopoly protection to works of creative expression, as codified in copyright law, has historically developed through a liberal discourse which takes such protection to be a natural right akin to property. Coombe, among many other voices in critical legal studies, has suggested that one of the strongest rhetorical forces driving the increasing protections afforded to intellectual property over the past century has been the Romantic individualist notion of the author as singular, creative genius, and the concept of property as a fundamental, inviolable right naturally granted to the autonomous, liberal subject: "The idea of an author's rights to control his expressive creations developed in a context that privileged a Lockean theory of the origin of property in labor in which the expressive creation is seen as an authorial >work< that creates an >Original< arising spontaneously from the vital root of >Genius<" \autocite*[219]{Coombe1998-yv}.\footnote{
  Coombe also notes, "Critical legal scholars have written extensively about the inadequacies of Romantic individualism and its understanding of subjectivity, cultural agency, freedom of speech, and creativity" \autocite*[212]{Coombe1998-yv}.
}
From these forces, one can imagine the fullest extent of a discourse formed by the metaphor of creativity as property: property rights would attach to an idea immediately and automatically upon conception, last perpetually, and no one is allowed to reproduce or reuse another's idea without permission, without exception. This metaphor treats the idea-object as a resource of limited quantity produced by a singular author through creative, material labor, such as a hand-crafted work of art, that provides the greatest benefit to society when the original owner is entitled to recoup his investment by fully exploiting the object's value through free and unlimited exchange on the global commodity marketplace. 

This ideal of intellectual property as a natural right that would authorize a pure, authoritarian notion of proprietary discourse has always been kept in balance against the principle of freedom of expression. If the transmission or creative use of any idea whatsoever first required its original author's permission, the proprietors of thoughts or ideas that were so stock or commonplace that they had become a habitual or even necessary cultural component of any future creative or political expression would be granted a state-authorized power to arbitrarily censor or tax helpless individuals. Copyright law is therefore viewed as a balance struck between the benefit to society that occurs when ideas are freely usable by all, and the benefit that strong intellectual property rights provides to individuals as an incentive for future investment of creative labor and materials.

These productive tensions between proprietary idea and freedom of expression have become codified in digital copyright law through what are known as the idea-expression and abstraction-filtration-comparison doctrines, both of which were discussed in the \emph{Tetris v. Xio} opinion. These doctrines both begin with the 1879 case \citetitle{baker-selden} as their inspiration, where the Supreme Court outlined a sharp distinction between copyrightable works of authorship and patentable "useful arts": "The description of the art in a book, though entitled to the benefit of copyright, lays no foundation for an exclusive claim to the art itself. The object of one is explanation; the object of the other is use. The former may be secured by copyright. The latter can only be secured, if it can be secured at all, by letters-patent" \autocite*[105]{baker-selden}. This influential Opinion explicitly distinguished between works of functional and aesthetic works, limiting its discussion to the former while avoiding any judgment on the latter:
\blockcquote[103--4]{baker-selden}{
  Of course, these observations are not intended to apply to ornamental designs, or pictorial illustrations addressed to the taste. Of these it may be said, that their form is their essence and their object, the production of pleasure in their contemplation. This is their final end. They are as much the product of genius and the result of composition, as are the lines of the poet or the historian's periods. On the other hand, the teachings of science and the rules and methods of useful art have their final end in application and use.
}
Thus, a suppressed ontological distinction between "fun" and "function" has deep roots in copyright law, one which is upheld in its application to digital intellectual property. The idea-expression doctrine derives from \citetitle{whelan-jaslow}, which condensed the analysis from \emph{Baker} into a simple test: "the purpose or function of a utilitarian work would be the work's idea, and everything that is not necessary to that purpose or function would be part of the expression of the idea" \autocite*[1236]{whelan-jaslow}. The abstraction-filtration-comparison doctrine comes from a 1992 court case, \citetitle{altai}, which criticized the Whelan test for relying "too heavily on metaphysical distinctions" and its "outdated appreciation of computer science" \autocite[706]{altai}, proposing its own more pragmatic test designed specifically for the structure of computer programs. This test involves three steps: first, "break down the allegedly infringed program into its constituent structural parts"; next, "sift out all non-protectible material"; finally, "left with a kernel, or possible kernels, of creative expression,…compare this material with the structure of an allegedly infringing program" \autocite[706]{altai}.

Despite such a seemingly rigid, programmatic doctrine for determining copyright infringement in computer programs, in practice, the process of sifting down to this "golden nugget" is exceedingly difficult and arbitrary. In particular, both of these doctrines followed \emph{Baker}'s suspension of judgement on aesthetic works, and only dealt with computer programs that served a clearly-identifiable business function, an element which both opinions saw as essential to the proper application of their methods:

\blockcquote[1238]{whelan-jaslow}{
  The rule has its greatest force in the analysis of utilitarian or >functional< works, for the purpose of such works is easily stated and identified. By contrast, in cases involving works of literature or >non-functional< visual representations, defining the purpose of the work may be difficult. Since it may be impossible to discuss the purpose or function of a novel, poem, sculpture or painting, the rule may have little or no application to cases involving such works.
}

\begin{quoting}
  The first step in this procedure is to identify a program's ultimate function or purpose. An example of such an ultimate purpose might be the creation and maintenance of a business ledger. \autocite[697]{altai}
\end{quoting}
For \emph{Tetris v. Xio}, the question of the appropriate application of these doctrines was definitively posed: which elements of a copyrighted videogame comprise its unprotected, functional ideas, and which are its protected, aesthetic expression?

The two opponents offered conflicting interpretations. Xio argued, citing Juul's theory of games as a hybrid of game rules and game fiction, that a game's unprotected ideas could be demarcated by a formal analysis of the "limitations and affordances" produced by the game rules. If a game element specifies a limitation or affordance that comprises a substantial component of the play experience, then it should be considered an unprotected idea. If a game element is primarily decorative, narrative or thematic in nature, only marginally affecting the algorithmic "core" of a play experience, it could be considered a protected expression. This interpretation stems from ludology's essentialist separation of game elements from story elements discussed in the first section of this chapter. It holds that as a paradigmatic abstract game, the basic rules of Tetris contain no verbal story elements, and therefore nothing essential to its operation constitutes protected creative expression.

Tetris Holding, however, disputed this ludological mapping of game rules/fiction to the legal concepts of idea/expression. Serving as an expert witness in support of The Tetris Company, Bogost argued in favor of a broad, vague determination of the idea and rules of Tetris:
\blockquote{
  The idea of Tetris is that of a game with blocks on the screen, which are assembled into specific shapes and manipulated by the player. The rules of Tetris are that an object appears on the playing field and the player manipulates the object to a final resting spot, to create a shape, which is then removed from the playing field. \autocite*[5--6]{bogost-tetris}
}
As Xio argued in response, Bogost's rules are so vague that they are completely divorced from any conceivable understanding of the rules of Tetris, and "would apply to games that look nothing like Tetris, like Connect Four, and arguably even checkers and chess" \autocite[31]{xio-summary}. In his court testimony, Bogost explained that the "rules" he specified do not necessarily contradict Juul's definition of game rules, but are merely on a "higher level of abstraction" from them: "There are many ways of understanding rules. In fact, many have illusions of different understandings of rules. To me, [Juul's] definition of rules sits at a level lower than my understanding of rules. This is a discussion of the way rules might be interpreted rather than rules themselves" \autocite[172.17--19]{Bogost-deposition}.

No argument was offered as an explanation of why other game elements were absent from this account of the rules of Tetris, or upon what analytical basis this "higher level of abstraction" is more appropriate than Juul's. The playing field dimensions and the specific movements of playing pieces, for example, are among the most standardized and regulated components of the rules of chess and checkers, as they are in most board games. Why should those elements remain absent from the rules of Tetris, other than as a \emph{post hoc} self-legitimation of the designer's protected, creative expression?

This methodical silence is repeated in the judge's opinion, who conceded that there was little on which to ground any such distinction: "While the unenviable task of dissecting a game's ideas from its expression is difficult, I am guided by case law and common sense, and find that the ideas underlying Tetris can be delineated by understanding the game at an abstract level and the concepts that drive the game" \autocite[408]{tetris-xio}. While refusing "to articulate a rigid, specific definition" of game rules, the judge nonetheless offered an account of the "general, abstract ideas" of Tetris:
\blockcquote[409]{tetris-xio}{
  Tetris is a puzzle game where a user manipulates pieces composed of square blocks, each made into a different geometric shape, that fall from the top of the game board to the bottom where the pieces accumulate. The user is given a new piece after the current one reaches the bottom of the available game space. While a piece is falling, the user rotates it in order to fit it in with the accumulated pieces. The object of the puzzle is to fill all spaces along a horizontal line. If that is accomplished, the line is erased, points are earned, and more of the game board is available for play. But if the pieces accumulate and reach the top of the screen, then the game is over. These then are the general, abstract ideas underlying Tetris and cannot be protected by copyright nor can expressive elements that are inseparable from them.
  }
While slightly more specific than Bogost's definition of the rules of Tetris, the conspicuous absence of a 10-by-20 playing field and one-sided tetrominoes from this description of the ideas underlying Tetris was enough to render a judgment of copyright infringement. In my analysis, I claim that this conspicuous absence is only made possible by the latent distinction between fun and function underlying copyright law since \emph{Baker} that has allowed creations "addressed to the taste" and designed only for the "production of pleasure" to be excluded from limitations against copyright protection applied to other explicitly purposeful, functional works.

\subsection*{The integrated cultural commodity}
In summary, the \emph{Tetris v. Xio} case serves as one of the most illuminating documents on the present state of the juridical commodification of the videogame medium. Tetris appears here not only as a canonical, paradigmatic abstract game, but as an integrated formation of trademark, trade dress, and copyright protections that have each expanded to accommodate the novel commodity forms being produced by the contemporary game industry. The result of these new concessions within the case-law doctrine of intellectual property is the transformative commodification of game-playing itself, from a socially-oriented, community-forming activity into an integrated, legally-protected, proprietary commodity. Prior to the game industry's aggressive expansion of the game into a commodity form, the concept of a game was as an oral, dialogic activity, with rules continuously and freely negotiated among players even if written down.

Within the field of game studies, \citeauthor{Bogost2012-wr} recognized the unique relation videogame production maintains with intellectual property law, and suggested that effective game criticism must become cognizant of this function:
\blockquote{
  Unlike psychoanalysis or literary theory, IP |intellectual property| is a stable relationship regulated by governments and markets instead of critics. The rules of IP are flexible and may change, but its fundamental principle is legal, not literary.…Intellectual property relations can be modified and interpreted by law, and effective criticism of games as cultural works may need to take the licensing operation into account in understanding how a work functions discursively. If there was ever any doubt about the political economy of works of art, game engines end that doubt. \autocite*[61-2]{Bogost2006-ec}
}
While I agree with Bogost that the transformation of intellectual property law and licensing relations as discussed above are a crucial component of "effective criticism of games as cultural works," this still begs the question: effective for whom? Bogost's own political-economic game criticism of Tetris amounted to a reinscription of the figure of the Romantic author of individual genius and unbounded creativity in the form of the contemporary game designer. His expert report successfully advocated a broad expansion of our contemporary system of intellectual property rights into even the most abstract form of procedural expression, which was certainly effective for The Tetris Company's continued proprietorship over its lucrative game and brand. In contrast, my own criticism of Tetris developed through this chapter speaks not on behalf of the powerful, valuable game-brands already dominating our social landscape, but instead in behalf of a broader public interest in the ubiquitous cultural commodities through which meanings and identities are constructed, an interest which is being continually eroded and exploited through the ongoing commodification of play, of which Tetris is one of the most powerful representatives.

The intrinsic and extrinsic interpretations of Tetris-space exhibit an intriguing allegorical symmetry, both representing a process that encloses a continuous, fluid, smooth space into a discrete, bounded, striated system. The intrinsic allegory relates the embodied manipulation of physical objects to the digital manipulation of a computational simulation; the extrinsic allegory relates the shared imagination and evolving languages, games and symbols of mass culture to the intellectual property regime's violent boundary construction and regulation of a proliferation of fixed cultural commodities designed for functionless consumption.

I will conclude by sketching out two thematic avenues for productive public engagement in full view of our particular cartography of the commodity form. First, there is the prospect of gradual cultural-institutional reform within the normative legal discourse itself, following the work within the field of critical legal studies that supplements a critique of the autonomous liberal subject with a progressive call for recognizing limitations to individual property rights with a recognition of public cultural interests. Second, there is the possibility of ethically and politically motivated cultural work to preserve and promote popular social histories of games, more readily emphasizing public contributions, engagements, and controversies that would otherwise have been suppressed by the workings of ludocapitalism.

\section{Conclusions}
\subsection*{Play as Everyday Practice}
One of the great ironies of the emergence of globally-commodified video games as the paradigmatic play form is that the conspicuous accumulation of constraints and restrictions on free-play with its cultural artifacts is a direct result of the increasing expansion of its commodity form. Through the expansion of legal arguments such as those that found in \emph{Tetris v. Xio} that present the videogame as a non-functional entertainment commodity, the subject of popular videogame culture is interpellated into a \emph{gamer}, a politically-impotent, fun-seeking, consuming recipient of legally and technologically guarded, untouchable game-objects, where any creative potential latent in the consumer-gamer's unique or individual experience, any room for the everyday practice of play, is reduced.

Having articulated the legal hegemony underlying the videogame as commodity form, I will next outline some potentially productive directions for institutional resistance and reform that I find particularly applicable. First, I find Michel de Certeau's concept of everyday practice to be an inspirational model of active resistance from within consumer culture. In his study of how the everyday practices of "users" and "consumers" of popular culture operate, \citeauthor{DeCerteau1984} recognized and encouraged the processes through which "everyday life invents itself by poaching in countless ways on the property of others" \autocite[xii]{DeCerteau1984}. Taking issue with Alvin Toffler's enthusiasm for the nomadic "self mobility" of a ">new species< of humanity," de Certeau lamented that under the conditions of commodified consumption, "instead of an increasing nomadism, we thus find a >reduction< and a confinement: consumption, organized by this expansionist grid takes on the appearance of something done by sheep progressively immobilized,…such an image of consumers is unacceptable" \autocite[165--6]{DeCerteau1984}. This passive image misunderstands the act of "consumption" as ">becoming similar to< what one absorbs, and not >making something similar< to what one is, making it one's own, appropriating or reappropriating it" \autocite[166]{DeCerteau1984}. This emancipatory figure of the reader is one characterized by "advances and retreats, tactics and games played with the text,…playful, protesting, fugitive" \autocite[175]{DeCerteau1984}.

Next, I find Henry Jenkins's interpretation of de Certeau's figure of the playful reader as "textual poacher" in the context of television fan culture to be a particularly applicable form of populist resistance to cases of commodified cultural hegemony similar to Tetris. In \citetitle{Jenkins2012-wc}, \citeauthor{Jenkins2012-wc} argues that fandom can create its own "particular forms of cultural production, aesthetic traditions and practices" that "appropriate raw materials from the commercial culture but use them as the basis for the creation of a contemporary folk culture" \autocite*[279]{Jenkins2012-wc}. The legitimacy that Jenkins strives to offer unofficial fandom gestures tentatively to a basis for a populist challenge to the copyright regime: "The nature of fan creation challenges the media industry's claims to hold copyrights on popular narratives. Once television characters enter into a broader circulation, intrude into our living rooms, pervade the fabric of our society, they belong to their audience and not simply to the artists who originated them" \autocite[279]{Jenkins2012-wc}. This argument dovetails with my above discussion of aura for its focus on the subject's involuntary associations with the object: to the extent that any authored object is interiorized by the consumer and integrated into public forms of cultural expression, property rights of the original author should cease. Such a celebration of fan culture is, if not a sustainable argument for the legal reform of copyright laws, at least a legitimate recognition and moral defense of an alternative model of audience-oriented, participatory ownership of cultural symbols under conditions of mass communication.

For more specific suggestions on intellectual property policy reform, I find Julie Cohen's proposals in \citetitle{Cohen2012} to be well-balanced and particularly applicable to the dialectics of play underlying the paradigmatic commodity form of the videogame discussed in this chapter and the cultural conditions of ludocapitalism more generally. \citeauthor{Cohen2012} draws a connection between the cultural and economic configurations that modern intellectual property law is designed to construct and the potentially transformative power of the play of everyday practice, articulating the two in an incremental blueprint for information policy reform. She calls for replacing the autonomous, rational liberal human subject with the play of everyday practice as "the means by which human beings flourish" \autocite[56]{Cohen2012}, an ambiguous but crucial component of creative practice that is becoming lost in our legal transition toward the totalization of intellectual property protection of cultural objects:
\blockcquote[55]{Cohen2012}{
  In the ongoing dialectic between ad hoc, reactive tactics and situated creativity, the most salient aspect is not one or the other, but the continual interplay between them. Play's ambiguous status---shaped by cultural constraints, but not wholly dictated by them---is the source of its potentially transformative power.
  }
Culminating in a set of concrete proposals for progressive intellectual property reform, Cohen argues in favor of reevaluating the balance of interests underlying intellectual property in a manner that doesn't automatically endorse the neoliberal tendency to enclose the entire space of cultural activity within a rationalized grid of property rights, systematically privileging and perpetuating an industrialized cultural hegemony. Instead, she advocates implementing a "cultural environmentalism" that can identify value in the play of everyday practice, providing an analytical "breathing room" that would preserve "interstitial complexity" within the analytical spaces that identify distinct creative and innovative works as protectible commodities \autocite[248]{Cohen2012}. In other words: "the system of copyright requires the deliberate introduction and maintenance of legal and institutional discontinuities that shelter cultural play" \autocite[240]{Cohen2012}.

With respect to games, such a greater cultivation of spaces and forms of semantic discontinuity would give long-deserved recognition to emergent practices and cultures of game play as a welcome supplement to the liberal author-figure of game design. Although such play practices have always existed, they have been made invisible from the object-perspective of autonomous authorship embodied in intellectual property law. In the case of Tetris, the few months of part-time work that Pajitnov reportedly invested into the original Tetris prototype is dwarfed by the collective engagement of hundreds of millions of players over the past quarter century and their collective material involvement in shaping the norms and rules of the game into what it is today, and this disparity should not be discounted or ignored when it comes to granting the right to enjoin follow-on creative work.

Although commodified-public spaces such as those cultivated by Creative Commons and the GNU General Public License\footnote{
  Coincidentally, the name "GNU" is itself an unauthorized tactical citation of a widely-adopted proprietary system: "It turns out that classic FOSS >hacker wordplay< names like >GNU< are actually an excellent way to avoid possible trademark infringement claims. Since >GNU< itself does not resemble the word >Unix< at all, and since when expanded it explicitly tells the reader that the product is \textbf{not} Unix (i.e., >Gnu's Not Unix<), a potential trademark holder on the term >Unix< would be hard pressed to make the case that consumers would be confused and think that GNU really is Unix" \autocite[5.8n2]{SFLC2008}.
}
provide legitimate alternative paradigms of creative coordination that confront and liberate the privatization of digital culture and infrastructure, it is also the case that "copyleft" models tend to implicitly sanction rather than resist the further legal commodification of culture, leading in the end to a greater acceptance of and affinity with the modern intellectual property regime that leaves the surface of culture without those porous, flexible spaces of "breathing room" that Cohen demands for the play of everyday practice. Even if the creation of a commons-friendly game-design practice where works would be encouraged to be published with blanket open-content licenses permitting free reuse of any intellectual property were to emerge, it would do little to challenge the hegemony of proprietary commodities embedded in popular cultural consciousness. Although tactical maneuvers to create such intellectual sanctuaries within the confines of the existing intellectual property regime are nonetheless welcome and encouraged, I believe that the cultivation of sustained, productive discourse on the ethical balance of property rights in the conditions of popular culture is equally important.

\subsection*{Toward a Hauntology of Tetris}
Careful attention to everyday practices of play thus reveals that the most powerful, threatening, potentially transformative specter within a hauntology of Tetris is the specter of play itself. In the conditions of post-Fordist commodification of videogame objects, The subject's freedom to play with the game is erased, leaving the mere ability to consume the game exactly as produced. However, the creative, potentially transformative power of the play of everyday practice beckons through the subject's own assimilation of the game to their own social-technical identity. The playing subject desires not only to play again and again in repetition as proper consumption, but to play differently, to tinker with, extend and explore the structure of the game itself, to produce something recognizably different---a port to a new digital platform, an extra feature, a slight tweak or customization, a set of "house rules" that signify local tastes or customs. The tensions that produce Tetris as a protected object of property also produce the haunting of an unfulfilled play-desire.

I have suggested throughout this chapter that Tetris, as paradigm of the ideal-type videogame commodity, is haunted by elements of culture and society that remain on the periphery of dominant liberal-humanist ideals of authorship and property. From Gerasimov's suppressed contributions of free labor on the original Tetris prototypes, to folk Tetris crafts reappropriated as viral brand marketing, to proprietary claims over basic spatial movements lodged deep within players' dreams and consciousness of algorithmic gamespace, this history of Tetris provokes tensions and co-dependencies between the market-enabling conditions of property ownership and the communicative conditions of free play.

As a contribution to the discourse of intellectual property, I encourage an ongoing critical account of "spectral" elements of culture within the intellectual properties that comprise our contemporary lives to draw our attention to the memory of those silenced, devalued, replaced or excluded from cultural speech within the technoliberal marketplaces that constitute our digital environments. In \citetitle{Derrida1994-ii}, \citeauthor{Derrida1994-ii} presents a deconstruction of Marx's figure of the specter within the Communist Manifesto and other works as a substantial critique of ontology as presence, and an exploration of its formative boundaries through provocatively manifesting the non-presence of its other:
\blockcquote[10]{Derrida1994-ii}{
  Altogether other. Staging for the end of history. Let us call it a \emph{hauntology}. This logic of haunting would not be merely larger and more powerful than an ontology or a thinking of Being.…It would harbor within itself, but like circumscribed places or particular effects, eschatology and teleology themselves. It would comprehend them, but incomprehensibly.
}
Applying this "logic of haunting" to the meaning, history, and properties of Tetris discussed in the chapter, I construct a political orientation from the margins of the ontological form of the videogame commodity itself. Through the ghostly figures that Derrida found in Marx's Communist Manifesto a call for the "spectre of communism" haunting Europe to be manifested into a living reality of proletarian revolution: "the essence of the political will always have the inessential figure, the very anessence of a ghost" \autocite[127]{Derrida1994-ii}. To form such an ethical-political relation to new media objects, we can be attentive to marginal figures and fantasies that haunt the hallowed grounds of authoritative properties, and bring to consciousness those marks and works of desire, alterity and play that have been and will be silenced by the overwhelming presence of the proprietary object and juridical-political tactics deployed to maintain its legitimacy and singularity.

It is through this figure of the specter that I find a voice in the silenced expressions of popular imaginations of Tetris in the forms of ports, hacks, clones, derivations and homages. Beyond functioning as an allegorical simulation of digital space, I have focused my case study of Tetris on its particular construction as a commodified, legally-protected property composed of brand and software, resting on an ideology of "fun" as expression of creative genius entitled to expansive legal protection that is uniquely pervasive in our era of post-Fordist ludocapitalism. I have linked proposals for intellectual property reform to a revaluation of the videogame commodity that recognizes cultural activity haunting the property's juridical-political contours, speaking against the monologic, idealized voice of the designer-author-proprietor through the actively forgotten, silenced, illegitimate, polyphonous voices of hackers, bootleggers, entrepreneurs, pirates, developers, hobbyists, and players who all perform their own "work" on, with and through the object in focus and on display.

In 2007, a panel of game academics and industry veterans chaired by Henry Lowood voted to include Tetris in their proposal for a "Digital Game Canon" of the "10 most important video games of all time" \autocite{Chaplin2008}. The goal of the canon, modeled after the work of the National Film Preservation Board, was "an assertion that digital games have a cultural significance and a historical significance," and "a way of saying, this is the stuff we have to protect first." While not opposed to such a model of preservation and archival, the aim of a game criticism modeled instead on a logic of haunting is for the public politicization of such neutral archival discourse, transforming its mission from consensus-oriented preservation of the most successful or influential creations into an active work of mourning, ontologizing and identifying the remains of those buried cultural works not creative enough to survive the business-legal logic of ludocapital. Such work of mourning counters the narrative of the singular, original author of culture representative of >the amnesiac order of capitalist bourgeoisie (the one that lives, like an animal, on the forgetting of ghosts)…< \autocite[139]{Derrida1994-ii}. This hauntological response to the ontological question gestures beyond the liberal humanist lineage of property, authorship and commodity formation to resurrect unexpected, surprising material from the unmarked void of technoculture, actively transforming rather than passively preserving the meaning of the canonical object on display. Instead of a cultural memory that merely perpetuates the Lockean ideal of "preservation of property," the alternative, hauntological form of historical work with which I have followed Tetris in this chapter reconstructs the object as symptomatic of its environment, presents a cognitive mapping of its particular regime of capitalist accumulation and makes analysis and criticism of this broader social formation possible.

I have studied in this chapter, through the example of Tetris, how the digital game object has undergone transformation into a commodified, legally-protected integration of authored product and marketed brand, a transformation paradigmatic of post-Fordist ludocapitalism and of the videogame industry that has been both driving force and benefactor of these changes. Beyond the legal consolidation and protection of creative properties as discussed in this chapter, another aspect of industrial videogame and software production that maintains cultural hegemony is the expanding technical gap between producers and consumers maintained by the high cost, restricted access, and other social-economic difficulties in acquiring the technical literacies necessary for producing creative work within new media platforms. I therefore shift in the next chapter toward a more concrete, material layer of digital game and software development, to a critical analysis of the computing technologies and programming languages that ground digital media in the industrial practices of software engineering. Here, I track how the liberal-humanist concept of literacy has transformed into the technocultural concept of procedural literacy, and I contrast the complex, expert-oriented technical codes and protocols underlying our vast communications networks with calls for a renewed digital public sphere accountable to a notion of civil society grounded in the languages of everyday life.
% ... and so on

% These commands fix an odd problem in which the bibliography line
% of the Table of Contents shows the wrong page number.
\clearpage
\phantomsection

% "References should be formatted in style most common in discipline",
% abbrv is only a suggestion.
\bibliographystyle{abbrv}
\bibliography{thesis}

% The Thesis Manual says not to include appendix figures and tables in
% the List of Figures and Tables, respectively, so these commands from
% the caption package turn it off from this point onwards. If needed,
% it can be re-enabled later (using list=yes argument).
\captionsetup[figure]{list=no}
\captionsetup[table]{list=no}

% If you have an appendix, it should come after the references.
% The original template (from Trevor) had a custom \appendix command,
% but I found it to break figure/table counters. I'm not sure how
% reliable my fix is, so I ended up reverting back to the standard
% latex version, and renaming the custom command to \myappendix.  You
% can try both and see how things work out:
% 1) Call \appendix once, and then make each appendix a \chapter
% 2) Call \myappendix once, and then make each appendix a \section.

\appendix
\chapter{Appendix Title}

Supplementary material goes here. See for instance Figure
\ref{fig:quote}.

\section{Lorem Ipsum}

dolor sit amet, consectetur adipisicing elit, sed do eiusmod tempor
incididunt ut labore et dolore magna aliqua. Ut enim ad minim veniam,
quis nostrud exercitation ullamco laboris nisi ut aliquip ex ea
commodo consequat. Duis aute irure dolor in reprehenderit in voluptate
velit esse cillum dolore eu fugiat nulla pariatur. Excepteur sint
occaecat cupidatat non proident, sunt in culpa qui officia deserunt
mollit anim id est laborum.

\begin{figure}
  \centering
  \begin{tabular}{l}
    ``I am glad I was up so late,\\
    \quad{}for that's the reason I was up so early.''\\
    \em \footnotesize William Shakespeare (1564-1616), British
    dramatist, poet.\\
    \em \footnotesize Cloten, in Cymbeline, act 2, sc. 3, l. 33-4.
  \end{tabular}
  \caption{A deep quote.}
  \label{fig:quote}
\end{figure}


%%% Local Variables: ***
%%% mode: latex ***
%%% TeX-master: "thesis.tex" ***
%%% End: ***


\end{document}
